%-----------------------------------------------------------------------------
%  Copyright (C) 2004-2017 Andrew Mathas, University of Sydney
%
%  Distributed under the terms of the GNU General Public License (GPL)
%                  http://www.gnu.org/licenses/
%
% This file is part of the MathQuiz system.
%
% <Andrew.Mathas@sydney.edu.au>
%-----------------------------------------------------------------------------

%% This file is part of the MathQuiz class distribution
\documentclass[svgnames]{mathquiz}
\usepackage{mathquiz-doc}
\usepackage{pst-plot}
\usepackage{hyperref}

\usepackage{ifpdf}
\ifpdf
  \PackageError{MathQuiz}{This file must be compiled using latex not pdflatex}
\fi
\newcommand{\C}{\mathbb C}
\newcommand*{\vect}[1]{\mathbf{#1}}
\UnitCode{MathQuiz}

\UnitName{MathQuiz Manual}
\UnitURL{/u/MOW/MathQuiz/doc/mathquiz-manual.html}
\BreadCrumb{MathQuiz on-line manual}
\University{University of Sydney}
\Department{School of Mathematics and Statistics}
\QuizzesURL{MOW/MathQuiz/doc/}

\hypersetup{pdftitle={MathQuiz on-line manual}}

\newcommand\qref[1]{%
  \ifdefined\HCode\relax
     \HCode{<a href="\#" onclick="gotoQuestion(#1)">Question #1</a>}
  \else\hyperref[question#1]{Question~\ref*{question#1}}%
  \fi%
}

\title{MathQuiz: Web Quizzes using LaTeX}
\begin{document}

\begin{discussion}[Introduction]
  \MathQuiz{} is a \LaTeX{} based system to create \textit{interactive}
  web quizzes.  The idea is that you write the quiz using \LaTeX{} and that
  MathQuiz creates the web page from this file. Anything which you can
  write using \LaTeX{} will be converted to HTML by MathQuiz. This
  includes, for example, mathematics and graphics written using
  PSTricks.  \MathQuiz{} supports three different types of quiz
  questions:

  \begin{enumerate}
  \item Multiple choice questions with a \emph{unique} correct answer.
 (See \qref{1})
  \item Multiple choice questions with \emph{several} (or no)
    correct answers.
 (See \qref{2})
  \item Questions which require a \emph{numerical} answer.
 (See \qref{3})
  \end{enumerate}
The use of \MathQuiz{} is described in the next section. Later
sections describe how each of the \MathQuiz{} environments are used.

The \verb|discussion| environment in \MathQuiz{} can also be used to
write Web Pages like this one (The pages you are reading here were
written using \MathQuiz.)
\end{discussion}
\begin{discussion}[Basic Usage]
Once you have a \MathQuiz{} file, you can run it through \LaTeX, in
the usual way, to produce a readable version of your quiz. When you
are happy with the quiz, use \verb|mathquiz| to create the HTML
version. Note that the printable version of the quiz does \emph{not}
look like the web page; rather, it contains all of the information in
an easily readable layout.

  If, for example, your quiz file is called \emph{quiz1.tex} then you
  can use the following commands:
\begin{latexcode}
      > latex quiz1                % latex a quiz file
      > pdflatex quiz1             % a PDF versio of the quiz
      > xdvi quiz1                 % view the quiz using xdvi
      > dvips quiz1                % print the quiz
      > mathquiz quiz1             % converts the quiz to html
\end{latexcode}
  Converting the quiz to html can take quite a long time, particularly
  if a large number of images need to be created.

\end{discussion}
\begin{discussion}[MathQuiz files]

  The basic structure of a \MathQuiz{} file is as follows:
\begin{latexcode}
      \documentclass{mathquiz}

      \title{Quiz 1: Complex numbers}

      \UnitCode{MATH1001}
      \UnitName{Differential Calculus}
      \UnitURL{/u/UG/JM/MATH1001/}
      \QuizzesURL{/u/UG/JM/MATH1001/Quizzes/}

      \begin{document}

      \begin{discussion}[optional title] % optional "discussion"
          . . .
      \end{discussion}

      \begin{question} % question 1
          . . .
      \end{question}

      \begin{question} % question 2
          . . .
      \end{question}
      .
      .
      \end{document}
\end{latexcode}
  The preamble of the \LaTeX{} file specifies the unit code, the
  name of the unit of study, the location of the homepage for the
  unit and the index file for the quizzes for this unit; this is
  done using the commands
  \verb|\UnitCode|,
  \verb|\UnitName|,
  \verb|\UnitURL| and
  \verb|\QuizzesURL| respectively. If the
  command \verb|\QuizzesURL| is omitted then the URL for the quiz
  index file is set to \verb|\UnitURL/|Quizzes. (Within the School of
  Mathematics and Statistics, there is also a package
  \href{smsquiz.html}{smsquiz.sty} which sets the default
  parameters from the unit code.)

  The title of the quiz should also be set in the preamble using the
  \verb|\title| command. Note that the \verb|\title|
  command \emph{must} appear before the \verb|\begin{document}| command.
  In the preamble you can also define macros and load
  any packages that you want in the usual way .

  The \verb|discussion| environment can be used to add comments or
  remarks to a quiz. For example, it can be used to summarize the
  material being tested in the quiz or to give references to the
  lecture notes for the unit. The syntax for the \verb|discussion| environment is as
  follows:
\begin{latexcode}
      \begin{discussion}[optional title]
        . . .
      \end{discussion}
\end{latexcode}
  Anything you like (text, mathematics, \ldots) can go inside
  discussion environments.  The \emph{optional title} is used both as
  the section heading on the web page and as the heading in the
  navigation bar which runs down the left hand side of the page. The
  headings should not be too long as they have to fit in the left hand
  margin. By default, the title is \verb|discussion|. There can be
  several \verb|discussion| environment, but they go all in front of the
  questions.

  Questions are set inside a \verb|question| environment. The text is
  followed by the answers.

  \MathQuiz{} supports three types of questions:
  \begin{itemize}
  \item Multiple choice questions with \emph{precisely one} correct
    answer;
  \item Multiple choice questions with \emph{several or no} correct
  answers;
  \item Questions taking a \emph{numerical} answer.
  \end{itemize}
  With each of these types of questions you can (and should) provide
  feedback to the students depending on whether their answer is
  correct or incorrect. Below we describe how to produce these
  different types of questions.
\end{discussion}
\begin{discussion}[Question 1]
  The syntax for a multiple choice question having \emph{precisely
    one} correct answer is as follows:
\begin{latexcode}
    \begin{question}
      . . .question text
      \begin{choice}
        \(in)correct . . . text for (in)correct option
        \response    . . . feedback on response

        \(in)correct . . . text for (in)correct option
        \response    . . . feedback on response
        .
        .
        .
      \end{choice}
    \end{question}
\end{latexcode}
  The multiple choice responses for each question go inside a choice environment.
  The correct answer goes after \verb|\correct|, incorrect answers
  after \verb|\incorrect|. Either can be followed by a
  \verb|\response|, providing comments on the students choice when
  checking their answers.  The \verb|\reponse| commands are optional;
  however, it is recommended that you use them as good feedback is
  very useful for the students.
  \par
  For example, \qref{1} below was typed into \MathQuiz{}
  using the following commands:
\begin{latexcode}
   \begin{question}
     The shaded region in the graph
%     \begin{center}
%       \begin{pspicture}(-3,-1.5)(3,4)
%         \pscircle[linewidth=1pt,linestyle=dashed,%
%                   fillcolor=SkyBlue,fillstyle=solid](1,1){2}
%         \psaxes[linecolor=red,linewidth=1pt,labels=none]%
%         {->}(0,0)(-1.5,-1.5)(3.5,3.5)
%         \rput(3.75,0){$x$}
%         \rput(0,3.85){$iy$}
%         \rput(3,-0.4){3}
%         \rput(-0.4,3){3$i$}
%         \psdots(1,1)
%       \end{pspicture}
%     \end{center}
     is equal to which of the following sets of complex numbers?
     \begin{choice}
       \incorrect $\{z \in \C : (z-1)^{2}+(z-(i+1))^{2}<2\}$
       \response  The equation of a circle in the complex plane with
       centre $a+ib$ and radius $r$ is
       \begin{displaymath}
         \{z\in\C : |z-(a+ib)|<r \}.
       \end{displaymath}

       \incorrect $\{z \in \C : z+(i+1)<2\}$
       \response  You want the set of points whose \textit{distance}
       from $1+i$ is less than $2$.

       \correct   $\{z \in \C : |z-(i+1)|<2\}$
       \response  The graph shows the set of complex numbers whose
       distance from $1+i$ is less than $2$.

       \incorrect $\{z \in \C : |z-2|<|i+1-2|\}$
       \response  As $|i+1-2|=\sqrt 2$, this is the set of complex
       numbers whose distance from $2$ is less than
       $\sqrt 2$.

       \incorrect None of the above.
       \response The graph shows the set of complex numbers whose
       distance from the centre of the circle is less than $2$.
     \end{choice}
   \end{question}
\end{latexcode}
\end{discussion}
\begin{discussion}[Question 2]
 To allow multiple (or no) correct answer we add \verb|multiple| as an
  optional argument to the \verb|choice| environment:
\begin{latexcode}
    \begin{question}
      . . .question text. . .
      \begin{choice}[multiple]
        \(in)correct . . . text for (in)correct option
        \response    . . . feedback on response

        \(in)correct . . . text for (in)correct option
        \response    . . . feedback on response
        .
        .
        .
      \end{choice}
    \end{question}
\end{latexcode}
  The only difference to the previous case is that multiple (or no)
  \verb|\correct| commands may be used.
  \par
  For example, \qref{2} below was typed into MathQuiz
  using the following commands:
\begin{latexcode}
    \begin{question}
      Which of the following numbers are prime?
      \begin{choice}[multiple]
        \incorrect 1
        \response  By definition, a prime is a number greater than 1
        whose only factors are 1 and itself.

        \correct   19
        \response  The only factors of 19 are 1 and itself.

        \incorrect 6
        \response  2 is a factor of 6

        \correct   23
        \response  The only factors of 23 are 1 and itself.

        \correct   191
        \response  The only factors of 191 are 1 and itself.
      \end{choice}
    \end{question}
\end{latexcode}
\end{discussion}
\begin{discussion}[Question 3]

  By default, the \verb|choice| environments puts the multiple choice
  options into two column format. Sometimes, however, the options look
  better when listed in a single column and, sometimes, three or more
  columns are better.  The \verb|choice|
  environment lets you specify the number of columns in the HTML
  version of the quiz.

\begin{latexcode}
    \begin{question}
      . . .question text. . .
      \begin{choice}[multiple, n]     . . . n columns
        \(in)correct . . . text for (in)correct option
        \response    . . . feedback on response

        \(in)correct . . . text for (in)correct option
        \response    . . . feedback on response
        .
        .
        .
      \end{choice}
    \end{question}
\end{latexcode}
  If the optional argument \verb|[multiple]| is not present, then the
  question admits precisely one correct answer.
  \par
  For example, \qref{3} below was typed into \MathQuiz{}
  using the following commands:
\begin{latexcode}
    \begin{question}
      What are suitable parametric equations for this plane curve?
%      \begin{center}
%        \psset{unit=.6cm}
%        \begin{pspicture}(-2.5,-0.5)(5,5.5)
%          \psaxes[linecolor=red,linewidth=1pt,labels=none]%
%          {->}(0,0)(-2.5,-1.5)(5,5)
%          \psellipse[linecolor=blue,linewidth=2pt](1,2)(3,2)
%        \end{pspicture}
%      \end{center}

      \begin{choice}{1}
        \incorrect $x=2\cos t + 1$, $y=3\sin t + 2$
        \response This is an ellipse with centre $(1,2)$ and with axes of
        length $4$ in the $x$-direction and $6$ in the $y$-direction.
%        \begin{center}
%          \psset{unit=.6cm}
%          \begin{pspicture}(-2.5,-0.5)(5,5.5)
%            \psaxes[linecolor=red,linewidth=1pt,labels=none]%
%            {->}(0,0)(-2.5,-1.5)(5,5)
%            \parametricplot[linecolor=blue,linewidth=2pt]{0}{360}%
%            {t cos 2 mul 1 add t sin 3 mul 2 add}
%          \end{pspicture}
%        \end{center}

        \correct $x=3\cos t + 1$, $y=2\sin t + 2$
        \response The curve is an ellipse centre (1,2) with axes length 6
        in the $x$ direction and 4 in the $y$ direction.

        \incorrect $x=3\cos t - 1$, $y=2\sin t - 2$
        \response This is an ellipse with centre $(-1,-2)$ and with axes
        of length $6$ in the $x$-direction and $4$ in the $y$-direction.
%        \begin{center}
%          \psset{unit=.6cm}
%          \begin{pspicture}(-5,-4)(1,2)
%            \psaxes[linecolor=red,linewidth=1pt,labels=none]%
%            {<-}(0,0)(-4.5,-5.5)(1,2)
%            \parametricplot[linecolor=blue,linewidth=2pt]{0}{360}%
%            {t cos 3 mul 1 sub t sin 2 mul 2 sub}
%          \end{pspicture}
%        \end{center}

        \incorrect $x=2\cos t - 1$, $y=3\sin t - 2$
        \response This is an ellipse with centre $(-1,-2)$ and with axes
        of length $4$ in the $x$-direction and $6$ in the $y$-direction.
%        \begin{center}
%          \psset{unit=.6cm}
%          \begin{pspicture}(-4,-5)(1,2)
%            \psaxes[linecolor=red,linewidth=1pt,labels=none]%
%            {<-}(0,0)(-4.5,-5.5)(1,2)
%            \parametricplot[linecolor=blue,linewidth=2pt]{0}{360}%
%            { t cos 2 mul 1 sub t sin 3 mul 2 sub}
%          \end{pspicture}
        \end{center}
      \end{choice}
    \end{question}
\end{latexcode}
\end{discussion}
\begin{discussion}[Question 4]
  The final type of question that \MathQuiz{} supports is a question
  which requires a \emph{numerical} as an answer. The numerical
  answer must be given in decimal form.
  The answer is typeset using the \verb|\answer| macro. That macro
  takes two arguments, some text appearing in an answer box after the
  question and the actual numerical answer. The text is optional. The
  syntax is \verb|\answer[text after answer box]{numerical answer}|.
  Then there is a mechanism for providing feedback for correct and
  incorrect answers. This is done using \verb|\whenRight| and
  \verb|\whenWrong|.  Unlike the \verb|\response| commands, the two
  reponses \verb|\whenRight| and \verb|\whenWrong| are both mandatory
  for questions of this type.  The syntax is for such questions is as
  follows:
\begin{latexcode}
       \begin{question}
         . . .question text. . .
         \answer[text after the answer box]{actual answer}
         \whenRight . . . feedback when right
         \whenWrong . . . feedback when wrong
       \end{question}
\end{latexcode}
  \par
  For example, \qref{4} below was typed into \MathQuiz{}
  using the following commands:
\begin{latexcode}
    \begin{question}
      If the vectors $\vect{a}$ (of magnitude 8 units) and $\vect{b}$
      (of magnitude 3 units) are perpendicular, what is the value
      of
      \begin{displaymath}
        |\vect{a} -2\vect{b}|~?
      \end{displaymath}
      (Hint: Draw a diagram!)

      \answer[units]{10}
      \whenRight The vectors $\vect{a}$, \(-2\vect{b}\), and
      $\vect{a} - 2\vect{b}$ form the sides of a right-angled
      triangle, with sides of length $8$ and $6$ and
      hypotenuse of length $|\vect{a} -2\vect{b}|$. Therefore
      by Pythagoras' Theorem,
      \(|\vect{a} -2\vect{b}|=\sqrt{8^2+6^2}=10\).

      \whenWrong Draw a diagram and then use Pythagoras' theorem.
    \end{question}
\end{latexcode}
\end{discussion}
\begin{discussion}[Index Files]
  \MathQuiz{} also provides a mechanism for creating a web page
  containing an index of all quizzes for a given Unit of Study.
  This is done with a \MathQuiz{} file which contains a |quizlist|
  environment. The syntax for this environment is as follows:
\begin{latexcode}
    \begin{quizlist}
      \quiz[url1]{title for quiz 1}
      \quiz[url2]{title for quiz 2}
      . . .
    \end{quizlist}
\end{latexcode}
  If no \textit{url} is given as an optional argument to |\quiz| then
  \MathQuiz{} sets the url(s) to \verb|quiz1.html|, \verb|quiz2.html|
  and so on.
\end{discussion}

\begin{discussion}[Credits]
    \MathQuiz{} was written and developed in the
    \href{http://www.maths.usyd.edu.au/}{School of Mathematics and
    Statistics} at the \href{http://www.usyd.edu.au/}{University of
    Sydney}.  The system is built on \LaTeX{} with the conversion from
    \LaTeX{} to HTML being done by Eitan Gurari's
    \href{http://www.cis.ohio-state.edu/~gurari/TeX4ht/mn.html}{TeX4ht},
    and Michal Hoftich's
    \href{https://github.com/michal-h21/make4ht}{make4ht}.

    To write quizzes using \MathQuiz it is only necessary to know
    \LaTeX, however, the \MathQuiz system has three components:
    \begin{itemize}
      \item A \LaTeX{} document class file, \texttt{mathquiz.cls}, and
      a \TeX 4ht configuration file, \texttt{mathquiz.cfg}, that enable the
      quiz files to be processed by \LaTeX{} and \TeX 4ht, respectively.
      \item A python program, \texttt{mathquiz}, that translates the xml
      file that is produced by \TeX 4ht into  HTML.
      \item Some javascript and css that controls the quiz web page.
    \end{itemize}

   The \LaTeX{} component of \MathQuiz{} was written by Andrew Mathas
   and the python, css and javascript code was written by Andrew Mathas
   (and Don Taylor), based on an initial protype of Don Taylor's from
   2001.  Since 2004 the program has been maintained and developed by
   Andrew Mathas. Although the program has changed substantially since
   2004 some of Don's code and his idea of using \TeX 4ht are still very
   much in use.

   Thanks are due to Bob Howlett for general help with CSS and, for
   Version~5, to  Michal Hoftich for technical advice.
\end{discussion}

\begin{question}
  \label{question1}
  The shaded region in the graph
% \begin{center}
%    \begin{pspicture}(-3,-1.5)(3,4)
%      \pscircle[linewidth=1pt,linestyle=dashed,fillcolor=SkyBlue,fillstyle=solid](1,1){2}
%      \psaxes[linecolor=red,linewidth=1pt,labels=none]%
%      {->}(0,0)(-1.5,-1.5)(3.5,3.5)
%      \rput(3.75,0){$x$}
%      \rput(0,3.85){$iy$}
%      \rput(3,-0.4){3}
%      \rput(-0.4,3){3$i$}
%      \psdots(1,1)
%    \end{pspicture}
% \end{center}
  is equal to which of the following sets of complex numbers?
  \begin{choice}
    \incorrect $\{z \in \C : (z-1)^{2}+(z-(i+1))^{2}<2\}$
    \response  The equation of a circle in the complex plane with
    centre $a+ib$ and radius $r$ is
    \begin{displaymath}
      \{z\in\C : |z-(a+ib)|<r \}.
    \end{displaymath}

    \incorrect $\{z \in \C : z+(i+1)<2\}$
    \response  You want the set of points whose \textit{distance}
    from $1+i$ is less than $2$.

    \correct   $\{z \in \C : |z-(i+1)|<2\}$
    \response  The graph shows the set of complex numbers whose
    distance from $1+i$ is less than $2$.

    \incorrect $\{z \in \C : |z-2|<|i+1-2|\}$
    \response  As $|i+1-2|=\sqrt 2$, this is the set of complex
    numbers whose distance from $2$ is less than
    $\sqrt 2$.

    \incorrect None of the above.
    \response The graph shows the set of complex numbers whose
    distance from the centre of the circle is less than $2$.
  \end{choice}
\end{question}

\begin{question}
  \label{question2}
  Which of the following numbers are prime?
  \begin{choice}[multiple]
    \incorrect 1
    \response  By definition, a prime is a number greater than 1
    whose only factors are 1 and itself.

    \correct   19
    \response  The only factors of 19 are 1 and itself.

    \incorrect 6
    \response  2 is a factor of 6

    \correct   23
    \response  The only factors of 23 are 1 and itself.

    \correct   191
    \response  The only factors of 191 are 1 and itself.
  \end{choice}
\end{question}

\begin{question}
  \label{question3}
  What are suitable parametric equations for this plane curve?
%  \begin{center}
%    \psset{unit=.6cm}
%    \begin{pspicture}(-2.5,-0.5)(5,5.5)
%      \psaxes[linecolor=red,linewidth=1pt,labels=none]%
%      {->}(0,0)(-2.5,-1.5)(5,5)
%      \psellipse[linecolor=blue,linewidth=2pt](1,2)(3,2)
%    \end{pspicture}
%  \end{center}

  \begin{choice}[1]
    \incorrect $x=2\cos t + 1$, $y=3\sin t + 2$
    \response This is an ellipse with centre $(1,2)$ and with axes of
    length $4$ in the $x$-direction and $6$ in the $y$-direction.
%    \begin{center}
%      \psset{unit=.6cm}
%      \begin{pspicture}(-2.5,-0.5)(5,5.5)
%        \psaxes[linecolor=red,linewidth=1pt,labels=none]%
%        {->}(0,0)(-2.5,-1.5)(5,5)
%        \parametricplot[linecolor=blue,linewidth=2pt]{0}{360}%
%        {t cos 2 mul 1 add t sin 3 mul 2 add}
%      \end{pspicture}
%    \end{center}

    \correct $x=3\cos t + 1$, $y=2\sin t + 2$
    \response The curve is an ellipse centre (1,2) with axes length 6
    in the $x$ direction and 4 in the $y$ direction.

    \incorrect $x=3\cos t - 1$, $y=2\sin t - 2$
    \response This is an ellipse with centre $(-1,-2)$ and with axes
    of length $6$ in the $x$-direction and $4$ in the $y$-direction.
%    \begin{center}
%      \psset{unit=.6cm}
%      \begin{pspicture}(-5,-4)(1,2)
%        \psaxes[linecolor=red,linewidth=1pt,labels=none]%
%        {<-}(0,0)(-4.5,-5.5)(1,2)
%        \parametricplot[linecolor=blue,linewidth=2pt]{0}{360}%
%        {t cos 3 mul 1 sub t sin 2 mul 2 sub}
%      \end{pspicture}
%    \end{center}

    \incorrect $x=2\cos t - 1$, $y=3\sin t - 2$
    \response This is an ellipse with centre $(-1,-2)$ and with axes
    of length $4$ in the $x$-direction and $6$ in the $y$-direction.
%    \begin{center}
%      \psset{unit=.6cm}
%      \begin{pspicture}(-4,-5)(1,2)
%        \psaxes[linecolor=red,linewidth=1pt,labels=none]%
%        {<-}(0,0)(-4.5,-5.5)(1,2)
%        \parametricplot[linecolor=blue,linewidth=2pt]{0}{360}%
%        { t cos 2 mul 1 sub t sin 3 mul 2 sub}
%      \end{pspicture}
%    \end{center}
  \end{choice}
\end{question}

\begin{question}
  \label{question4}
  If the vectors $\vect{a}$ (of magnitude 8 units) and $\vect{b}$
  (of magnitude 3 units) are perpendicular, what is the value
  of
  \(|\vect{a} -2\vect{b}|\)~?
  (Hint: Draw a diagram!)

  \answer[units]{10}
  \whenRight The vectors $\vect{a}$, \(-2\vect{b}\), and
  $\vect{a} - 2\vect{b}$ form the sides of a right-angled
  triangle, with sides of length $8$ and $6$ and
  hypotenuse of length $|\vect{a} -2\vect{b}|$. Therefore
  by Pythagoras' Theorem,
  \(|\vect{a} -2\vect{b}|=\sqrt{8^2+6^2}=10\).

  \whenWrong Draw a diagram and then use Pythagoras' theorem.
\end{question}
\end{document}
\endinput
%%
%% End of file `mathquiz-manual.tex'.
