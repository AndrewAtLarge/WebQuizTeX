%-----------------------------------------------------------------------------
%  Copyright (C) 2004-2017 Andrew Mathas, University of Sydney
%
%  Distributed under the terms of the GNU General Public License (GPL)
%                  http://www.gnu.org/licenses/
%
% This file is part of the MathQuiz system.
%
% <Andrew.Mathas@sydney.edu.au>
%-----------------------------------------------------------------------------

\synctex=1

\PassOptionsToClass{tikz,svgnames}{xcolor}
\documentclass[svgnames]{article}
\usepackage[a4paper,margin=30mm]{geometry}
\parindent=4mm
\parskip=1mm
\hfuzz 5pt

\usepackage{pdfpages}
\usepackage{etoolbox}% to patch l@section to remove extraneous spacing
\def\sectionautorefname{Chapter}
\def\subsectionautorefname{Section}

\usepackage[nonewpage]{imakeidx}
\indexsetup{level=\section*,toclevel=section,noclearpage}
\makeindex[intoc,noautomatic,columns=3]
\indexprologue{\textit{This is an index only for the main \MathQuiz manual. It does not index
the on-line manual.}}

\usepackage{mathquiz-doc}

\setcounter{secnumdepth}{2}
\setcounter{tocdepth}{3}
\newcommand\lowerCaseIndex[1]{%
  \lowercase{\def\temp{#1}}%
  \expandafter\index\expandafter{\temp@#1}%
}
% usage: \CrossIndex[*]{main entry}[*]{subentry} with *'s for macros
\NewDocumentCommand\CrossIndex{ smsm }{%
  \IfBooleanTF{#1}{%
    \lowercase{\def\tempa{#2}}%
    \xdef\tempa{\tempa@\noexpand\textbackslash#2}%
  }{\def\tempa{#2}}%
  \IfBooleanTF{#3}{%
    \lowercase{\def\tempb{#4}}%
    \xdef\tempb{\tempb@\noexpand\textbackslash#4}%
  }{\def\tempb{#4}}%
  \expandafter\index\expandafter{\tempa!\tempb}%
  \expandafter\index\expandafter{\tempb}%
}
\newcommand\macroIndex[1]{%
  \lowercase{\def\temp{#1}}%
  \expandafter\index\expandafter{\temp@\textbackslash#1}%
}
\newcommand\gobbleone[1]{}% https://tex.stackexchange.com/questions/318472
\newcommand{\See}[2]{\unskip\emph{see } #1}
\newcommand\SeeIndex[2]{\index{#1!zzzz@\protect\gobbleone|See{#2}}}

\newif\ifCtan\Ctanfalse % condition compilation for ctan distribution

\renewcommand*\contentsname{\relax}

% hyperref links to ctan
\NewDocumentCommand\ctan{ O{pkg/#2} m}{\href{https://www.ctan.org/#1}{\texttt{#2}}}

\newcommand\ddash{\texttt{\textemdash\textemdash}}
\newcommand\mathquizopt[1]{\textsf{mathquiz \ddash#1}}
\newcommand\mathquizrc{\index{mathquizrc}\textsf{mathquizrc}\xspace}
\newcommand\advancedOption{\textit{This is an advanced option. An incorrect value for this setting may stop \MathQuiz from working.}}
%%%%%%%%%%%%%%%%%%%%%%%%%%%%%%%%%%%%%%%%%%%%%%%%%%%%%%%%%%%%%%%%%%%%%%%%%%%%%%%%%%%%%%%
%% MathQuiz title box for front page
\usepackage{tikz}
\usetikzlibrary{shadows.blur}

\definecolor{stone}{HTML}{E9E0D8}
\tikzset{shadowed/.style={blur shadow={shadow blur steps=5},
                          top color=stone,
                          bottom color=PapayaWhip,
                          draw=SaddleBrown,
                          shade,
                          font=\normalfont\Huge\bfseries\scshape,
                          rounded corners=8pt,
      },
      boxes/.style={draw=Sienna,
                    fill=Cornsilk,
                    font=\sffamily\small,
                    inner sep=5pt,
                    rectangle,
                    rounded corners=8pt,
                    text=Brown,
     }
}

\def\MathQuizTitle{
  \begin{tikzpicture}[remember picture,overlay]
      \node[yshift=-3cm] at (current page.north west)
        {\begin{tikzpicture}[remember picture, overlay]
          \draw[shadowed](30mm,0) rectangle node[Brown]{MathQuiz} (\paperwidth-30mm,16mm);
          \node[Sienna,font=\normalfont\small\itshape] at (\paperwidth/2,2mm)
          {\small \mathquiz{description}};
          \node[anchor=west,boxes] at (4cm,0cm) {\mathquiz{name}};
          \node[anchor=east,boxes] at (\paperwidth-4cm,0) {Version \mathquiz{version}};
         \end{tikzpicture}
        };
   \end{tikzpicture}
}

\hypersetup{pdftitle={MathQuiz manual}}

%%%%%%%%%%%%%%%%%%%%%%%%%%%%%%%%%%%%%%%%%%%%%%%%%%%%%%%%%%%%%%%%%%%%%%%%%%%%%%%%%%%%%%%
%% headers and footers
\makeatletter
\def\ps@mathquiz{
  \ps@empty
  \def\@oddfoot{\tiny\MathQuiz -- \mathquiz{version}\hfill%
     \textsc{\ifodd\thepage The MathQuiz manual\else \mathquizheader\fi}\hfill\thepage}
}
% hijack section and subsectionmark for our headers
\def\mathquizheader{\MathQuiz}
\def\sectionmark#1{\def\mathquizheader{#1}}
\def\subsectionmark#1{\def\mathquizheader{#1}}
\pagestyle{mathquiz}
\patchcmd\l@section{1.0em}{0.5em}{}{}
\makeatother

\begin{document}

    \MathQuizTitle

    \begin{quote}
      \MathQuiz makes it possible to write on-line quizzes using
      \LaTeX, providing an easy way for anyone who knows \LaTeX\ to
      create an on-line quiz using \LaTeX\ file.  The
      allows the quiz author to concentrate on the content of quizzes,
      unencumbered by the technicalities of HTML and javascript.

      \ScreenShot{Example \MathQuiz web page}{examples/quiz-page}

      \begin{center}
        \vskip-10mm
        \begin{minipage}{0.7\textwidth}
          \hspace*{3em}\tableofcontents
        \end{minipage}
      \end{center}
    \end{quote}

    \newpage

\section{Introduction}
    On-line quizzes provide a good way to reinforce learning, especially
    because they can give ``interactive'' feedback to the students based
    on the answers that they give. Unfortunately, in addition to writing
    the quiz content there are significant technical hurdles that need
    to be overcome when writing an on-line quiz -- and there are
    additional complications if the quiz involves mathematics or
    diagrams.

    \MathQuiz makes it possible to write on-line quizzes using \LaTeX,
    which is the typesetting language used by mathematicians and
    educators use \LaTeX\ to write their research papers, books and
    teaching materials.  In principle, the quiz can contain anything
    that can be typeset using \LaTeX.  In practise, the \LaTeX\ is
    converted to HTML using
    \href{https://www.tug.org/applications/tex4ht/mn.html}{\TeX 4ht}
    (and \href{https://github.com/michal-h21/make4ht}{make4ht}), so the
    quizzes can contain any \LaTeX commands that are understood by \TeX 4ht,
    which is almost everything. In particular, it is possible to use
    graphics constructed using packages like \ctan{pstricks} and
    \ctan{tikz}; however, see \autoref{SS:classOptions}.

    \MathQuiz supports the following three types of questions:
    \begin{itemize}
      \item Multiple choice questions with a unique correct answer
      \item Multiple choice questions zero or more correct answers
      \item Questions with a numerical answer.
    \end{itemize}
    Each time a student answers a question they are told whether their
    answer is correct and, moreover, it is possible for the quiz author
    to give targeted feedback to the student based on their answer. This
    feed back can be used to explain why the answer is right, why it is
    wrong or to give further hints to the student to help them answer the
    question.

    This introduction outlines how to use \MathQuiz, however, the
    impatient reader may want to skip ahead directly to the
    \autoref{S:documentclass}, where the \LaTeX\ commands used by
    \MathQuiz are described.

    The easiest way to explain how \MathQuiz works is by example. The
    following \LaTeX\ file defines a quiz with a single multiple choice
    question that has four possible answers, each of which has a
    customised response.  (Responses to answers are optional, but they
    are one of the main pedagogical advantages of on-line quizzes
    because the quiz can explain to the student why their answer was
    correct or incorrect.)

    \lstinputlisting[style=latexcode]{examples/simple}

    Since this is a \LaTeX\ file it can be processed using
    \texttt{pdflatex}, or \texttt{latex}, to produce a readable and
    printable version of the quiz, which can be useful when
    proofreading. In this case, the \LaTeX\ version of the quiz looks
    something like this:

    \ScreenShot[0.5]{Sample PDF output}{examples/simple-pdf}

    Of course, the real reason for using \MathQuiz is that it is also
    possible to make an on-line version of the quiz by processing the
    quiz using the \texttt{mathquiz} command. If you do this, and then open
    the resulting web page in your favourite browser, you will see a web page
    that looks something like this:

    \ScreenShot{Sample web page}{examples/simple}

    The on-line version of the quiz displays one question at a time,
    with the question buttons serving the dual purpose of, first,
    providing a way to navigate between the different questions in the
    quiz and, secondly, displaying whether or not the question has been
    attempted and, if so, whether it answered correctly or incorrectly.
    Targeted feedback can be given to the person taking the quiz based
    on their responses.

    As explained in \autoref{S:documentclass}, it is possible to
    customise some aspects of the web pages constructed by \MathQuiz.
    With some rudimentary knowledge of python, it is possible to change the page
    layout of the quizzes or to embed them into your local web pages; see
    \autoref{SS:customisation}.

\subsection{What MathQuiz does and does not do}

    The \MathQuiz program was designed to be run from the command-line,
    so to process the file \textsf{quiz.tex} using \MathQuiz you would
    type

    $>$ \Verb|mathquiz quiz| \qquad or \qquad $>$ \Verb|mathquiz quiz.tex|

    \noindent from the command-line. Although I have not tested this, it
    should also be possible to run \MathQuiz from inside programs like
    \TeX Shop by setting the compiler equal to \textsf{mathquiz}.

    \MathQuiz can be used to ask the student a series of questions. In
    the on-line version of the quiz, one question is displayed at a
    time. Each question in a quiz, and each quiz itself, can be
    attempted as many times as the student wants. \MathQuiz does not
    limit the number of times that questions can be attempted.
    The following three types of questions are supported:
    \begin{itemize}
      \item Multiple choice questions with a unique correct answer
      \item Multiple choice questions zero or more correct answers
      \item Questions with a numerical answer.
    \end{itemize}
    Questions with symbolic answers are not supported.
    For example, if the answer to a question is $0$ then $\sin(0)$ will
    not be accented as a correct answer and the only way to ask for the
    indefinite integral of a function is as a multiple choice question.

    The quizzes made using \MathQuiz are intended to be used as a
    revision resource rather than as an assessment tool. Consequently,
    \MathQuiz does not provide a mechanism for recording the marks
    obtained by the students taking the quiz. Technically, it probably
    would not be very hard to record marks but this introduces a
    significant amount of extra overhead in terms of student
    authentication and interfacing with a database. In addition, if
    \MathQuiz were used as an assessment tool then there would be
    additional ``security issues'' to ensure that the quiz content is
    secure. Currently, even though the solutions to the quiz questions
    do not appear in the HTML source code for the quiz it is possible to
    access the answers if you know when you are doing.

    The questions in a \MathQuiz quiz are static. In particular, they do
    not accept variables and exactly the same questions will appear in
    exactly the same order each time the quiz is taken. It would not be
    hard to make the questions appear in a random order. Randomising the
    order of the multiple choice answers would be more be difficult with
    the current implementation.

    The quizzes are not timed and do not have time-limits.

    Finally, the point of \MathQuiz is to make it possible to write
    on-line quizzes without knowing any HTML, so \MathQuiz provides
    almost no mechanisms for controlling the style and format of the
    quizzes that it creates (although see \autoref{SS:customisation}).
    In particular, the colour scheme of the quizzes is
    fixed.\footnote{In principle, adding support for different colour
    schemes would be easy to do, the only real difficulty is in finding
    compatible colours. There is another minor complication in that the
    \MathQuiz CSS file, \textsf{mathquiz.css}, is written using
    \href{http://sass-lang.com/}{sass}. This said, I would be happy to
    incorporate different colour schemes that are sent to me.}
    It is possible to add some styling using, for example, the
    \Verb|\Css| command from \ctan{tex4ht}, however, if you really want
    to do this then you are probably better off writing
    \href{https://www.w3schools.com/css/}{CSS} directly.

    Some of the ``missing features'' listed above may be added to \MathQuiz in a future release.

\subsection{Credits}
    \MathQuiz{} was written and developed in the
    \href{http://www.maths.usyd.edu.au/}{School of Mathematics and
    Statistics} at the \href{http://www.usyd.edu.au/}{University of
    Sydney}.  The system is built on \LaTeX{} with the conversion from
    \LaTeX{} to HTML being done by Eitan Gurari's
    \href{http://www.cis.ohio-state.edu/~gurari/TeX4ht/mn.html}{TeX4ht}
    and
    \href{https://github.com/michal-h21/make4ht}{make4ht}.

    To write quizzes using \MathQuiz it is only necessary to know
    \LaTeX, however, the underlying \MathQuiz system actually has three components:
    \begin{itemize}
      \item A \href{https://www.latex-project.org/}{\LaTeX} document class file, \texttt{mathquiz.cls}, and
      a \ctan[tex4ht]{\TeX 4ht} configuration file, \texttt{mathquiz.cfg}, that enable the
      quiz files to be processed by \LaTeX{} and \TeX 4ht, respectively.
      \item A \href{https://www.python.org/}{python} program, \texttt{mathquiz}, that translates the
      \LaTeX{} into xml, using \TeX 4ht, and then into HTML.
      \item \href{https://www.w3schools.com/Js/}{Javascript} and \href{https://www.w3schools.com/css/}{css}
      code that works behind the screens to control and style the quiz web pages.
    \end{itemize}

   The \LaTeX{} component of \MathQuiz{} was written by Andrew Mathas
   and the python, css and javascript code was written by Andrew Mathas
   (and Don Taylor), based on an initial prototype of Don Taylor's from
   2001.  Since 2004 the program has been maintained and developed by
   Andrew Mathas. Although the program has changed substantially since
   2004 some of Don's code, and in particular his idea of using
   \TeX4ht, is still in use.

   Thanks are due to Bob Howlett for general help with CSS and to
   Michal Hoftich for invaluable technical advice on \TeX4ht.

 \section{System requirements, installation and configuration}\label{C:configuration}
  \index{system requirements}
  \CrossIndex{system requirements}{tex4ht}
  \CrossIndex{system requirements}{make4ht}
  \CrossIndex{system requirements}{python}

  \subsubsection[System requirements: Python3, \LaTeX{} and \TeX4ht]%
          {System requirements: Python3 and \LaTeX, including \TeX 4ht
    and \textsc{make4ht}}

    It is advisable to have an up-to-date \LaTeX{} distribution, such
    as that provided by \href{https://www.tug.org/texlive/}{\TeX live},
    as well as a recent version of \href{https://www.python.org/}{Python 3}
    (as of writing, python 3.6 is available). I have tested the
    \MathQuiz system extensively on Linux and Mac operating systems.
    Several people have used \MathQuiz on windows computers, but I
    have not tested the program on a window myself.

    \subsubsection{\MathQuiz components --- \LaTeX, python, web, doc}

    The \MathQuiz program has different four components:
      \begin{itemize}
           \item \LaTeX\ files (a class file and \TeX4ht configuration files)
           \item Python3 executables that use \TeX4ht to convert \LaTeX\ files into web pages
           \item Web files (javascript, css and on-line documentation)
           \item Documentation
      \end{itemize}
    Of course, to use the on-line quizzes created by \MathQuiz you will
    also need a web server. To use \MathQuiz all of these files need to
    be in appropriate places.
    \ifCtan Fortunately, \ctan[/]{ctan} takes care of most of this but
    \else
    You will need to install the \LaTeX{} files somewhere in the
    \LaTeX{} search path and
    \fi
    the web-related files still need to be put onto your web server.

    \subsubsection{Installation}\index{installation}\index{initialisation}

    The easiest way to install the web-related files is to run \MathQuiz
    from the command line using:
    \begin{quote}
      \mathquizopt{initialise}
    \end{quote}
    In fact, it is enough to run \textsf{mathquiz} from the command-line
    without any options because it will keep asking you to initialise
    until you do so. \MathQuiz will work without being initialised,
    however, any web pages that are created before initialisation will
    be emblazoned with a message reminding you to initialise \MathQuiz.
    As described in more detail below, initialisation copies the
    javascript and CSS files used by \MathQuiz into the filespace used
    by your web server.
    \ifCtan\else If you are installing \MathQuiz from the zip file then
    you need to read the next section, otherwise
    \fi

    I recommend that the TLDR-crowd just run the initialisation command
    and follow the prompts. More detail about what is required appears
    below for those who want it.

    \ifCtan\else

    \subsubsection{Installation of \MathQuiz from the zip file}

    (\textit{This section of the manual will be removed when the
    package is put onto \ctan[/]{ctan}.})

      The \MathQuiz zip file has three directories, or folders:

      \begin{itemize}
        \item[--] mathquiz/latex
        \item[--] mathquiz/doc
        \item[--] mathquiz/scripts
      \end{itemize}

      The files in the \textsf{latex} directory need to be put somewhere in the \LaTeX\
    search path. For example, on my computer I have these files in
    \begin{center}
       \Verb|/usr/local/texlive/texmf-local/tex/latex/local/mathquiz|
    \end{center}
    After you have moved these files to an appropriate place you will need to run
    something like texhash. The exact command that you need to run to
    tell \LaTeX\ that you have installed some latex files depends on the
    \TeX distribution that you are using.

    The files in the \textsf{doc} directory are the documentation. You can
    put these files where ever you like, although if you want programs
    like \textsf{texdoc} to find them then you will need to put them
    into an appropriate place in the \LaTeX{} directory tree.

    The \MathQuiz program itself lives in the script directory. On a UNIX
    like system I recommend making a link to the file mathquiz.py, which is
    the entry point to the code, using something like:
    \begin{center}
        \Verb|ln -s <path to scripts directory>/mathquiz.py mathquiz|
    \end{center}
    Of course, the \textsf{mathquiz} executable should be in
    the system path. Under windows, \textit{I believe} that you need to create a
    \textit{batch file}, \textsf{mathquiz.bat}, that contains something like this:
    \begin{center}
        \Verb|c:\python3\python.exe c:\MathQuiz\mathquiz.py %*|
    \end{center}
    Of course, you will need to modify the paths in the batch.

    It remains to install the web files used by \MathQuiz, which can be
    done using \MathQuiz.
    \fi

    \subsubsection{Installing the web files used by \MathQuiz}
    \CrossIndex{command-line options}{initialise}

     The quiz files created by \MathQuiz use
     \href{https://en.wikipedia.org/wiki/JavaScript}{javascript} and
     \href{https://www.w3schools.com/css/css_intro.asp}{cascading style
     sheets} (CSS) to render the quizzes. You do not need to understand
     how this works but you do need to put the \MathQuiz javascript and
     CSS files onto your web server.\footnote{In fact, \MathQuiz will work
     even if you do not install these files on your web server, however,
     the quiz pages that it creates will have an annoying message at the
     top of the web page to remind you to install these files.} It is
     necessary that these files are accessible to your web server, but
     it does not matter whether they are in a ``system directory'' or in
     your personal web directories.

     \MathQuiz provides a semi-automated way of moving these files into
     place, so the short version of the installation process is to fun
     the following command from the command-line:
     \begin{center}
        $>$ \mathquizopt{initialise}
     \end{center}
     You will be asked a series of questions that should be
     straightforward to answer because \MathQuiz talks you through what
     needs to be done. More details follows about the initialisation
     process in the hope that this might be useful.

     When you run the \MathQuiz initialisation command you will be prompted for the
     following:
     \begin{itemize}
       \item The \MathQuiz web directory, which is a directory on your local file system that is visible
             to your web server
       \item The relative URL for this directory, which tells your web browser where to find these files
     \end{itemize}
     For example, on my system the base directory for our web server is
     \textsf{/usr/local/httpd/} and the \MathQuiz web
     files are in \textsf{/usr/local/httpd/MathQuiz}. So, I set:
     \begin{quote}
       \begin{tabular}{lll}
         \MathQuiz web directory &=& \textsf{/usr/local/httpd/MathQuiz}\\
         \MathQuiz relative URL  &=& \textsf{/MathQuiz}
       \end{tabular}
     \end{quote}
     In addition, the initialisation command lets you set global defaults for the following:
     \begin{itemize}
       \item breadcrumbs = the breadcrumbs at the top of the quiz web page
       \item department = a (short) name for your department
       \item department\_url = the URL for your department's web server
       \item institution = a (short) name for your institution
       \item institution = the URL for your institution
     \end{itemize}
     These settings are mainly used in the \textit{breadcrumbs} on the
     quiz web pages; see \autoref{S:documentclass} for
     details.\index{breadcrumbs}
     If in doubt just hit return to accept the default values for these
     variables, which for most of these amounts to leaving them blank.
     You can come back and change these settings at any time using the command:
     \begin{center}
        \mathquizopt{edit-settings}.
     \end{center}
     In addition, these defaults can always be overwritten by commands
     in the individual quiz files.

     Finally, the initialisation command will ask you about some more
     advanced settings. It is very unlikely that you will need to change
     these when you first use \MathQuiz, so I recommend that you hit
     return to accept the default options for each of these settings.

    \bigskip

    \MathQuiz is now ready to use!

  \section{The MathQuiz document class --- the \LaTeX{} commands}\label{S:documentclass}

  This section describes the commands and environments provided by the
  \MathQuiz document class. The aim of \MathQuiz is to construct web
  pages for quizzes the \LaTeX\ commands provided all aim to put
  material onto a web page. All of the code examples given in this, and
  other sections, can be found in the \textsf{examples} directory of the
  \MathQuiz web directory. More details and some examples can be found
  in the on-line manual in \autoref{S:online}.

  A typical \MathQuiz quiz file is just a \LaTeX file of the form:

    \lstinputlisting[style=latexcode]{examples/typicalquizfile}

  \noindent You should write your quizzes in whatever editor you
  normally use to write \LaTeX{} and you should use \Verb|latex| or
  \Verb|pdflatex| to check that your quiz compiles and that you are
  happy with the output as you write it Once you are happy with the
  content of the quiz you can then convert it to an on-line quiz using
  \MathQuiz. There are several reasons for using this workflow:
  \begin{itemize}
    \item
    \textit{Very file that you give to \MathQuiz should be a valid \LaTeX{} file!}

    \item The \textsf{dvi} or \textsf{PDF} file produced by \LaTeX{}
    will show all of the information about the questions one the same
    page (that is, the question, the answers and the feedback that you
    will give to the students based on their answers). In contrast, by
    design, the web pages hide most of this information.

    \item Typesetting the quiz file with \LaTeX{} is \textit{much
    faster} than processing it with \MathQuiz. In fact, \MathQuiz uses
    \textsf{htlatex} to process the quiz file \textit{three times} in
    order to produce an \textsf{xml} file and it is only then the
    \MathQuiz program kicks in to rewrite this data as an \textsf{HTML}
    file.

    \item If \LaTeX{} produces errors then \MathQuiz will produce more
    errors. Further, \textit{\LaTeX{} error messages are much easier to read and
    debug than those produced by \TeX4ht and \MathQuiz}.
  \end{itemize}
  Note that the PDF version of a quiz does not contain any
  information about the unit, department or institution. This
  information was included in previous releases of \MathQuiz, however, it
  is now omitted because this data can now be specified in the \mathquizrc
  file, in which case \LaTeX{} has incomplete information.

  The next sections describe the different environments and commands
  provided by \MathQuiz. The last section in this chapter,
  \autoref{SS:classOptions}, lists the class options for the \MathQuiz
  document class. You should read this section if you are using
  \ctan{pstricks} or \ctan[pgf]{tikz}.

  Finally, if you want to use \MathQuiz for your students then you will
  need to \textit{initialise} \MathQuiz using the command:
  \mathquizopt{initialise}; see \autoref{C:configuration} for more
  details.

\subsection{\MathQuiz environments}

The \MathQuiz document class defines the following four environments:
\begin{quote}
  \begin{description}
    \item[question] Each quiz question needs to be inside a
    \Verb|question| environment
    \item[choice] For typesetting multiple choice questions
    \item[discussion] For including (optional) discussion, or revision, material at
    the top of the quiz
    \item[quizlist] For writing an index file for a related ``family'' of quizzes,
    such as the quizzes for a unit of study
  \end{description}
\end{quote}
This section describes these environments and gives examples
of their use.

\subsubsection{Question environments}
\CrossIndex{environment}{question environment}

Quizzes are composed of questions and each quiz question must to be
placed inside a \Verb|question| environment. Typically, a quiz will have
several questions, each wrapped in its own \Verb|question| environment.
For brevity, all of the examples in this chapter have only one question
-- see the on-inline manual in the \MathQuiz web directory for a complete quiz file.

  \lstinputlisting[style=latexcode]{examples/question}

  This example code shows how to use \Verb|\answer|
  \macroIndex{answer}\CrossIndex{question environment}*{answer}
  \CrossIndex*{answer}*{whenRight}
  \CrossIndex*{answer}*{whenWrong}
  \index{feedback!whenRight}
  \index{feedback!whenWrong}
  \SeeIndex{numeric answer}{\textbackslash answer}

\subsubsection{Choice environments}
\CrossIndex{environment}{choice environment}
\CrossIndex{question environment}{choice environment}
\SeeIndex{multiple choice}{choice environment}

The multiple choice options for a quiz question need to be placed inside
a \Verb|choice| environment. The \Verb|choice| environment accepts two
optional arguments, which can appear in any order:
\begin{itemize}
  \item\CrossIndex{choice environment}{single}\CrossIndex{choice environment}{multiple}
  The word \Verb|single| (default, and can be omitted) or
  \Verb|multiple|, which indicates whether the quiz has a
  \textit{single} correct answer or whether 0 or more of the answers are
  correct, respectively.
  \item \index{choice environment!columns}
  A non-negative integer $n$, specifying that the choices should
  be rendered in $n$ columns.
\end{itemize}
Of course, \Verb|choice| environments need to be put inside a
\Verb|question| environment.

A \Verb|choice| environment  is similar to the list-like environments
(\textit{enumerate}, \textit{itemize}, \textit{description}, ...) except
that rather than using \Verb|\item| to separate the different list item
you should use
\Verb|\correct| \CrossIndex{choice environment}*{correct}
for correct responses and
\Verb|\incorrect| \CrossIndex{choice environment}*{incorrect}
for incorrect responses. In addition, you can use
\Verb|\response| \CrossIndex{choice environment}*{response}
to give a feedback response to the person taking the quiz when they
select the last correct or incorrect answer.
\index{feedback!response}

Here is an example of a multiple choice question with a unique answer

  \lstinputlisting[style=latexcode]{examples/choice-single}

\noindent
It is not necessary to put the \Verb|\response| lines on the same line
as the \Verb|\(in)correct| lines; this is done here only to make the
example more compact.  Here is an example of a multiple choice question
with two correct answers:

  \lstinputlisting[style=latexcode]{examples/choice-multiple}

When option \Verb|multiple| is used the question is marked correct if
and only if all of the correct responses are selected.

  \subsubsection{Discussion environments}
  \CrossIndex{environment}{discussion environment}

In addition to asking questions it is possible to have revision, or
discussion, material at the \textit{start} of the quiz.  The discussion
items always appear before the quiz questions in the menu down the
left-hand side. It is not possible to interleave discussion items and
questions. Each quiz can have zero or more discussion environments and
these environments can, in principle, contain arbitrary \LaTeX\ code.

For example, running the \LaTeX\ file through \MathQuiz

  \lstinputlisting[style=latexcode]{examples/discussion}

produces the web page:

\ScreenShot{Web page: discussion environment}{examples/discussion}

As with the questions, only one \Verb|discussion| environment is
displayed on the quiz web page at a time. It is possible to have quizzes
that contain only \Verb|discussion| environments, with no questions, and
quizzes that contain only \Verb|question| environments, with no
discussion.

  \subsubsection{Quiz indexes}

  \index{quizlist environment}
  \CrossIndex{environment}{quizlist environment}

  Most quizzes occur in sets that cover related material, for example
  for a particular unit of study. The \Verb|quizlist| environment can be
  used to create an index page for the quizzes.

  \lstinputlisting[style=latexcode]{examples/index}

  This code will produce an index web page of the form:

  \ScreenShot{Example index page}{examples/index}

  As the next section describes, index files are also used to
  automatically adding a drop-down menu to the breadcrumbs for a web
  page, which access the other quizzes for the unit. As the example
  above shows, the entries in the \Verb|quizlist| ae all given using the
  \Verb|\quiz| command.\CrossIndex{quizlist environment}*{quiz} The
  syntax for this command is

  \begin{center}
      \Verb|\quiz|[URL for quiz]\{Title for quiz\}
  \end{center}

  \noindent
  The \Verb|\quiz| macro automatically inserts the quiz numbers
  into the list and, by default, it assumes that the URL for the quiz
  is of the form \textsf{quiz1.html}, \textsf{quiz2.html},
  \textsf{quiz3.html}, .... If the URL for the quiz is something
  different this can be given as a optional argument to \Verb|\quiz|.

  \subsubsection{Breadcrumbs}\label{SS:breadcrumbs}\index{breadcrumbs}

  \MathQuiz provides a straightforward way to place navigation breadcrumbs
  at the top of the quiz web page. By default these
  breadcrumbs look something like this:

  \ScreenShot{Example of breadcrumbs}{examples/examples-breadcrumbs}

  \noindent
  The breadcrumbs will typically be links, so clicking on the
  \textit{School of Mathematics and Statistics} will take you to the
  School's web page, clicking on \textit{MATH1001} will take you to the
  unit of study web page and clicking on \textit{Quizzes} will take you
  to the index page for the quizzes for the current unit. As alluded to
  in the last section, clicking on the~{\large\color{red} $\equiv$} symbol after
  \textit{Quizzes} causes a drop-down menu to appear that
  contains links to all of the quizzes for the current unit of study.

  \ScreenShot{Drop-down menu giving index of quizzes}{examples/quizlist-dropdown.png}

  \noindent
  The breadcrumbs for the quiz web page can be either be configured quiz-by-quiz, using the
  \Verb|\BreadCrumbs| macro \macroIndex{BreadCrumbs}, or by setting \textsf{breadcrumbs} ``globally'' in the
  \mathquizrc file using the command-line option
  \begin{center}
        \mathquizopt{edit-settings}\qquad (see \autoref{SS:commandline}).
  \end{center}

  \noindent
  The breadcrumbs are specified as a ``|-separated list''. For example,
  the default breadcrumbs can be specified as

  \begin{lstlisting}[style=latexcode]
    \BreadCrumbs{ department | unitcode | quiz_index | breadcrumb }
  \end{lstlisting}

  \noindent
  or by setting \textsf{breadcrumbs} equal to the string
  \begin{center}
    \texttt{department | unitcode | quiz\_index | breadcrumb}
  \end{center}
  in the \mathquizrc file. More generally, the breadcrumbs are specified
  as:

  \begin{lstlisting}[style=latexcode]
    \BreadCrumbs{ crumb1 | crumb2 | crumb3 | crumb4 | ...  }
  \end{lstlisting}

  \noindent
  In principle, there can be arbitrarily many crumbs in your
  breadcrumbs but, in practice, five is more than enough. Using
  \Verb|\BreadCrumbs{none}| will disable the breadcrumbs. Similarly,
  setting \textsf{breadcrumbs = none} in the \mathquizrc will also disable
  the breadcrumbs, unless there is a \Verb|\BreadCrumbs| command in the
  quiz file, which will overwrite the default settins in the rc-file.

  The \Verb|\BreadCrumbs| command accepts the following ``magic crumbs'':

  \begin{description}
    \item[breadcrumb] The crumb for the current page, which is set using
    \Verb|\BreadCrumb| or \Verb|\title|
    \CrossIndex*{BreadCrumb}*{title}

    \item[department] This expands to a link to your department, where
    the deparment is set using \Verb|\Department| and its URL is set by
    \Verb|\DepartmentURL|
    \macroIndex{Department}\macroIndex{DepartmentURL}

    \item[institution] This expands to a link to your institution, where
    the deparment is set using \Verb|\Institution| and its URL is set by
    \Verb|\InstitutionURL|
    \macroIndex{Department}\macroIndex{DepartmentURL}

    \item[quiz\_index] This expands to ``Quizzes'', which is a link to
    the index page for your unit as given by \Verb|\QuizzesURL|. In
    addition, a drop-down menu to the index page is added using the
    symbol {\large\color{red} $\equiv$}.
    \macroIndex{QuizzesURL}

    \item[unitcode] This expands to a link to the unit code, where
    the unit code is set using \Verb|\UnitCode|, and its URL is set by
    \Verb|\UnitURL|
    \macroIndex{UnitCode}\macroIndex{UnitURL}

    \item[unitname] This expands to a link to the unit name, where
    the unit name is set using \Verb|\UnitName|, and its URL is set by
    \Verb|\UnitURL|
    \macroIndex{UnitName}\macroIndex{UnitURL}

  \end{description}

  \noindent
  In addition, each \textit{crumb} in a breadcrumb is allowed to be
  arbitrary text, where the last ``word'' is treated as a URL if it
  either beings with a backslash, $\backslash$, or with \texttt{http}.
  For example, the follwing is perfectly valid:

  \begin{lstlisting}[style=latexcode]
   \BreadCrumbs{ Monty Python http://www.montypython.com/
                 | Work http://www.montypython.com/ourwork
                 | Meaning of life http://www.montypython.com/film_Monty\%20Python's\%20The\%20Meaning\%20of\%20Life\%20(1983)/17
   }
  \end{lstlisting}

  \noindent
  If you use this in your quiz file then you will find following breadcrumbs at the top of your quiz pages:

  \ScreenShot{Guaranteed to offend some one}{examples/montypython}

  If anything used in a breadcrumb is not specified then the crumb name
  together with some question marks will appear on the web page. For
  example, uyou wuill see \textit{?? unit name} if the unit name has not
  been specified using \Verb|\UnitName|.

  In more detail, here is the list of \MathQuiz commands that can be used to
  specify the contents of the breadcrumbs. Some of these can set
  globally in the \mathquizrc file.

  \newcommand\Item[1]{\item[\textbackslash#1]\CrossIndex{breadcrumbs}*{#1}}
\begin{description}
  \Item{BreadCrumb}
     Sets last item in the web page breadcrumb, which refers to the
     current quiz. By default, the breadcrumb is as set to be the
     part of the quiz title, as set by \verb!\title!,
     before the first colon. For example, the title

     \hspace*{20mm}\verb!\title{Quiz 1: Some interesting questions about frogs}!

     sets the breadcrumb to ``Quiz 1''.

  \Item{Department}
    The name of the department that runs this unit (for example,
    Mathematics). This appears below the question buttons on the quiz web
    page.

    The department can be set globally using \texttt{mathquiz \textemdash\textemdash edit-settings}.

  \Item{DepartmentURL}
    The URL for the department that runs this unit. This appears below
    the question buttons on the quiz web page.

    The department URL can be set globally using \mathquizopt{edit-settings}.

  \Item{Institution}
    The institution, or university, that appears below the question
    buttons on the quiz web page. The institution also appears below
    tyhe question buttons in the left-hand navigation menu.

    The institution can be set globally using \mathquizopt{edit-settings}.

  \Item{InstitutionURL}
    The URL for the Institution, or institution, that appears below the
    question buttons on the quiz web page.

    The institution URL can be set globally using \mathquizopt{edit-settings}.

  \Item{QuizzesURL}
    The URL for the suite of quizzes attached to this unit of study. This
    is used in the breadcrumb at the top of the quiz web page.

  \Item{UnitCode}
    The unit of study code for the unit that the quiz is attached
    to. This is used in the breadcrumb at the top of the quiz web page.

  \Item{UnitName}
    The name of the unit of study for the unit that the quiz is attached
    to. This is used in the breadcrumb at the top of the quiz web page.

  \Item{UnitURL}
    The URL for the unit of study code for the unit that the quiz is attached
    to. This is used in the breadcrumb at the top of the quiz web page.

\end{description}

  \subsection{MathQuiz class options}\label{SS:classOptions}

The two most commonly used packages for drawing pictures or diagrams in
\LaTeX\ are \ctan{pstricks} and \ctan{tikz}. Unfortunately, both of
these have issues when used with \TeX 4ht. The two \MathQuiz options try
to mitigate for some of the known issues.

\begin{description}
  \item[pst2pdf] \index{pst2pdf}\index{pstricks}
    For the most part \ctan{pstricks} drawings will display correctly and
    when they fail they can often be salvaged by using \ctan{pst2pdf}. The
    \Verb|pst2pdf| class option automates what needs to be done to apply
    \ctan{pst2pdf} with \MathQuiz.to convert all postscript objects in the
    quiz to images. Whilst not guaranteed to work, this often fixes issues
    with \ctan{pstricks} diagrams. This class option is equivalent to
    using the \Verb|pst2pdf| command-line option; see
    \autoref{SS:commandline}.

    \lstinputlisting[style=latexcode]{examples/pst2pdf}

    \textbf{Note} According to the \ctan{pst2pdf} manual:

    \begin{quote}
      \textsf{pst2pdf} needs Ghostscript (>=9.14), perl (>=5.18), pdf2svg, pdftoppm
      and pdftops (from poppler-utils or xpdf-utils) for the process file.
    \end{quote}

    Unfortunately, \textsf{pst2pdf} can fail silently without giving any warnings. If
    using \textsf{pst2pdf} does not produce an image then the problem
    might be that you have not installed all of the programs that
    \textsf{pst2pdf} relies upon, so check the list above.

    \textit{Try the pst3pdf class option if you are having trouble
    displaying an image created using \ctan{pstricks}. It is not
    guaranteed to work but it does sometimes fix the problem}.

  \item[tikz]\index{tikz}
    Giving this class option both loads the \ctan[pgf]{tikz} package (so
    you do not need to have \Verb|\usepackage{tikz}| in your \LaTeX\ file)
    and, as a bonus, fixes several issues with PGF that prevent it from
    working with \TeX 4ht. Thanks are due to Michal Hoftich for supplying
    both fixes!\footnote{The first issue is a bug that has been reported
    to the PGF developers, together with a one-line solution, but for
    reasons unknown they have not fixed the problem; see
    \href{https://tex.stackexchange.com/questions/386757}{Work around for
    bug in pgf when used with htlatex}.  The second issue is that the PGF
    files hard code \textsf{ISO-8859-1} encoding, which is a problem if
    you use UTF-8; see
    \href{https://tex.stackexchange.com/questions/390421}{Make4ht, tikz
    and UTF 8 encoding question}.  }

    \lstinputlisting[style=latexcode]{examples/tikz}
\end{description}

All other class options that are given to the \MathQuiz document class
are passed to the \texttt{article} class, which is the base class used
by \texttt{mathquiz}.

\section{The MathQuiz program}
\index{usage}
    The \MathQuiz program was designed to be run from the command-line,
    so to process the file \textsf{quiz.tex} using \MathQuiz type:

    $>$ \Verb|mathquiz quiz| \qquad or \qquad $>$ \Verb|mathquiz quiz.tex|

    \noindent (Here ``$>$'' is the command-line prompt.) It should
    be possible to run \MathQuiz from inside programs like \TeX Shop by
    setting the compiler equal to \textsf{mathquiz}, but this has not
    been tested. Once useful feature of \MathQuiz is that you can ask it
    to process more than one quiz file at a time. For example, if you have
    quiz files \textsf{quiz1.tex}, ..., \textsf{quiz9.tex} in the
    current directory then, on a UNIX system, you can rebuild the web
    pages for all of these quizzes using the single command:

    $>$ \Verb|mathquiz quiz[1-9].tex|

    \noindent
    This is useful if some aspect of all of the quizzes has changed. In
    fact, one would probably use

    $>$ \Verb|mathquiz --qq quiz[1-9].tex|

    \noindent
    because the \mathquizopt{qq} command-line option suppresses almost all of
    the output produced by \LaTeX\ and \TeX4ht. The next section
    discusses the \MathQuiz command-line options.

    \subsection{Command-line options}\label{SS:commandline}
\index{command-line options}

\begin{verbatim}
usage: mathquiz [-h] [-i] [-q] [--settings] [--edit-settings  [SETTING]] [-s]
                [-r RCFILE] [-p] [--build MATHQUIZ_MK4]
                [--mathquiz_format MATHQUIZ_FORMAT]
                [quiz_file [quiz_file ...]]
\end{verbatim}

The command-line options are listed on separate lines here to improve
readability but they can appear in any order when you use them provided
that they are all on the same line.  The order that the command-line
options are listed in indicates how often you are likely to need this
option.

    \begin{description}
       \item[ -h, \ddash help] \CrossIndex{command-line options}{help}
          list the command-line options and exit

       \item[-i, \ddash initialise] \CrossIndex{command-line options}{initialise}
          Initialise files and settings for mathquiz. The command
          \mathquizopt{initialise} should be run before using
          \MathQuiz. This command will help you to copy the web files needed by
          \MathQuiz into the directories used by your web server. See
          \autoref{C:configuration} for more details.

          \item[\ddash settings {[SETTING]}] \CrossIndex{command-line options}{settings}
          \SeeIndex{rc file}{mathquizrc}
          \SeeIndex{default settings}{mathquizrc}
          List system settings for mathquiz stored in the  rc-file
          (\textit{run-time configuration file}); see the \textsf{\ddash
          rcfile} command-line option below. Optionally, a single
          \textsf{SETTING} can be given inb which case the value of only that
          setting is returned. Typical settings returned by this
          command look like:
          \begin{verbatim}
  MathQuiz settings stored in /Users/andrew/.mathquizrc

    # Relative URL for mathquiz web directory
    mathquiz_url    = /MathQuiz

    # Breadcrumbs at the top of quiz page
    breadcrumbs     = department|unitcode|Quuiz_index|breadcrumb

    # Department running quiz
    department      = School of Mathematics and Statistics

    # (Relative) URL for department running quiz
    department_url  = http://www.maths.edu.au

    # University or institution
    institution     = University of Sydney

    # (Relative) URL for university or institution
    institution_url = http://sydney.edu.au/

    # The python module that determines the format of the quiz web page (default)
    mathquiz_format = mathquiz_standard

    # Local URL for mathjax (default)
    mathjax         = https://cdnjs.cloudflare.com/ajax/libs/mathjax/2.7.0/
          \end{verbatim}

          These settings are used by all of the quizzes created by
          \MathQuiz unless the quiz overrides them. These defaults are
          normally set when \MathQuiz is initialised, using the command:
          \mathquizopt{initialise}. The defaults can be changed any time using:
          \mathquizopt{edit-settings}.

       \item[\ddash edit-settings] \CrossIndex{command-line options}{edit-settings}
          Edit the mathquiz settings in the rc-file. The settings that
          are marked as advacned should be changed only with care. To
          change the \textsf{mathquiz\_dir} and \textsf{mathquiz\_url}
          use: \mathquizopt{initialise}.

          \textit{The settings will be automatically saved to an rc-file,
          \textsf{.mathquizrc}, in your home directory if you do
          not have permission to write to the system rc-file in the
          \MathQuiz scripts directory.}

       \item[-q, -qq, \ddash quiet] \CrossIndex{command-line options}{quiet mode}
       Suppress tex4ht messages: \textsf{\ddash q} is quiet and \textsf{\ddash qq} is  very quiet. If you use
       \textsf{mathquiz --qq texfile.tex} then almost no output will
       be printed by \MathQuiz when it is processing your quiz file. Be
       warned, however, that both of these options csn make it harder to find and fix
       errors, so using the \textsf{\ddash q} and \textsf{\ddash qq} options is not
       recommended if your file is not compiling.

       \item[-s,\ddash shell-escape] \CrossIndex{command-line options}{shell-escape}
          Shell escape for htlatex/make4ht

       \item[-r RCFILE, \ddash rcfile RCFILE]\index{mathquizrc}
          Specify the rc-file (\textit{run-time configuration file}), file
          that \MathQuiz should read (and write to).

          By default, \MathQuiz will first read the system rc-file (the
          file \textsf{mathquizrc} in the \MathQuiz script directory),
          followed by the rc-file in your home directory (the file
          \textsf{$\sim$/.mathquizrc}), if it exists. If it is unable to
          write to the system rc-file then \MathQuiz will create an
          rc-file in your home directory, which it will subsequently
          read whenever you run \MathQuiz. This setting is only
          necessary if you want to override the default rc-file.

          The rc-file can be editted by hand, however, it is
          recommended that you instead use
          \begin{center}
            \mathquizopt{edit-settings}
          \end{center}
       \item[-p, \ddash pst2pdf] \CrossIndex{command-line options}{pst2pdf}
          Use the \textsf{pst2pdf} command-line option to, \textit{potentially},
          fix issues with images generated by pstricks.  This option is
          equivalent to using the \Verb|pst2pdf| document class option;
          see \autoref{SS:classOptions}.
          \newline
          \textit{Try this command-line option if you are having trouble
          displaying an image created using \ctan{pstricks}. It is not
          guaranteed to work but it does sometimes fix the problem}

       \item[\ddash build MATHQUIZ\_MK4] \CrossIndex{command-line options}{mk4 build file}
          Build file for make4ht. This can be used to add local
          customisations for make4ht. See the \ctan{make4ht} manual for
          more information. This
          option is equivalent to setting the \textsf{mathquiz\_mk4}
          in the \mathquizrc file. For more detail see
          \autoref{S:rcfile.}

          \advancedOption

       \item[\ddash mathquiz\_format MATHQUIZ\_FORMAT] \CrossIndex{command-line options}{format}
          Local python code for formatting the quiz web page. This
          option is equivalent to setting the \textsf{mathquiz\_format}
          in the \mathquizrc file.
          \advancedOption

    \end{description}

    \subsection{The \mathquizrc
    file}\label{SS:customisation}\label{S:rcfile}

    \MathQuiz stores various \textit{run-time configuration settings} in the \mathquizrc file. The
    system rc-file, which is in the \MathQuiz scripts directory, is read
    first  followed by the rc-file \textsf{.mathquizrc} in the users home
    directory, if it exists. Use
   \begin{center}
       \mathquizopt{rcfile RCFILE} to specify
   \end{center}
    a different rc-file. The following settings may be used:

    \begin{description}
        \item[mathquiz\_url] This is the relative URL for mathquiz web directory
        \item[mathquiz\_web] This is the full path to the mathquiz web
        directory. The on-line manual and other example code can be
        found in the docs subdirectory
        \item[breadcrumbs]
          The default breadcrumbs at the top of quiz page; see \Verb|\Breadcrumbs| in \autoref{SS:breadcrumbs}
          \macroIndex{Breadcrumbs}
        \item[department]
          The default department name; see \Verb|\Department| in \autoref{SS:breadcrumbs}
          \macroIndex{Department}
        \item[department\_url]
          The URL for the department; see \Verb|\DepartmentURL| in \autoref{SS:breadcrumbs}
          \macroIndex{DepartmentURL}
        \item[institution]
          The default institution name; see \Verb|\Institution| in \autoref{SS:breadcrumbs}
          \macroIndex{Department}
        \item[institution\_url]
          The URL for the institution; see \Verb|\InstitutionURL| in \autoref{SS:breadcrumbs}
          \macroIndex{InstitutionURL}
          \item[mathjax] \MathQuiz web pages use \href{https://www.mathjax.org/}{mathjax} to
          render the mathematics on the quiz web pages. By default this
          is done by loading \textsf{mathjax} from
          \begin{center}
             \texttt{https://cdnjs.cloudflare.com/ajax/libs/mathjax/2.7.1/MathJax.js}
          \end{center}
          Fetching \textsf{mathjax} from an external site can
          cause a delay when the quiz web pages are loaded. This setting
          in the rc-file allows you to change where \textsf{mathjax} is
          loaded from. For example, if you install \textsf{mathjax} on
          your web server then you would replace this will the
          corresponding relative URL.
        \item[mathquiz\_mk4]
          Sets a default \textsf{mk4 build file}. This can be used to further
          customise how \TeX4ht and \textsf{make4ht} convert \LaTeX.
          This option was added to \MathQuiz to override how svg images
          were handled but, currently, \MathQuiz does not use a  build
          file. Most people will not need this option. See Chapter~2 of
          the \ctan[make4ht]{make4ht manual} for more information.
        \item[mathquiz\_format]
          Sets the python file that writes the HTML file for the quiz.
          Most people will not need this option. See below for more details.
    \end{description}


  \subsubsection{Changing the format of the \MathQuiz web pages}
  The construction of the online quizzes is controlled by the python
  file \verb!mathquiz_standard.py!. If you want to change the structure
  of the web pages for the quizzes then the easiest way to do this is to make a
  copy of \verb!mathquiz_standard.py!, say \textsf{mathquiz\_myformat.py}, and
  then edit this file directly. This will require some knowledge of
  python and HTML. To see what the new format looks like you can run the
  mathquiz script with an optional argument that tells mathquiz to use
  your format instead:
  \begin{center}
    \textsf{mathquiz \ddash mathquiz\_format mathquiz\_myformat quizfile.tex}
  \end{center}

  \noindent
  You may also want to change the CSS style sheet for mathquiz, which is
  the file \textsf{web/mathquiz.css}. More sophisticated versions of
  \verb!mathquiz_standard.py!  where you change the underlying python
  code are of course possible.  For example, at the University of Sydney
  our version of this python file calls our content management system directly
  and uses this to create the web page for the quiz. Once you have
  finalised the new format you can make this the default format using
  \mathquizopt{edit-settings}.\index{edit-settings}\index{mathquiz\_format}


\section{The on-line MathQuiz manual}\label{S:online}

  \MathQuiz has an
  \href{http://www.maths.usyd.edu.au/u/MOW/MathQuiz/doc/mathquiz-manual.html}{on-line
  manual} that is a \LaTeX file, which uses the \textsf{mathquiz}
  document class,  that converted into a web page using \MathQuiz. The
  PDF version of this manual is included here for your convenience. The
  source file for the on-line manual is included in the documentation of
  \MathQuiz to allow you to create a local version of the on-line
  manual.

  The on-line manual was written for ``internal use'' when \MathQuiz was
  first written in 2004. \MathQuiz has evolved a little since then,
  although mainly on the javascript side. If there are any discrepancies
  between the on-line manual and the earlier sections of this manual
  then the on-line manual should be discounted. There is quite a bit of
  overlap between the on-line manual and previous sections, however,
  more recent functionality such as breadcrumbs is not described in the
  on-line manual.

  The on-line manual has diagrams that are drawn using
  \ctan{pstricks} and, as a result, to create a PDF version of the
  on-line manual use
  \Verb|latex mathquiz-online-manual| to create a \textsf{dvi file}, which can be  converted to PDF using
  \Verb|dvipdf|. If you try to use \Verb|pdflatex| directly you will
  get an error.

  \includepdf[pages=-,pagecommand={\pagestyle{mathquiz}}]{mathquiz-online-manual}

\section{Licence}

Copyright (C) 2013-2017

\href{https://www.gnu.org/licenses/gpl-3.0.en.html}{GNU General Public License, Version 3, 29 June 2007}

This program is free software: you can redistribute it and/or modify it under
the terms of the GNU General Public License (GPL) as published by the Free
Software Foundation, either version 3 of the License, or (at your option) any
later version.

This program is distributed in the hope that it will be useful, but WITHOUT ANY
WARRANTY; without even the implied warranty of MERCHANTABILITY or FITNESS FOR A
PARTICULAR PURPOSE.  See the GNU General Public License for more details.


\vfil
\begin{tabular}{@{}ll}
Authors             & \mathquiz{authors}\\
Description         & \mathquiz{description}\\
Maintainer          & \mathquiz{name}\\
System requirements & \mathquiz{requirements}\\
Licence             & \mathquiz{licence}\\
Release date        & \mathquiz{release date}\\
Repository          & \href{https://\mathquiz{repository}}{\mathquiz{repository}}
\end{tabular}
\eject

\printindex

\end{document}
