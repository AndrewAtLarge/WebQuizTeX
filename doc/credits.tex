\documentclass{webquiz}
\usepackage{color,hyperref}
\newcommand{\WebQuiz}{\textbf{WebQuiz}}

\title{WebQuiz credits}
\UnitCode{WebQuiz}

\begin{document}

\begin{discussion}[Credits]
    \WebQuiz{} was written and developed in the
    \href{http://www.maths.usyd.edu.au/}{School of Mathematics and
    Statistics} at the \href{http://www.usyd.edu.au/}{University of
    Sydney}.  The system is built on \LaTeX{} with the conversion from
    \LaTeX{} to HTML being done by Eitan Gurari's
    \href{http://www.cis.ohio-state.edu/~gurari/TeX4ht/mn.html}{TeX4ht},
    and Michal Hoftich's
    \href{https://github.com/michal-h21/make4ht}{make4ht}.

    To write quizzes using \WebQuiz{} it is only necessary to know
    \LaTeX, however, the \WebQuiz{} system has three components:
    \begin{itemize}
      \item A \LaTeX{} document class file, \texttt{webquiz.cls}, and
      a \TeX 4ht configuration file, \texttt{webquiz.cfg}, that enable the
      quiz files to be processed by \LaTeX{} and \TeX 4ht, respectively.
      \item A python program, \texttt{webquiz}, that translates the xml
      file that is produced by \TeX 4ht into  HTML.
      \item Some javascript and css that controls the quiz web page.
    \end{itemize}

   The \LaTeX{} component of \WebQuiz{} was written by Andrew Mathas
   and the python, css and javascript code was written by Andrew Mathas
   (and Don Taylor), based on an initial protype of Don Taylor's from
   2001.  Since 2004 the program has been maintained and developed by
   Andrew Mathas. Although the program has changed substantially since
   2004 some of Don's code, and his idea of using \TeX 4ht, is still
   in use.

   For information on using \WebQuiz{} see the
   \href{http://www.maths.usyd.edu.au/u/mathas/WebQuiz/webquiz-online-manual.html}{online documentation}.

   Thanks are due to Bob Howlett for general help with CSS and, for
   Version~5, to  Michal Hoftich for technical advice.
\end{discussion}

\end{document}
