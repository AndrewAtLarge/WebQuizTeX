%------------------------------------------ -----the
%  Copyright (C) 2004-2017 Andrew Mathas, University of Sydney
%
%  Distributed under the terms of the GNU General Public License (GPL)
%                  http://www.gnu.org/licenses/
%
% This file is part of the WebQuiz system.
%
% <Andrew.Mathas@sydney.edu.au>
%-----------------------------------------------------------------------------

\synctex=1

\PassOptionsToClass{tikz,svgnames}{xcolor}
\documentclass[svgnames]{article}
\usepackage[a4paper,margin=30mm]{geometry}
\parindent=4mm
\parskip=1mm
\hfuzz 5pt
\synctex=1

\usepackage{textcomp}
\usepackage{pdfpages}
\usepackage{manfnt}
\usepackage{booktabs}

\usepackage{etoolbox}% to patch l@section to remove extraneous spacing
\usepackage{appendix}
\def\sectionautorefname{Chapter}
\def\subsectionautorefname{Section}
\def\subsubsectionautorefname{\S\kern-0.8ex}
\def\appendixautorefname{Appendix}

\usepackage[nonewpage]{imakeidx}
\indexsetup{level=\section*,toclevel=section,noclearpage}
\makeindex[intoc,noautomatic,columns=3]
\indexprologue{\textit{This is an index only for the main \WebQuiz manual. It does not index
the on-line manual.}}

\usepackage{webquiz-doc}

\setcounter{secnumdepth}{3}
\setcounter{tocdepth}{3}
\renewcommand\thesubsubsection{\thesubsection\alph{subsubsection}}
\newcommand\lowerCaseIndex[1]{%
  \lowercase{\def\temp{#1}}%
  \expandafter\index\expandafter{\temp@#1}%
}
% usage: \CrossIndex[*]{main entry}[*]{subentry} with *'s for macros
\NewDocumentCommand\CrossIndex{ smsm }{%
  \IfBooleanTF{#1}{%
    \lowercase{\def\tempa{#2}}%
    \xdef\tempa{\tempa@\noexpand\textbackslash#2}%
  }{\def\tempa{#2}}%
  \IfBooleanTF{#3}{%
    \lowercase{\def\tempb{#4}}%
    \xdef\tempb{\tempb@\noexpand\textbackslash#4}%
  }{\def\tempb{#4}}%
  \expandafter\index\expandafter{\tempa!\tempb}%
  \expandafter\index\expandafter{\tempb}%
}
\newcommand\macroIndex[1]{%
  \lowercase{\def\temp{#1}}%
  \expandafter\index\expandafter{\temp@\textbackslash#1}%
}
\newcommand\gobbleone[1]{}% https://tex.stackexchange.com/questions/318472
\newcommand{\See}[2]{\unskip\emph{see } #1}
\newcommand\SeeIndex[2]{\index{#1!zzzz@\protect\gobbleone|See{#2}}}

\newif\ifCtan\Ctantrue % condition compilation for ctan distribution

\renewcommand*\contentsname{\relax}

% hyperref links to ctan
\NewDocumentCommand\ctan{ O{pkg/#2} m}{\href{https://www.ctan.org/#1}{\texttt{#2}}}
\newcommand\TeXLive{\ctan[/]{\TeX Live}\xspace}
\newcommand\TeXfht{\href{https://www.ctan.org/tex4ht}{TeX4ht}\xspace}
\newcommand\Ctan{\ctan[]{ctan}}
\newcommand\ddash{\texttt{\textemdash\textemdash}}
\newcommand\webquizrc{\index{webquizrc}\BashCode|webquizrc|\xspace}

%%%%%%%%%%%%%%%%%%%%%%%%%%%%%%%%%%%%%%%%%%%%%%%%%%%%%%%%%%%%%%%%%%%%%%%%%%%%%%%%%%%%%%%
%% WebQuiz title box for front page
\usepackage{tikz}
\usetikzlibrary{shadows.blur}

\definecolor{stone}{HTML}{E9E0D8}
\tikzset{shadowed/.style={blur shadow={shadow blur steps=5},
                          top color=stone,
                          bottom color=PapayaWhip,
                          draw=SaddleBrown,
                          shade,
                          font=\normalfont\Huge\bfseries\scshape,
                          rounded corners=8pt,
      },
      boxes/.style={draw=Sienna,
                    fill=Cornsilk,
                    font=\sffamily\small,
                    inner sep=5pt,
                    rectangle,
                    rounded corners=8pt,
                    text=Brown,
     }
}

\def\WebQuizTitle{
  \begin{tikzpicture}[remember picture,overlay]
      \node[yshift=-3cm] at (current page.north west)
        {\begin{tikzpicture}[remember picture, overlay]
          \draw[shadowed](30mm,0) rectangle node[Brown]{WebQuiz} (\paperwidth-30mm,16mm);
          \node[Sienna,font=\normalfont\small\itshape] at (\paperwidth/2,2mm)
          {\small \webquiz{description}};
          \node[anchor=west,boxes] at (4cm,0cm) {\webquiz{name}};
          \node[anchor=east,boxes] at (\paperwidth-4cm,0) {Version \webquiz{version}};
         \end{tikzpicture}
        };
   \end{tikzpicture}
}


%%%%%%%%%%%%%%%%%%%%%%%%%%%%%%%%%%%%%%%%%%%%%%%%%%%%%%%%%%%%%%%%%%%%%%%%%%%%%%%%%%%%%%%
%% headers and footers
\makeatletter
\def\ps@webquiz{
  \ps@empty
  \def\@oddfoot{\tiny\WebQuiz\space -- version \webquiz{version}\hfill%
     \textsc{\ifodd\thepage The WebQuiz manual\else \webquizheader\fi}\hfill\thepage}
}
% hijack section and subsectionmark for our headers
\def\webquizheader{\WebQuiz}
\def\sectionmark#1{\def\webquizheader{#1}}
\def\subsectionmark#1{\def\webquizheader{#1}}
\pagestyle{webquiz}
\patchcmd\l@section{1.0em}{0.5em}{}{}
\makeatother

\begin{document}
\hypersetup{pdftitle={WebQuiz manual}}

    \WebQuizTitle

    \begin{quote}
      \WebQuiz providing an easy way to write on-line quizzes using
      a ``standard'' \LaTeX{} file. This allows
      the quiz author to concentrate on the content of quizzes, written
      in \LaTeX, unencumbered by the technicalities of HTML and
      javascript. \WebQuiz was originally developed for writing
      mathematics quizzes, however, the quizzes can cover any material
      that can be reasonably typeset using \LaTeX.

      \ScreenShot[0.6]{Example \WebQuiz web page}{examples/quiz-page}

      \begin{center}
        \hfil
        \begin{minipage}{0.7\textwidth}
          \tableofcontents
        \end{minipage}
      \end{center}
    \end{quote}

    \newpage

\section{Introduction}
    On-line quizzes provide a good way to reinforce learning, especially
    because they can give ``interactive'' feedback to the
    students\footnote{Throughout this manual, ``student'' means the
    person taking the on-line quiz.} based on the answers that they
    give. Unfortunately, in addition to writing the quiz content there
    are significant technical hurdles that need to be overcome when
    writing an on-line quiz -- and there are additional complications if
    the quiz involves mathematics or diagrams.

    \WebQuiz makes it possible to write on-line quizzes using \LaTeX,
    which is the typesetting language used by mathematicians who use
    \LaTeX\ to write their research papers, books and teaching
    materials. In principle, a \WebQuiz quiz can contain anything that
    can be typeset using \LaTeX.  In practise, the \LaTeX\ is converted
    to HTML using \TeXfht (and
    \href{https://github.com/michal-h21/make4ht}{make4ht}), so the
    quizzes can contain any \LaTeX commands that are understood by \TeX
    4ht, which is almost everything. In particular, it is possible to
    use graphics constructed using packages like \ctan{pstricks} and
    \ctan{tikz}; see \autoref{SS:graphics}.

    \WebQuiz supports the following three types of questions:
    \begin{itemize}
      \item Multiple choice questions with a unique correct answer
      \item Multiple choice questions zero or more correct answers
      \item Questions with an answer that is supplied by the student.
    \end{itemize}
    Each time a student answers a question it is possible to give them
    feedback, reinforcing their learning when they answer correctly and
    giving them further hints when they are wrong. This allows the
    quiz author to give targeted feedback to the student based on their
    answer.

    The on-line quizzes constructed using \WebQuiz can, in principle,
    contain anything that can be typeset by \LaTeX. In particular, they
    do not need to contain mathematics. In fact, the quizzes do not even
    have to contain ``questions'' as it is possible for a \WebQuiz
    ``quiz'' to contain only \LatexCode|discussion| environments that can
    be used to revise material, or to introduce new material, for the
    students; see \autoref{S:documentclass}.

    This introduction outlines how to use \WebQuiz, however, the
    impatient reader may want to skip ahead directly to the
    \autoref{S:documentclass}, where the \LaTeX\ commands used by
    \WebQuiz are described.

    The easiest way to explain how \WebQuiz works is by example. The
    following \LaTeX\ file defines a quiz with a single multiple choice
    question that has four possible answers, each of which has a
    customised response. Giving feedback to the students in each
    question is optional but the capability of being able to give
    students feedback on their answer is one of the main pedagogical
    advantages of on-line quizzes.

    \InputLatexCode{examples/simple}

    Since this is a \LaTeX\ file it can be processed using
    \BashCode|pdflatex|, or \BashCode|latex|, to produce a readable and
    printable version of the quiz, which can be useful when
    proofreading. In this case, the \LaTeX\ version of the quiz looks
    like this:
    \ScreenShot[0.5]{Sample PDF output}{examples/simple-pdf}
    Of course, the real reason for using \WebQuiz is that it is also
    possible to make an on-line version of the quiz by processing the
    quiz using the \BashCode|webquiz| command. If you do this, open
    the resulting web page in your favourite browser and select answer~(a),
    then you will see a web page like this:
    \ScreenShot{Sample web page}{examples/simple-html}
    By default, the on-line version of the quiz displays one question at a time,
    with the question buttons serving the dual purpose of navigation
    between questions and displaying how successful the student was
    in answering the question. The decorations on the question
    buttons indicate whether the question has been attempted and, if so,
    whether it was answered correctly or incorrectly on the first of
    subsequent attempts. One of the main points of \WebQuiz is that
    targeted feedback can be given to the person taking the quiz based
    on their responses.

    The underlying engine used by \WebQuiz is \TeXfht so, because
    \TeXfht is not able process all \LaTeX{} code, there is \LaTeX{}
    code that \WebQuiz is not able to cope with.  This said, \TeXfht is
    able to display \textit{most} \LaTeX{} code and \WebQuiz has been
    used to write literally thousands of quiz questions so it is likely
    that you will be able to typeset what you want in your on-line
    quizzes.  In particular, it is possible to use \ctan[pgf]{tikz} and
    \ctan{pstricks} in \WebQuiz quizzes, although complicated
    drawings with these packages may fail; see \autoref{SS:graphics} for
    more details.

    \newpage
\subsection{What WebQuiz does and does not do}

    \WebQuiz is a tool that makes it possible to write ``interactive''
    on-line quizzes using \LaTeX{}. To use \WebQuiz you only need to
    know how to use \LaTeX{} and, in particular, no knowledge of the
    underlying HTML CSS, or javascript is required. \WebQuiz can be used
    to ask students a series of ``quiz'' questions.

    By default, the on-line quizzes display one question at a time.  It
    is also possible to display all questions at once. One of the key
    features of \WebQuiz is that you can give feedback to the students
    based on their answers. In this way you can give hints to the
    students to correct their mistakes and you can reinforce the
    students' understanding when they are correct. Each question in a
    quiz, and each quiz itself, can be attempted as many times as the
    student wants: \WebQuiz does not limit the number of times that
    questions can be attempted.

    The following question types are currently supported:
    \begin{itemize}
      \item Multiple choice questions with a unique correct answer
      \item Multiple choice questions zero or more correct answers
      \item Questions with an answer. There are several different
      ``comparison'' methods available so that, for example, the
      inputted answers can be a ``string'' or a number.
    \end{itemize}

    Questions can appear in either the same order that they appear in
    the \LaTeX{} file for the quiz or in a random order that changes
    each time the quiz page is loaded. For multiple choice questions the
    order in which the choices appear is fixed, even if the questions
    appear in random order.

    \WebQuiz supports several different languages and it
    provides a number of different colour schemes (see
    \autoref{SS:commandline} and \hyperref[SS:themes]{Appendix A}). Any additional
    languages files that are sent to me, as well as corrections to
    existing languages, will be incorporated in future releases.

    The quizzes are not timed and do not have time-limits.

    The quizzes made using \WebQuiz are intended to be used as a
    revision resource rather than as an assessment tool. In particular,
    \WebQuiz does not provide a mechanism for recording the marks
    obtained by the students taking the quiz. Technically, it probably
    would not be very hard to record marks but this introduces a
    significant amount of extra overhead in terms of student
    authentication and interfacing with a database. In addition, if
    \WebQuiz were used as an assessment tool then there would be
    additional ``security issues'' to ensure that the quiz content is
    secure. Currently, even though the solutions to the quiz questions
    do not appear in the HTML source code for the quiz pages it is
    possible to access the answers if you know what you are doing.

    The questions in a \WebQuiz quiz are static. In particular, they do
    not accept variables.

    The \WebQuiz program was designed to be run from the command-line.
    To process the file \BashCode|quiz.tex| using \WebQuiz you would
    type

    \BashCode|> webquiz quiz| \qquad or \qquad
    \BashCode|> webquiz quiz.tex|

    \noindent from the command-line. Although this is untested, it
    should be possible to run \WebQuiz from inside programs like \TeX
    Shop by setting the compiler equal to \BashCode|webquiz|.

\subsection{Credits}
    \WebQuiz{} was written and developed in the
    \href{http://www.maths.usyd.edu.au/}{School of Mathematics and
    Statistics} at the \href{http://www.usyd.edu.au/}{University of
    Sydney}.  The system is built on \LaTeX{} with the conversion from
    \LaTeX{} to HTML being done by Eitan Gurari's
    \href{http://www.cis.ohio-state.edu/~gurari/TeXfht/mn.html}{TeX4ht}
    and
    \href{https://github.com/michal-h21/make4ht}{make4ht}.

    To write quizzes using \WebQuiz it is only necessary to know
    \LaTeX, however, the underlying \WebQuiz system actually has three
    components:
    \begin{itemize}
      \item A \href{https://www.latex-project.org/}{\LaTeX} document class
      file, \BashCode|webquiz.cls|, and a \TeXfht
      configuration file, \BashCode|webquiz.cfg|, that enables the
      quiz files to be processed by \LaTeX{} and \TeXfht, respectively.
      \item A \href{https://www.python.org/}{python} program,
      \BashCode|webquiz|, that translates the
      \LaTeX{} into xml, using \TeX 4ht, and then into HTML.
      \item
      \href{https://www.w3schools.com/html/html_intro.asp}{HTML},
      \href{https://www.w3schools.com/css/}{CSS} and
      \href{https://www.w3schools.com/Js/}{Javascript}
      code that works behind the scenes to control and style the quiz
      web pages.
    \end{itemize}

   The \LaTeX{} component of \WebQuiz{} was written by Andrew Mathas and
   the python, css and javascript code was written by Andrew Mathas (and
   Don Taylor), based on an initial prototype of Don Taylor's from 2001.
   Since 2004 the program has been maintained and developed by Andrew
   Mathas. Although the program has changed substantially since 2004
   some of Don's idea of using \TeXfht, and some of his code, is still
   in use. Prior to releasing \WebQuiz on \Ctan, the program was known
   as \texttt{MathQuiz}.

   Thanks are due to Bob Howlett for general help with CSS and to
   Michal Hoftich for invaluable technical advice on \TeXfht.

 \section{The WebQuiz document class --- \LaTeX{} commands}
 \label{S:documentclass}

  The aim of \WebQuiz is to write on-line quizzes using only \LaTeX.
  This chapter describes the \WebQuiz commands and environments and how
  they put material onto a web page.  All of the code examples given in
  this, and other sections, can be found in the \BashCode|examples|
  directory of the \WebQuiz web directory.%
  \footnote{Once \WebQuiz is initialised you can find the \WebQuiz web
  directory using the command:\newline
  \hspace*{10mm}\lstinline[style=bashcode]|webquiz --settings webquiz-www|.}
  More details and additional examples can be found in the on-line manual,
  which is included, in PDF form, as \hyperref[S:online]{Appendix B}.

  This chapter describes the commands and environments provided by the
  \WebQuiz document class. This assumes that you have already installed
  and configured \WebQuiz. If you have not done initialised \WebQuiz
  then follow the instructions in \autoref{S:configuration}.

  A typical \WebQuiz quiz file is a \LaTeX{} file of the form:

    \InputLatexCode{examples/typicalquizfile}

  \noindent You should write your quizzes using the editor that you
  normally use to write \LaTeX. As you write your quiz, say
  \texttt{quizfiule.text}, you should use (\BashCode|latex| or)
  \BashCode|pdflatex| in the usual way:
  \begin{bashcode}
      > pdflatex quizfile
  \end{bashcode}
  This is the easiest way to check that your quiz compiles and to
  proof-read the output, just as if you were writing a normal
  \LaTeX{} document. When you are satisfied with the content of the
  quiz, then you can convert the quiz to an on-line quiz using \WebQuiz
  using the command
  \begin{bashcode}
     > webquiz quizfile
  \end{bashcode}
  from the command-line prompt \BashCode|>|. The quiz file,
  \BashCode|quizfile.tex|, should be on your web server because \WebQuiz
  creates a number of different files and directories when it converts
  the file into an on-line quiz and all of these files are needed to
  display the quiz on the web.

  The reasons for using this workflow include: \begin{itemize} \item
    \textit{Every file that you give to \WebQuiz should be a valid
    \LaTeX{} file!}

    \item The \BashCode|dvi| or \BashCode|PDF| file produced by \LaTeX{}
    will show all of the information about the questions
    \textit{in an easy-to-read format}. That is, the PDF file will
    display the question, the answers and the feedback that you are
    giving to the students. In contrast, by design, the on-line version
    of the quiz hides most of this information and displays it to the
    student only when they need to see it.

    \item Typesetting the quiz file with \LaTeX{} is \textit{much
    faster} than processing it with \WebQuiz. In fact, \WebQuiz uses
    \BashCode|htlatex| to process the quiz file at least \textit{three times} in
    order to produce an \BashCode|xml| file and it is only then that the
    \WebQuiz program kicks in to rewrite this data as an \BashCode|HTML|
    file.

    \item If \LaTeX{} produces errors then \WebQuiz will produce more
    errors. Further, \textit{\LaTeX{} error messages are much easier to read and
    understand than those produced by \TeXfht and \WebQuiz}.
  \end{itemize}
  This said, \WebQuiz does check for more errors in the quiz than
  \LaTeX{} is (easily) able to do.

  The PDF version of a quiz does not contain information about
  the unit, department or institution, which can be used in the
  breadcrumbs. This information was included in previous releases of
  \WebQuiz, however, this data is now omitted from the PDF because this
  data can now be specified in the \webquizrc file (see
  \autoref{SS:rcfile}) and \LaTeX{} does not have access to the
  \webquizrc file.

  The next sections describe the commands and environments provided by
  \WebQuiz for typesetting quizzes as well as the document-class options
  for the package. If you plan to use \ctan{pstricks} or
  \ctan[pgf]{tikz} then you should read \autoref{SS:graphics}, which
  describes how to use graphics in a \WebQuiz quiz.
  \autoref{SS:config} describes a work-around for using (some?) \LaTeX{}
  features that have not been configured for use with \TeXfht.

\subsection{Question environments}

The \WebQuiz document class defines the following four environments:
\begin{quote}
  \begin{description}
    \item[question] Each quiz question needs to be inside
    \LatexCode|question| environment
    \item[choice] For typesetting multiple choice questions, with
    one or more correct answers
    \item[discussion] For including (optional) discussion, or revision,
    material at the top of the quiz
    \item[quizindex] For writing an index file for the quizzes in
    one ``unit of study''.
  \end{description}
\end{quote}
This section describes these environments and gives examples
of their use.

\subsubsection{Question environments and the \textcolor{blue}{\textbackslash answer} macro}
\CrossIndex{environment}{question}

Quizzes are composed of questions and each quiz question must to be
placed inside a \LatexCode|question| environment. Typically, a quiz will have
several questions, each wrapped in its own \LatexCode|question| environment.
For brevity, most of the examples in this chapter have only one
question. See the on-line manual (\hyperref[S:online]{Appendix B}), in the \WebQuiz web
directory for a complete quiz file.

  \InputLatexCode{examples/answer-string}
  \macroIndex{answer}\CrossIndex{question environment}*{answer}
  \CrossIndex*{answer}*{whenRight}
  \CrossIndex*{answer}*{whenWrong}
  \index{feedback!whenRight}
  \index{feedback!whenWrong}

  The \textit{optional} macros \LatexCode|\whenRight| and
  \LatexCode|\whenWrong| are used to give the student additional
  feedback, when they are right, or further hints etc for approaching
  the question, when they are wrong. This feedback is displayed on the
  quiz page only when the student checks their answer.

  The web page created by the code above, when an answer of
  ``canberra'', instead of ``Canberra'' is given, is:
  \ScreenShot{A question with an answer}{examples/answer-string}
  This example code shows one way of using the \LatexCode|\answer| macro, which
  asks for the student to type in an answer to the question. The syntax of the
  \LatexCode|\answer| macro is:
  \begin{latexcode}
      \answer[comparison type]{correct answer}
  \end{latexcode}
  where the optional \textit{comparison type} is one of:
  \begin{description}
      \item[complex] \CrossIndex*{answer}{complex}
      The \textit{correct answer} and the inputted answer
      are compared as \textit{complex} numbers
      \item[integer] \CrossIndex*{answer}{integer}
      The \textit{correct answer} and the inputted answer
      are compared as \textit{integers}
      \item[lowercase] \CrossIndex*{answer}{lowercase}
      The \textit{correct answer} and the inputted answer
      are converted to lower case and then compared as strings
      \item[number] \CrossIndex*{answer}{number}
      The \textit{correct answer} and the inputted answer
      are compared as (real) \textit{numbers}. For example, if the
      \textit{correct answer} was $0.8$ then $4/5$ would be marked as
      correct.
      \item[string] \CrossIndex*{answer}{string}
      The \textit{correct answer} and the inputted answer are compared
      as \textit{strings}, so they have to match exactly. In particular,
      \textit{string} is case sensitive. This is the default.
      % \item[eval] \CrossIndex*{answer}{eval}
      % The inputted answer is marked correct if the \textit{difference}
      % of the \textit{correct answer} and the inputted answer
      % evaluates, as a mathematical expression, to $0$. Standard
      % functions such as $\sin(x)$, $\cos(x)$ etc. are accepted.
  \end{description}

  Since \textit{string} is the default comparison type, the example
  code above is shows how string comparisons work. The following code
  blocks and screen shots give examples of the different \LatexCode|\answer|
  comparison types.

  \CrossIndex*{answer}{complex}
  \textcolor{blue}{Complex answer} --- the answers are compared as complex
  numbers: the answer is marked as correct if it has the same real and
  imaginary parts.
  \InputLatexCode{examples/answer-complex}
  \ScreenShot{A question with a complex answer}{examples/answer-complex}

  \textcolor{blue}{Integer answer} --- the answers are compared as
  integers. If the correct answer was $18$ and the students entered
  $36/2$ then they would be marked wrong.
  \CrossIndex*{answer}{integer}
  \InputLatexCode{examples/answer-integer}
  \ScreenShot{A question with a integer answer}{examples/answer-integer}

  \textcolor{blue}{Lowercase string answer} --- the quiz answer and the
  students' answer are both converted to lower case and then compared as
  strings.
  \CrossIndex*{answer}{lowercase}
  \InputLatexCode{examples/answer-lowercase}
  \ScreenShot{A question with a lowercase string answer}{examples/answer-lowercase}

  \textcolor{blue}{Numeric answer} --- the quiz answer and the students'
  answer are compared as numbers. For example, if the answer was $0.75$ and the student
  answered $\frac 34$ then they would be marked correct.
  \CrossIndex*{answer}{number}
  \InputLatexCode{examples/answer-number}
  \ScreenShot{A question with a numeric answer}{examples/answer-number}

  \textcolor{blue}{String answer answer}
  This is the default: \LatexCode|\answer{some string}| and
  \LatexCode|\answer[string]{some string}| are equivalent. The student's
  answer is marked correct if it agrees exactly with the quiz answer.
  \CrossIndex*{answer}{string}

  \subsubsection{Multiple choice questions}
  \CrossIndex{environment}{choice}
  \CrossIndex{question environment}{choice environment}
  \SeeIndex{multiple choice}{choice environment}

The multiple choice options for a quiz question need to be placed inside
a \LatexCode|choice| environment. The \LatexCode|choice| environment
accepts two optional arguments, which can appear in any order:
\begin{itemize}
  \item\CrossIndex{choice environment}{single}\CrossIndex{choice environment}{multiple}
  The word \LatexCode|single| (default, and can be omitted) or
  \LatexCode|multiple|, which indicates whether the quiz has a
  \textit{single} correct answer or whether 0 or more of the answers are
  correct, respectively.
  \item \index{choice environment!columns}
  The number of \textit{columns} in which to typeset the choices. This
  is specified as \LatexCode|columns=n|, where $n$ is a non-negative integer.
  By default, the choices appear in $n=1$ columns
  (\DefaultValue{columns=1}).
\end{itemize}
The key difference between these two types of \LatexCode|choice|
questions is that a \LatexCode|single|-choice environment uses radio
boxes, so it is only possible to select one correct answer, whereas in a
\LatexCode|multiple|-choice environment checkboxes are used, so that is
possible to select zero or more correct answers.  Every
\LatexCode|choice| environment needs to be inside a \LatexCode|question|
environment.

A \LatexCode|choice| environment is similar to the \LaTeX{} list environments
(\textit{enumerate}, \textit{itemize}, \textit{description}, ...) except
that instead of using \LatexCode|\item| to separate the items
\LatexCode|\correct| \CrossIndex{choice environment}*{correct}
\LatexCode|\incorrect| \CrossIndex{choice environment}*{incorrect} are
used, indicating correct and incorrect responses respectively.
In addition, after each \LatexCode|\correct| or \LatexCode|\incorrect|
you can , optionally, use
\LatexCode|\response| \CrossIndex{choice environment}*{response}
to give a feedback response to the person taking the quiz. Like
\LatexCode|\whenRight| and \LatexCode|\whenWrong| this feedback is
displayed only when the student checks their answer.
\index{feedback!response}

Here is an example of a \LatexCode|single|-choice question with a unique answer:
  \InputLatexCode{examples/choice-single}
It is not necessary to put the \LatexCode|\response| lines on the same
line as the \LatexCode|\incorrect| and \LatexCode|\incorrect|; this is
done only to make the example more compact. This results in the
following web page:
\ScreenShot{Single answer multiple choice question}{examples/choice-single}

Here is an example of a multiple choice question that has
two correct answers:
  \InputLatexCode{examples/choice-multiple}\label{choice-multiple}
Notice that this example uses the documentclass-option
\LatexCode{theme=vibrant}, which changes the colour scheme:
\ScreenShot{Multi-answer multiple choice question}{examples/choice-multiple}

When the optional argument \LatexCode|multiple| is used, as above, then the
question is marked correct if and only if all of the correct responses
are selected. Feedback to the students is randomly selected from the
feedback for their incorrect responses.

Finally, observe that the multiple choice items in the screenshot above
are labelled by roman numerals. The items in a \LatexCode{choice} are
labelled by a standard \LaTeX{} counter, which is also called
\LatexCode|choice|. Redefining \LatexCode|\thechoice| changes how the
items in the on-line quiz are labelled. For example, to label the items
in a \LatexCode|choice| environment by A), B), C) $\dots$ add
\begin{latexcode}
  \renewcommand\thechoice{\Alph{choice})}
\end{latexcode}
to the preamble of the \LaTeX{} file for your quiz.
\CrossIndex{choice environment}*{thechoice}\macroIndex{thechoice}

  \subsubsection{Discussion environments}
  \CrossIndex{environment}{discussion}

In addition to asking questions it is possible to have revision, or
discussion, material \textit{before} the question in the quiz. The
discussion items always appear before the quiz questions in the menu
down the left-hand side --- it is not possible to interleave discussion
items and questions in the side menu. Each quiz can have zero or more
discussion environments (and zero or more question environments) and
these environments can, in principle, contain arbitrary \LaTeX{} code.

For example, running the \LaTeX\ file through \WebQuiz
  \InputLatexCode{examples/discussion}
produces the web page:
\ScreenShot[0.7]{Web page: discussion environment}{examples/discussion}
As with the questions, only one \LatexCode|discussion| environment is
displayed on the quiz web page at a time (unless the document-class
option \LatexCode|onepage| is used). It is possible to have quizzes
that contain only \LatexCode|discussion| environments, with no questions, and
quizzes that contain only \LatexCode|question| environments, with no
discussion.

\subsubsection{Index pages for qizzes}\label{SS:index}

  \index{quizindex environment}
  \CrossIndex{environment}{quizindex}

  Most quizzes occur in sets that cover related material. For example,
  quizzes that cover the material each week in a particular unit of
  study. The \LatexCode|quizindex| environment can be used to create an
  index web page for related sets of quizzes, all of which must appear in the same
  directory, or folder. The index web page is a \WebQuiz file of the
  form:
  \InputLatexCode{examples/index}
  which generates a web page that looks like:
  \ScreenShot{Example index page}{examples/index}

  As the next section describes, index files are also used to
  automatically add a drop-down menu to the breadcrumbs on the quiz web
  pages. This drop-down menu accesses all of the quizzes for the unit.
  As the example above shows, the entries in the \LatexCode|quizindex| are all
  given using the \LatexCode|\quiz| command.
  \CrossIndex{quizindex environment}*{quiz}
  The syntax for this command is

  \begin{latexcode}
      \quiz[URL for quiz]{Title for quiz}
  \end{latexcode}

  \noindent The \LatexCode|\quiz| macro automatically inserts the quiz
  numbers into the index listing and, by default, it assumes that the
  URL for the quiz is of the form \BashCode|quiz1.html|,
  \BashCode|quiz2.html|, \BashCode|quiz3.html|, .... If the URL for the quiz
  does not conform to this pattern then it can be given as a optional
  argument to \LatexCode|\quiz|.

  \subsubsection{Breadcrumbs}\label{SS:breadcrumbs}\index{breadcrumbs}

  \WebQuiz provides a straightforward way to place navigation breadcrumbs
  at the top of the quiz web page. By default the
  breadcrumbs are disabled. Adding \LatexCode|\BreadCrumbs| command
  like:
  \begin{latexcode}
      \BreadCrumbs{Mathematics /|Math1001 /u/Math1001|quiz_index|title}
  \end{latexcode}
  adds the following breadcrumbs, or navigation links,
  to the top of the quiz page:
  \ScreenShot{Example breadcrumbs}{examples/breadcrumbs}
  The breadcrumbs are typically navigation links\footnote{In a web browser all of the
  parts of the breadcrumbs have the same height. In contrast, in this
  manual the ``Quizzes'' breadcrumb is uneven. This is just an
  unfortunate artefact of how the images are generated for the manual.}.
  In the example above:
  \begin{itemize}
    \item clicking on \BashCode|Mathematics| takes you to the
    \textit{relative URL} \BashCode{/}, the root directory for the web
    server, which is often the correct URL for the department (or the
    institution)
    \item clicking on
    \BashCode|MATH1001| takes you to the
    unit of study web page, which has relative URL \BashCode{/u/Math1001}, and,
    \item clicking on \BashCode|Quizzes| takes you to the index page for
    the quizzes for the current unit (see \autoref{SS:index}), that, by
    default, has URL \textit{unit URL}\BashCode|/Quizzes|.  As alluded
    to in the last section, clicking on the~{\large\color{red} $\equiv$}
    symbol after \textit{Quizzes} causes a drop-down menu to appear that
    contains links to all of the quizzes for the current unit of study:
    \ScreenShot{Drop-down menu giving index of quizzes}{examples/dropdown.png}
    This drop-down menu  will appear only if the current directory
    contains a \WebQuiz index file that has been compiled.
    \CrossIndex{breadcrumbs}{quizindex}.
  \end{itemize}
  The breadcrumbs for the quiz web page can be either be configured
  quiz-by-quiz, using the \LatexCode|\BreadCrumbs| macro
  \macroIndex{BreadCrumbs} as above, or by setting a default for the
  \LatexCode|breadcrumbs|  in the \webquizrc file using the
  command-line option
  \begin{bashcode}
    > webquiz --edit-settings
  \end{bashcode}
  as described in \autoref{SS:commandline}.
  (The breadcrumb are disabled by default.)
  The breadcrumbs are given as a ``|-separated list''. For example,
  quite reasonable breadcrumbs are given by:
  \begin{latexcode}
    \BreadCrumbs{ department | unitcode | quiz_index | title }
  \end{latexcode}
  To make this the default set of breadcrumbs use
  \BashCode{webquiz --edit-settings} to set breadcrumbs
  in the \webquizrc file to:
  \begin{latexcode}
     department | unitcode | quiz-index | title
  \end{latexcode}
  More generally, the breadcrumbs can be specified as:
  \begin{latexcode}
    \BreadCrumbs{ crumb1 | crumb2 | crumb3 | crumb4 | ...  }
  \end{latexcode}
  where the crumbs are interpreted as described below.
  In principle, there can be arbitrarily many crumbs in your
  breadcrumbs but, in practice, five is more than enough.

  The \LatexCode|\BreadCrumbs| command accepts the following ``magic crumbs'':

  \begin{description}
    \item[breadcrumb] The breadcrumb for the current quiz, which is set using \LatexCode|\BreadCrumb|.
    This breadcrumb is purely descriptive with only
    the \LatexCode|\BreadCrumb| text appearing. In particular, no
    hyperlink is added.
    \CrossIndex*{BreadCrumb}*{breadcrumb}

    \item[department] This expands to a link to your department, where
    the department text is set using \LatexCode|\Department| and its URL is set by
    \LatexCode|\DepartmentURL|.
    \macroIndex{Department}\macroIndex{DepartmentURL}
    \CrossIndex*{BreadCrumb}{department}

    \item[institution] This expands to a link to your institution, where
    the institution text is set using \LatexCode|\Institution| and its URL is set by
    \LatexCode|\InstitutionURL|. The institution also appears in the
    side-menu above the \WebQuiz copyright notice.
    \macroIndex{Institution}\macroIndex{InstitutionURL}
    \CrossIndex*{BreadCrumb}{institution}

    \item[quiz-index] This expands to ``Quizzes'', which is a link to
    the index page for your unit, as defined by \LatexCode|\QuizzesURL|. In
    addition, a drop-down menu to the index page is added to the
    breadcrumbs, which is accessed via {\large\color{red} $\equiv$}.
    \macroIndex{QuizzesURL}
    \CrossIndex*{BreadCrumb}{quiz-index}

    \item[Title] This expands to the full title of the quiz page, as given by
    the \LatexCode|\title|.
    \CrossIndex*{BreadCrumb}*{Title}

    \item[title] This expands to the part of title of the quiz page, as given by
    the \LatexCode|\title| command, that is \textit{before} the first
    colon ``:'' in the title. So, for example, if the title is
    ``Quiz 1: The wonders of life'' then ``Quiz 1'' will be added
    to the list of breadcrumbs.
    \CrossIndex*{BreadCrumb}*{title}

    \item[unitcode] This expands to a link to the unit code, where
    the unit code text is set using \LatexCode|\UnitCode|, and its URL is set by
    \LatexCode|\UnitURL|
    \macroIndex{UnitCode}\macroIndex{UnitURL}
    \CrossIndex*{BreadCrumb}{unitcode}

    \item[unitname] This expands to a link to the unit name, where
    the unit name text is set using \LatexCode|\UnitName|, and its URL is set by
    \LatexCode|\UnitURL|
    \macroIndex{UnitName}\macroIndex{UnitURL}
    \CrossIndex*{BreadCrumb}{unitname}

  \end{description}

  \noindent
  In addition, each \textit{crumb} in a breadcrumb, except for the
  ``magic crumbs'' listed above, is allowed to be arbitrary text. In
  this case, the last ``word'' in the crumb is treated as a URL if it
  either beings with a backslash, $\backslash$, or it begins with
  \BashCode|http|.  For example, the code:
  \InputLatexCode{examples/montypython}
  \noindent
  results in the following breadcrumbs:
  \ScreenShot{Guaranteed to offend some one}{examples/montypython}
  Notice that it is necessary to correctly escape spaces etc in
  URLs that are specified this way.

  If any part of a ``magic'' breadcrumb has not been defined it will be
  printed with question marks on the web page. For example, the quiz
  file
  \InputLatexCode{examples/nounits}
  leads to the breadcrumbs:
  \ScreenShot{Breadcrumbs with no unit code}{examples/nounits}

  Finally, here is the list of \WebQuiz commands that can be used to
  specify the contents of the breadcrumbs. Default values for many of these
  ``crumbs'' can be given in the \webquizrc file; see \autoref{SS:rcfile}.

  \newcommand\Item[1]{\item[\textbackslash#1]\CrossIndex{breadcrumbs}*{#1}}
\begin{description}
  \Item{BreadCrumb}\macroIndex{BreadCrumb}
    The \LatexCode|\BreadCrumb| command sets the \LatexCode|breadcrumb|
    variable in the \LatexCode|\BreadCrumbs|. The primary use for this
    is when you have default breadcrumbs in the \webquizrc file like
    \begin{bashcode}
        breadcrumbs = department | unitcode | breadcrumb
    \end{bashcode}
    Using \LatexCode|\BreadCrumb| allows you to set the last crumb to
    something that describes the quiz.

  \Item{Department}\macroIndex{Department}
    The \LatexCode|\Department| command sets the name of the
    \LatexCode|department|. As described earlier in this section, by
    default, the \LatexCode|department| is the first item in the
    breadcrumbs that appear at the top of the web page.

    The default department can be set in the \webquizrc file using
    \BashCode|webquiz --edit-settings|.

    Default value: \BashCode{''}

  \Item{DepartmentURL}\macroIndex{DepartmentURL}
    The \LatexCode|\DepartmentURL| command sets URL for the department. As
    described earlier in this section, by default the department URL is
    the link in the first breadcrumb on each web page.

    The default department URL can be set in the \webquizrc file using
    \BashCode|webquiz --edit-settings|.

    Default value: \DefaultValue{/}

    \Item{Institution}\macroIndex{Institution}
    The \LatexCode|\Institution| command sets the institution, or university.
    The \LatexCode|institution| appears below the question buttons in the
    left-hand navigation menu that appears on every quiz web page
    (provided that the screen size is not too small).  As described
    earlier in this section, the institution can be used in the web
    page breadcrumbs.

    The default institution can be set in the \webquizrc file using
    \BashCode|webquiz --edit-settings|.

    Default value: \BashCode{''}


    \Item{InstitutionURL}\macroIndex{InstitutionURL}
    The \LatexCode|\InstitutionURL| command sets the institution URL. This is
    used as the link for the \LatexCode|institution| in the left-hand
    navigation menu that appears on every quiz page.  As described
    earlier in this section, the institution URL can be used in the
    web page breadcrumbs.

    The default institution URL can be set in the \webquizrc file using
    \BashCode|webquiz --edit-settings|.

    Default value: \DefaultValue{/}

    \Item{QuizzesURL}\macroIndex{QuizzesURL}
    The \LatexCode|\QuizzesURL| command sets the URL for the suite of quizzes
    attached to this unit of study. As described earlier in this
    section, this can be used in the breadcrumb at the top of the quiz
    web page.

    Default value:  \DefaultValue{UnitURL/Quizzes}, where \DefaultValue{UnitURL} is set
    using \LatexCode|\UnitURL|

    \Item{UnitCode}\macroIndex{UnitCode}
    The \LatexCode|\UnitCode| command sets the unit of study code for the
    unit that the quiz is part of.

    \Item{UnitName}\macroIndex{UnitName}
    The \LatexCode|\UnitName| command sets the name of the unit of study for
    the unit that the quiz is attached to.

    \Item{UnitURL}\macroIndex{UnitURL}
    The \LatexCode|\UnitURL| command sets the URL for the unit of study code
    for the unit that the quiz is attached to.

\end{description}

\subsection{WebQuiz class options}\label{SS:classOptions}

  The \textbf{webquiz} document class: supports the following document
  class options:
  \begin{latexcode}
    fixedorder, hidesidemenu, language, onepage, pst2pdf,
    separatepages, showsidemenu, theme, tikz
  \end{latexcode}
  This section describes all of these document-class options except for
  \LatexCode|tikz| and \LatexCode|pst2pdf|, which are discussed in
  \autoref{SS:graphics}. Many of the document-class options below occur
  in pairs and defaults for all of them can be set in the \webquizrc
  file; see \autoref{SS:rcfile}. The settings given in the \LaTeX{} file
  for the quiz override the default settings in the \webquizrc file.

\begin{description}
  \item[fixedorder, randomorder]
  \CrossIndex{document class options}{fixedorder}
  \CrossIndex{document class options}{randomorder}
    By default, the questions in the quiz are displayed in
    \DefaultValue{fixedorder}, which is the order that they appear in
    the \LaTeX{} file for the quiz. That is, question~1 is the first
    question appearing in the \LaTeX file,m question~2 is the second
    question and so on. With the \LatexCode|randomorder| option the
    questions in the quiz will be displayed in a random order that
    changes each time that the quiz is run and so is generally different for
    each student. For example, the code:
    \InputLatexCode{examples/random}
    produces the quiz with randomly arranged quiz questions:
    \ScreenShot{Randomly ordered questions}{examples/random}
    When using the \LatexCode|randomorder| document-class option only
    the questions appear in random order. If the quiz contains multiple
    choice questions then the choices are \textit{not} randomly
    permuted. That is, the choices always appear in the order that they
    are written in the \LaTeX{} file.

  \item[sidemenu, hidesidemenu]
  \CrossIndex{document class options}{hidesidemenu}
  \CrossIndex{document class options}{showsidemenu}
    If this option is set the end the side menu is not displayed
    initially. By default, the side menu appears unless the screen size
    is too small, such as on a mobile phone. Many examples of the
    \LatexCode|hidesidemenu| and \LatexCode|showsidemenu| class options
    can be found above above.

  \item[onepage, separatepages]
  \CrossIndex{document class options}{onepage}
  \CrossIndex{document class options}{separatepages}
  By default, only one question, or discussion environment, is displayed
  by the quiz at any time. As \LatexCode|separatepages| is the default, every
  example so far is of this form. With the document-class option
  \LatexCode|onepage| all questions and discussion environments are
  displayed at the same time. So, for example, the code:
  \InputLatexCode{examples/onepage}
  produces the web page:
  \ScreenShot{A one page quiz}{examples/onepage}

  \item[language=<lang>] \CrossIndex{document class options}{language}
  Set the language used by the \textit{webpages} constructed by \WebQuiz.
  The currently supported languages are:

    \begin{quote}
        % -----------------------------------------------------------------------
%   webquiz.cfg | webquiz TeX4ht configuration file
% -----------------------------------------------------------------------
%
%   Copyright (C) Andrew Mathas and Don Taylor, University of Sydney
%
%   Distributed under the terms of the GNU General Public License (GPL)
%               http://www.gnu.org/licenses/
%
%   This file is part of the WebQuiz system.
%
%   <Andrew.Mathas@sydney.edu.au>
% ----------------------------------------------------------------------

\NeedsTeXFormat{LaTeX2e}[1996/12/01]
\ProvidesFile{webquiz.cfg}[\webquiz{release date} v\webquiz{version} WebQuiz tex4ht configuration]

\makeatletter% quite surprisingly, this is needed!

% shortcut for inserting newlines into xml ifle
\newcommand\HNewLine{\HCode{\Hnewline}}

% Generate HTML5 + MathML code
\Preamble{xhtml,mathml,html5,NoFonts,charset="utf-8",ext=xml,svg}

% Don't output xml version tag
\Configure{VERSION}{}

% Output a WebQuiz doctype instead of the default for HTML5
\Configure{DOCTYPE}{\HCode{<?xml version="1.0" encoding="UTF-8" ?>\Hnewline
<!DOCTYPE webquiz>}}

\ifWQ@Debugging\def\WQ@debugging{true}\else\def\WQ@debugging{false}\fi

% store the quiz specifications as attributes to <quiz>
\Configure{HTML}{%
  \HNewLine%
  \Tg<webquiz debugging="\WQ@debugging"
                 hide_side_menu="\WQ@ClassOption{hidesidemenu value}"
                 language="\WQ@ClassOption{language}"
                 one_page="\WQ@ClassOption{onepage value}"
                 pst2pdf="\WQ@ClassOption{pst2pdf value}"
                 random_order="\WQ@ClassOption{randomorder value}"
                 src="\jobname.tex"
                 time_limit="\WQ@ClassOption{timelimit}"
                 theme="\WQ@ClassOption{theme}">
  \HNewLine%
  \Tg<title>\@title\Tg</title>%
  \HNewLine%
  \Tg<breadcrumb breadcrumbs="\WQ@breadcrumbs">\WQ@breadcrumb\Tg</breadcrumb>%
  \HNewLine%
  \Tg<unit_name url="\uos@url" quizzes_url="\WQ@quizzesUrl">\uos@name\Tg</unit_name>%
  \HNewLine%
  \Tg<unit_code>\uos@code\Tg</unit_code>%
  \HNewLine%
  \Tg<department url="\WQ@departmentURL">\WQ@department\Tg</department>%
  \HNewLine%
  \Tg<institution url="\WQ@institutionURL">\WQ@institution\Tg</institution>%
  \HNewLine%
}{\Tg</webquiz>}

% reset all configurations
\Configure{HEAD}{}{}
\Configure{BODY}{}{}
\Configure{TITLE}{}{}
\Configure{TITLE+}{}
\Configure{thanks author date and}{}{}{}{}{}{}{}{}
\renewcommand{\maketitle}{}

% convert pictures to svg
\Configure{Picture}{.svg}
%\Configure{Picture*}{.svg}

%\Configure{textit}{\HCode{<span class="textit">}\NoFonts}{\EndNoFonts\HCode{</span>}}
%\Configure{textsf}{\HCode{<span class="textsf">}\NoFonts}{\EndNoFonts\HCode{</span>}} 

% The webquiz macros and environments need to write opening and closing tags to
% the xml file. Opening these tags is easy. To close them we use \WQ@closeXXX
% and \WQ@@closeXXX macros. Each time a tag is opened we \let the \WQ@closeTag
% macro to the corresponding \WQ@@close macro. The \WQ@@close macro closes the
% tag and then \let's the \WQ@closeTag macro to \relax. This allows us to use
% the \WQ@close macros at the start of the tag routines below, which results in
% fairly transparent and readable code.

% make the @close tags \relax initially
\let\WQ@closeTag\relax % closes \correct, \incorrect, \feedback, \whenRight and \whenWrong
\let\WQ@closeText\relax   % closes text

% text tags - everything in wrapped inside <text>...</text> tags
\def\WQ@openText{\Tg<text>\HtmlParOff\HCode{<![CDATA[}\let\WQ@closeText\WQ@@closeText}
\def\WQ@@closeText{\HCode{]]>}\HtmlParOn\Tg</text>\HNewLine\global\let\WQ@closeText\relax}

% -----------------------------------------------------------------------
% discussion environment
% -----------------------------------------------------------------------

% discussion environments are number -1, -2, ... in the webquiz.js
\renewcommand\thediscussion{-\arabic{discussion}}

% inside a discussion environment any label defines two labels, the second of 
% which is a label to the short-title for the discussion item. Of there are many
% labels this is slightly in efficient but, because of tex4ht magic, it proved 
% too painful to extract the discussion number from the label.
\let\WQ@reallabel=\label
\newcommand\WQ@discussionLabel[1]{%
  \WQ@reallabel{#1}%
  \bgroup\let\@currentlabel\WQ@shortDiscussionTitle%
  \WQ@reallabel{discussion-title-#1}%
  \egroup%
}

% discussion environments
\RenewDocumentEnvironment{discussion}{O{Discussion}O{#1}}{%
    \WQ@DiscussionStart%
    \HNewLine%
    \Tg<discussion>\HNewLine
    \Tg<short_heading>#1\Tg</short_heading>\HNewLine
    \Tg<heading>#2\Tg</heading>\HNewLine
    \Tg<text>\HCode{<![CDATA[}% don't use \WQ@openText as we want paragraphs
    \def\WQ@shortDiscussionTitle{#1}%
    \let\label\WQ@discussionLabel%
}{\EndP\HCode{]]>}\Tg</text>\HNewLine\Tg</discussion>\HNewLine\WQ@DiscussionEnd}

% ---------------------------------------------------------------------------
% Cross referencing question and discussion environments
% ---------------------------------------------------------------------------

% tex4ht redefines the \ref internals so that r@<label> uses \rEfLiNK,
% so we temporarily redefine \rEfLiNK to access the question/discussion
% number
\def\WQ@rEfLiNK#1#2{#2}
% \WQ@Ref[optional *][optional text]{mandatory label}
\RenewDocumentCommand\WQ@ref{ m s o m }{%
  \HNewLine%
  \bgroup%
  \let\rEfLiNK\WQ@rEfLiNK%
  \IfBooleanTF{#2}{% link
    \HCode{<a onClick="gotoQuestion(}\ref{#4}\HCode{);">}%
    \IfNoValueTF{#3}{\ref{#1#4}}{#3}%
    \HCode{</a>}%
  }{% button
    \HCode{<span class="button blank" onClick="gotoQuestion(}\ref{#4}%
    \HCode{);">}%
    \IfNoValueTF{#3}{\ref{#1#4}}{#3}%
    \HCode{</span>}%
  }%
  \egroup%
}

\renewcommand\dref{\WQ@ref{discussion-title-}}
\renewcommand\qref{\WQ@ref{}}
\RenewDocumentCommand\Qref{ s O{#3} m }{%
  \HNewLine%
  \IfBooleanTF{#1}{% link
    \HCode{<a onClick="gotoQuestion(#3);">}#2\HCode{</a>}%
  }{% button
    \HCode{<span class="button blank" onClick="gotoQuestion(#3);">}#2\HCode{</span>}%
  }%
}

% -----------------------------------------------------------------------
% question environments
% -----------------------------------------------------------------------

\renewenvironment{question}{%
  \WQ@QuestionStart%
  \IgnorePar\HCode{\Hnewline<question>}% open the question tag
  \WQ@openText% start some text
}{\WQ@QuestionEnd\WQ@closeText\WQ@closeTag\Tg</question>\HNewLine}

% -----------------------------------------------------------------------
% choice environment, \correct and \incorrect choices and \feedback
% -----------------------------------------------------------------------

\RenewDocumentEnvironment{choice}{O{}}{%
  \WQ@ChoiceStart{#1}%
  \WQ@closeText% close any open text tags and then insert tex4ht code
  \IgnorePar%
  \Tg<choice type="\pgfkeysvalueof{/webquiz checker/mode}"
             columns="\pgfkeysvalueof{/webquiz checker/columns value}">%
  \let\WQ@closeTag\relax%
}{\WQ@ChoiceEnd\WQ@closeTag\Tg</choice>}

% correct and incorrect items
\def\WQ@@closeItem{\WQ@closeText\Tg</item>\HNewLine\let\WQ@closeTag\relax}
\def\WQ@Item#1{%
  \WQ@ItemStart%
  \WQ@closeTag%
  \Tg<item correct="#1" symbol="\thechoice">\WQ@openText%
  \let\WQ@closeTag\WQ@@closeItem%
}
\def\correct{\WQ@Item{true}}
\def\incorrect{\WQ@Item{false}}

% feedback
\def\WQ@@closeFeedback{\WQ@closeText\Tg</feedback>\HNewLine\let\WQ@closeTag\relax}
\def\feedback{%
  \WQ@FeedbackStart%
  \WQ@closeTag\Tg<feedback>\WQ@openText\let\WQ@closeTag\WQ@@closeFeedback%
}

% -----------------------------------------------------------------------
% The \answer macro and \whenRight, \whenWrong
% -----------------------------------------------------------------------

\RenewDocumentCommand\answer {sO{string}m}{%
  \WQ@AnswerStart{#2}%
  \WQ@closeText%
  \IfBooleanTF{#1}{\def\WQ@prompt{false}}{\def\WQ@prompt{true}}%
  \Tg<answer prompt="\WQ@prompt" comparison="#2">\HNewLine%
  \WQ@openText#3\WQ@closeText%
  \Tg</answer>%
  \HNewLine%
  \let\WQ@closeTag\relax%
  \WQ@openText%
}

% whenRight and whenWrong
\def\WQ@closeWhen{\Tg</when>\HNewLine\let\WQ@closeTag\relax}
\def\WQ@When#1{%
  \WQ@WhenStart{#1}%
  \WQ@closeText\WQ@closeTag%
  \IgnorePar\Tg<when type="#1">\WQ@openText%
  \let\WQ@closeTag\WQ@closeWhen%
}
\def\whenRight{\WQ@When{right}}
\def\whenWrong{\WQ@When{wrong}}

% -----------------------------------------------------------------------
% the quizindex environment and \quiz macro
% -----------------------------------------------------------------------

% quizindex environment and \quiz
\ConfigureEnv{quizindex}{\Tg<quizindex>\HNewLine}{\Tg</quizindex>\HNewLine}{}{}
\RenewDocumentCommand\quiz{som}{\refstepcounter{quiz}%
  \WQ@QuizStart%
  \IfBooleanTF{#1}{\def\WQ@prompt{false}}{\def\WQ@prompt{true}}%
  \IfNoValueTF{#2}{\def\WQ@url{quiz\thequiz.html}}{\def\WQ@url{#2}}%
  \Tg<index_item prompt="\WQ@prompt" url="\WQ@url">\WQ@openText #3 \WQ@closeText\Tg</index_item>\HNewLine%
}

% -----------------------------------------------------------------------
% making listing environment work with colour I
% - this code appears to break \color and \textcolor in the online manual
% -----------------------------------------------------------------------


% from https://tex.stackexchange.com/questions/225554/syntax-highlighting-in-an-html-presentation
% extract current color as hexadecimal value
\makeatletter
\ifdefined\lst@version
\newcommand\tsf@getColor[1][.]{%
      % colorname `.` holds current color
      \extractcolorspec{.}{\tsf@color}
      \expandafter\convertcolorspec\tsf@color{HTML}\tsf@color
      %\tmpcolor
  }

  % write css color for given css selector
  \newcommand\CssColor[1]{%on-l
      % save current color
      \tsf@getColor%
      \Css{#1{color:\#\tsf@color;}}%
  }
\fi

% Do not set indent and noindent classes on paragraphs
\Configure{HtmlPar}
    {\EndP\Tg<p>}
    {\EndP\Tg<p>}
    {\HCode{</p>\Hnewline}}
    {\HCode{</p>\Hnewline}}

\renewcommand\DisplayAsImage[2][]{%
  \csletcs{real:#2}{#2}%
  \NewConfigure{#2}{2}
  \csdef{#2}##1{\Picture+[#1]{}\csuse{real:#2}{##1}\EndPicture}
  \Configure{#2}{\Picture+[#1]{}}{\EndPicture}
}

\begin{document}

% disable \title after \begin{document}
\def\title{\WQ@Error{\@backslashchar title can only be used in the preamble}\@ehc}

% -----------------------------------------------------------------------
% making listing environment work with colour II
% - this code appears to break \color and \textcolor in the online manual
% -----------------------------------------------------------------------

%% this code is for processing the listing environments in the online manual
\ifdefined\lst@version
  % from https://www.mail-archive.com/tex4ht@tug.org/msg00116.html
  \let\savecolor\color
  \NewConfigure{color}[2]{\def\a@color{#1}\def\b@color{#2}}
  \def\color@@tmp#1{\a@color#1\b@color\savecolor{#1}\aftergroup\endspan}
  \let\color\color@@tmp
  \def\endspan{\Tg</span>}
  \Configure{color}{\HCode{<span style="color:}}{\HCode{">}}

  \ConfigureEnv{lstlisting}
     {\let\color\savecolor
      \ifvmode \IgnorePar\fi \EndP
      \gHAdvance\listingN by 1
      \HCode{<!--l. \the\inputlineno-->}%
      \gdef\start:LstLn{\HCode{<div class="lstlisting" id="listing-\listingN">}%
        \gdef\start:LstLn{\leavevmode\special{t4ht@+\string&{35}x00A0{59}}x%
      \HCode{<br/>}}}
      \bgroup
         \Configure{listings}
           {{\everypar{}\leavevmode}}
           {{\everypar{}\leavevmode}}
           {\start:LstLn \HCode{<span class="label">}}
           {\HCode{</span>}}%
     }
     {\egroup
      \ifvmode \IgnorePar\fi \EndP \HCode{</div>}\par}
     {}{}

  % for listings in webquiz-manual
  % from https://tex.stackexchange.com/questions/225554/syntax-highlighting-in-an-html-presentation
  \newcommand\LstCss[2]{%
      \bgroup%
          \csname lst@#2\endcsname%
          \CssColor{#1}%
      \egroup%
  }

  \LstCss{div.lstlisting .ecbx-1000}{keywordstyle}
  \LstCss{div.lstlisting .ecss-1000}{commentstyle}
  \LstCss{div.lstlisting .ectt-1000}{basicstyle}
\fi

\makeatother

\EndPreamble

% bug fix for \mathbf from http://tex.stackexchange.com/questions/362178
%\Configure{mathbf}{\HCode{<mi mathvariant="bold">}\PauseMathClass}{\EndPauseMathClass\HCode{</mi>}}

\endinput
%%
%% End of file `webquiz.cfg'.

    \end{quote}

    The languages files are used to print the various buttons and text
    that is generated on the web pages constructed by \WebQuiz. The
    \LatexCode|language| option does not affect the DVI or PDF versions of
    the quiz and it does not load language packages like \ctan{babel} or
    \ctan{polyglossia}.

    Typical usage of the \LatexCode|language| option is the following:
    \lstinputlisting[style=latexcode,extendedchars=false]{examples/french}
    This produces a web page like this:
    \ScreenShot{A web page with \textsf{language=french}}{examples/french}
    As a general rule, \LaTeX{} and \TeXfht do not cope well with
    unicode characters, so if your quiz contains (a lot of) unicode
    characters then we recommend using Xe\LaTeX{} or Lau\LaTeX, which
    corresponds to using \WebQuiz with the command-line options
    \BashCode|webquiz -x| or \BashCode|webquiz -l|, respectively. The
    default \TeX{} engine can be set in the \webquizrc file; see
    \autoref{SS:rcfile} for more details.
    \index{unicode}\index{lualatex}\index{xelatex}

    The language files were created largely using google translate so
    they may well need fine-tuning\footnote{For example, the word
    ``Copyright'' in lefthand the side-margin is not translated but
    perhaps it should be.}. You can use \BashCode|kpsewhich| to look at
    the language files, which all have names of the form
    \BashCode|webquiz-language.lang|.  For example, the
    \textit{English} language file, which is the default, can be found
    using the command:
    \begin{bashcode}
         > kpsewhich webquiz-english.lang
    \end{bashcode}
    This file contains the following:
    \InputBashCode{webquiz-english.lang}
    In these files, the material to the left of the equals signs are
    effectively variables, and so they should never be changed, or
    deleted, whereas anything the right of the equals signs is the text
    that is written to the \WebQuiz web pages. To add \WebQuiz support
    for a new language, say \LatexCode|XXX|, copy any \WebQuiz language
    file and translate the material to the right of the equals signs and
    then save the file as \BashCode|webquiz-XXX.lang|, and then put this
    file anywhere in the \LaTeX{} search path. Note that the expressions
    \LatexCode|{}| in the language files should be not be removed
    because they expand to expressions like \LatexCode|(a)|, \LatexCode|(b)|
    etc. when used in the on-line quizzes. Once the new language file
    \BashCode|webquiz-XXX.lang| is in the \LaTeX{} search path it can be
    used by \WebQuiz using the \LatexCode|language| document-class
    option:
    \begin{latexcode}
       \documentclass[language=XXX]{webquiz}
       \begin{document}
            insert quiz code
       \end{document}
    \end{latexcode}
    Please submit any new language files, or corrections to existing
    language files, as a \textit{new issue} at:
      \href{https://\webquiz{repository}}{\webquiz{repository}}.

  \item[theme]\CrossIndex{document class options}{theme}

     \WebQuiz\ has a small number of different themes for setting the
     colours on the quiz web pages. The theme can be set as an option to
     the document class or in the \webquizrc file. Most of the examples
     so far have used the \LatexCode|default| theme. As
     an example, if we use:
    \InputLatexCode{examples/choice-dark}
    then the quiz page looks like:
    \ScreenShot{Multiple choice with light theme}{examples/choice-dark}
    This should be compared with the corresponding screenshot in section
    \autoref{choice-multiple}, which uses the default theme.

    The list of currently supported themes is:
      \begin{quote}
          % -----------------------------------------------------------------------
%   webquiz.cfg | webquiz TeX4ht configuration file
% -----------------------------------------------------------------------
%
%   Copyright (C) Andrew Mathas and Don Taylor, University of Sydney
%
%   Distributed under the terms of the GNU General Public License (GPL)
%               http://www.gnu.org/licenses/
%
%   This file is part of the WebQuiz system.
%
%   <Andrew.Mathas@sydney.edu.au>
% ----------------------------------------------------------------------

\NeedsTeXFormat{LaTeX2e}[1996/12/01]
\ProvidesFile{webquiz.cfg}[\webquiz{release date} v\webquiz{version} WebQuiz tex4ht configuration]

\makeatletter% quite surprisingly, this is needed!

% shortcut for inserting newlines into xml ifle
\newcommand\HNewLine{\HCode{\Hnewline}}

% Generate HTML5 + MathML code
\Preamble{xhtml,mathml,html5,NoFonts,charset="utf-8",ext=xml,svg}

% Don't output xml version tag
\Configure{VERSION}{}

% Output a WebQuiz doctype instead of the default for HTML5
\Configure{DOCTYPE}{\HCode{<?xml version="1.0" encoding="UTF-8" ?>\Hnewline
<!DOCTYPE webquiz>}}

\ifWQ@Debugging\def\WQ@debugging{true}\else\def\WQ@debugging{false}\fi

% store the quiz specifications as attributes to <quiz>
\Configure{HTML}{%
  \HNewLine%
  \Tg<webquiz debugging="\WQ@debugging"
                 hide_side_menu="\WQ@ClassOption{hidesidemenu value}"
                 language="\WQ@ClassOption{language}"
                 one_page="\WQ@ClassOption{onepage value}"
                 pst2pdf="\WQ@ClassOption{pst2pdf value}"
                 random_order="\WQ@ClassOption{randomorder value}"
                 src="\jobname.tex"
                 time_limit="\WQ@ClassOption{timelimit}"
                 theme="\WQ@ClassOption{theme}">
  \HNewLine%
  \Tg<title>\@title\Tg</title>%
  \HNewLine%
  \Tg<breadcrumb breadcrumbs="\WQ@breadcrumbs">\WQ@breadcrumb\Tg</breadcrumb>%
  \HNewLine%
  \Tg<unit_name url="\uos@url" quizzes_url="\WQ@quizzesUrl">\uos@name\Tg</unit_name>%
  \HNewLine%
  \Tg<unit_code>\uos@code\Tg</unit_code>%
  \HNewLine%
  \Tg<department url="\WQ@departmentURL">\WQ@department\Tg</department>%
  \HNewLine%
  \Tg<institution url="\WQ@institutionURL">\WQ@institution\Tg</institution>%
  \HNewLine%
}{\Tg</webquiz>}

% reset all configurations
\Configure{HEAD}{}{}
\Configure{BODY}{}{}
\Configure{TITLE}{}{}
\Configure{TITLE+}{}
\Configure{thanks author date and}{}{}{}{}{}{}{}{}
\renewcommand{\maketitle}{}

% convert pictures to svg
\Configure{Picture}{.svg}
%\Configure{Picture*}{.svg}

%\Configure{textit}{\HCode{<span class="textit">}\NoFonts}{\EndNoFonts\HCode{</span>}}
%\Configure{textsf}{\HCode{<span class="textsf">}\NoFonts}{\EndNoFonts\HCode{</span>}} 

% The webquiz macros and environments need to write opening and closing tags to
% the xml file. Opening these tags is easy. To close them we use \WQ@closeXXX
% and \WQ@@closeXXX macros. Each time a tag is opened we \let the \WQ@closeTag
% macro to the corresponding \WQ@@close macro. The \WQ@@close macro closes the
% tag and then \let's the \WQ@closeTag macro to \relax. This allows us to use
% the \WQ@close macros at the start of the tag routines below, which results in
% fairly transparent and readable code.

% make the @close tags \relax initially
\let\WQ@closeTag\relax % closes \correct, \incorrect, \feedback, \whenRight and \whenWrong
\let\WQ@closeText\relax   % closes text

% text tags - everything in wrapped inside <text>...</text> tags
\def\WQ@openText{\Tg<text>\HtmlParOff\HCode{<![CDATA[}\let\WQ@closeText\WQ@@closeText}
\def\WQ@@closeText{\HCode{]]>}\HtmlParOn\Tg</text>\HNewLine\global\let\WQ@closeText\relax}

% -----------------------------------------------------------------------
% discussion environment
% -----------------------------------------------------------------------

% discussion environments are number -1, -2, ... in the webquiz.js
\renewcommand\thediscussion{-\arabic{discussion}}

% inside a discussion environment any label defines two labels, the second of 
% which is a label to the short-title for the discussion item. Of there are many
% labels this is slightly in efficient but, because of tex4ht magic, it proved 
% too painful to extract the discussion number from the label.
\let\WQ@reallabel=\label
\newcommand\WQ@discussionLabel[1]{%
  \WQ@reallabel{#1}%
  \bgroup\let\@currentlabel\WQ@shortDiscussionTitle%
  \WQ@reallabel{discussion-title-#1}%
  \egroup%
}

% discussion environments
\RenewDocumentEnvironment{discussion}{O{Discussion}O{#1}}{%
    \WQ@DiscussionStart%
    \HNewLine%
    \Tg<discussion>\HNewLine
    \Tg<short_heading>#1\Tg</short_heading>\HNewLine
    \Tg<heading>#2\Tg</heading>\HNewLine
    \Tg<text>\HCode{<![CDATA[}% don't use \WQ@openText as we want paragraphs
    \def\WQ@shortDiscussionTitle{#1}%
    \let\label\WQ@discussionLabel%
}{\EndP\HCode{]]>}\Tg</text>\HNewLine\Tg</discussion>\HNewLine\WQ@DiscussionEnd}

% ---------------------------------------------------------------------------
% Cross referencing question and discussion environments
% ---------------------------------------------------------------------------

% tex4ht redefines the \ref internals so that r@<label> uses \rEfLiNK,
% so we temporarily redefine \rEfLiNK to access the question/discussion
% number
\def\WQ@rEfLiNK#1#2{#2}
% \WQ@Ref[optional *][optional text]{mandatory label}
\RenewDocumentCommand\WQ@ref{ m s o m }{%
  \HNewLine%
  \bgroup%
  \let\rEfLiNK\WQ@rEfLiNK%
  \IfBooleanTF{#2}{% link
    \HCode{<a onClick="gotoQuestion(}\ref{#4}\HCode{);">}%
    \IfNoValueTF{#3}{\ref{#1#4}}{#3}%
    \HCode{</a>}%
  }{% button
    \HCode{<span class="button blank" onClick="gotoQuestion(}\ref{#4}%
    \HCode{);">}%
    \IfNoValueTF{#3}{\ref{#1#4}}{#3}%
    \HCode{</span>}%
  }%
  \egroup%
}

\renewcommand\dref{\WQ@ref{discussion-title-}}
\renewcommand\qref{\WQ@ref{}}
\RenewDocumentCommand\Qref{ s O{#3} m }{%
  \HNewLine%
  \IfBooleanTF{#1}{% link
    \HCode{<a onClick="gotoQuestion(#3);">}#2\HCode{</a>}%
  }{% button
    \HCode{<span class="button blank" onClick="gotoQuestion(#3);">}#2\HCode{</span>}%
  }%
}

% -----------------------------------------------------------------------
% question environments
% -----------------------------------------------------------------------

\renewenvironment{question}{%
  \WQ@QuestionStart%
  \IgnorePar\HCode{\Hnewline<question>}% open the question tag
  \WQ@openText% start some text
}{\WQ@QuestionEnd\WQ@closeText\WQ@closeTag\Tg</question>\HNewLine}

% -----------------------------------------------------------------------
% choice environment, \correct and \incorrect choices and \feedback
% -----------------------------------------------------------------------

\RenewDocumentEnvironment{choice}{O{}}{%
  \WQ@ChoiceStart{#1}%
  \WQ@closeText% close any open text tags and then insert tex4ht code
  \IgnorePar%
  \Tg<choice type="\pgfkeysvalueof{/webquiz checker/mode}"
             columns="\pgfkeysvalueof{/webquiz checker/columns value}">%
  \let\WQ@closeTag\relax%
}{\WQ@ChoiceEnd\WQ@closeTag\Tg</choice>}

% correct and incorrect items
\def\WQ@@closeItem{\WQ@closeText\Tg</item>\HNewLine\let\WQ@closeTag\relax}
\def\WQ@Item#1{%
  \WQ@ItemStart%
  \WQ@closeTag%
  \Tg<item correct="#1" symbol="\thechoice">\WQ@openText%
  \let\WQ@closeTag\WQ@@closeItem%
}
\def\correct{\WQ@Item{true}}
\def\incorrect{\WQ@Item{false}}

% feedback
\def\WQ@@closeFeedback{\WQ@closeText\Tg</feedback>\HNewLine\let\WQ@closeTag\relax}
\def\feedback{%
  \WQ@FeedbackStart%
  \WQ@closeTag\Tg<feedback>\WQ@openText\let\WQ@closeTag\WQ@@closeFeedback%
}

% -----------------------------------------------------------------------
% The \answer macro and \whenRight, \whenWrong
% -----------------------------------------------------------------------

\RenewDocumentCommand\answer {sO{string}m}{%
  \WQ@AnswerStart{#2}%
  \WQ@closeText%
  \IfBooleanTF{#1}{\def\WQ@prompt{false}}{\def\WQ@prompt{true}}%
  \Tg<answer prompt="\WQ@prompt" comparison="#2">\HNewLine%
  \WQ@openText#3\WQ@closeText%
  \Tg</answer>%
  \HNewLine%
  \let\WQ@closeTag\relax%
  \WQ@openText%
}

% whenRight and whenWrong
\def\WQ@closeWhen{\Tg</when>\HNewLine\let\WQ@closeTag\relax}
\def\WQ@When#1{%
  \WQ@WhenStart{#1}%
  \WQ@closeText\WQ@closeTag%
  \IgnorePar\Tg<when type="#1">\WQ@openText%
  \let\WQ@closeTag\WQ@closeWhen%
}
\def\whenRight{\WQ@When{right}}
\def\whenWrong{\WQ@When{wrong}}

% -----------------------------------------------------------------------
% the quizindex environment and \quiz macro
% -----------------------------------------------------------------------

% quizindex environment and \quiz
\ConfigureEnv{quizindex}{\Tg<quizindex>\HNewLine}{\Tg</quizindex>\HNewLine}{}{}
\RenewDocumentCommand\quiz{som}{\refstepcounter{quiz}%
  \WQ@QuizStart%
  \IfBooleanTF{#1}{\def\WQ@prompt{false}}{\def\WQ@prompt{true}}%
  \IfNoValueTF{#2}{\def\WQ@url{quiz\thequiz.html}}{\def\WQ@url{#2}}%
  \Tg<index_item prompt="\WQ@prompt" url="\WQ@url">\WQ@openText #3 \WQ@closeText\Tg</index_item>\HNewLine%
}

% -----------------------------------------------------------------------
% making listing environment work with colour I
% - this code appears to break \color and \textcolor in the online manual
% -----------------------------------------------------------------------


% from https://tex.stackexchange.com/questions/225554/syntax-highlighting-in-an-html-presentation
% extract current color as hexadecimal value
\makeatletter
\ifdefined\lst@version
\newcommand\tsf@getColor[1][.]{%
      % colorname `.` holds current color
      \extractcolorspec{.}{\tsf@color}
      \expandafter\convertcolorspec\tsf@color{HTML}\tsf@color
      %\tmpcolor
  }

  % write css color for given css selector
  \newcommand\CssColor[1]{%on-l
      % save current color
      \tsf@getColor%
      \Css{#1{color:\#\tsf@color;}}%
  }
\fi

% Do not set indent and noindent classes on paragraphs
\Configure{HtmlPar}
    {\EndP\Tg<p>}
    {\EndP\Tg<p>}
    {\HCode{</p>\Hnewline}}
    {\HCode{</p>\Hnewline}}

\renewcommand\DisplayAsImage[2][]{%
  \csletcs{real:#2}{#2}%
  \NewConfigure{#2}{2}
  \csdef{#2}##1{\Picture+[#1]{}\csuse{real:#2}{##1}\EndPicture}
  \Configure{#2}{\Picture+[#1]{}}{\EndPicture}
}

\begin{document}

% disable \title after \begin{document}
\def\title{\WQ@Error{\@backslashchar title can only be used in the preamble}\@ehc}

% -----------------------------------------------------------------------
% making listing environment work with colour II
% - this code appears to break \color and \textcolor in the online manual
% -----------------------------------------------------------------------

%% this code is for processing the listing environments in the online manual
\ifdefined\lst@version
  % from https://www.mail-archive.com/tex4ht@tug.org/msg00116.html
  \let\savecolor\color
  \NewConfigure{color}[2]{\def\a@color{#1}\def\b@color{#2}}
  \def\color@@tmp#1{\a@color#1\b@color\savecolor{#1}\aftergroup\endspan}
  \let\color\color@@tmp
  \def\endspan{\Tg</span>}
  \Configure{color}{\HCode{<span style="color:}}{\HCode{">}}

  \ConfigureEnv{lstlisting}
     {\let\color\savecolor
      \ifvmode \IgnorePar\fi \EndP
      \gHAdvance\listingN by 1
      \HCode{<!--l. \the\inputlineno-->}%
      \gdef\start:LstLn{\HCode{<div class="lstlisting" id="listing-\listingN">}%
        \gdef\start:LstLn{\leavevmode\special{t4ht@+\string&{35}x00A0{59}}x%
      \HCode{<br/>}}}
      \bgroup
         \Configure{listings}
           {{\everypar{}\leavevmode}}
           {{\everypar{}\leavevmode}}
           {\start:LstLn \HCode{<span class="label">}}
           {\HCode{</span>}}%
     }
     {\egroup
      \ifvmode \IgnorePar\fi \EndP \HCode{</div>}\par}
     {}{}

  % for listings in webquiz-manual
  % from https://tex.stackexchange.com/questions/225554/syntax-highlighting-in-an-html-presentation
  \newcommand\LstCss[2]{%
      \bgroup%
          \csname lst@#2\endcsname%
          \CssColor{#1}%
      \egroup%
  }

  \LstCss{div.lstlisting .ecbx-1000}{keywordstyle}
  \LstCss{div.lstlisting .ecss-1000}{commentstyle}
  \LstCss{div.lstlisting .ectt-1000}{basicstyle}
\fi

\makeatother

\EndPreamble

% bug fix for \mathbf from http://tex.stackexchange.com/questions/362178
%\Configure{mathbf}{\HCode{<mi mathvariant="bold">}\PauseMathClass}{\EndPauseMathClass\HCode{</mi>}}

\endinput
%%
%% End of file `webquiz.cfg'.

      \end{quote}
    Screenshots for the complete list of themes can be found in
    \hyperref[SS:themes]{Appendix A}.

\end{description}


\subsection{Including graphics and using pstricks and tikz}\label{SS:graphics}

By far the easiest way to include images when using \WebQuiz is by
adding
\begin{latexcode}
    \usepackage[dvipdfmx]{graphicx}\DeclareGraphicsExtensions{.png}
\end{latexcode}
to your document preamble. You need to use
\LatexCode|\DeclareGraphicsExtensions| to tell \WebQuiz what types of
images you are using, so the code above works for \LatexCode{png}
images. More generally, you can use a comma separated list of
extensions. such as:
\begin{latexcode}
  \DeclareGraphicsExtensions{.png, .jpg, .gif}
\end{latexcode}
The option \LatexCode|dvipdfmx|
to \LatexCode|graphicx| is only necessary if you want to be able to
rescale images. For example, the code:

    \InputLatexCode{examples/lion-image}

produces the web page:

\ScreenShot{Including images}{examples/lion-image}

It is also possible to draw quite complicated diagrams in
\LaTeX\ documents using packages like \ctan{tikz} and \ctan{pstricks}.
These packages can be used in \WebQuiz quizzes although,
unfortunately, both of these packages have issues when used with \TeX
4ht. The good news is that \WebQuiz attempts to solve these problems for
you if you use the appropriate document class options.

As there have been several recent updates it is advisable to install the
latest version of both of these packages, as well as the packages
\ctan{make4ht}, \ctan{pgf} and \TeXfht. In fact, I recommend updating
all of your packages.

\begin{description}
    \item[pst2pdf] \index{pstricks}\CrossIndex{document class options}{pst2pdf}

    For the most part \ctan{pstricks} drawings will display correctly
    and when they fail they can often be salvaged by using
    \ctan{pst2pdf}. Try the \LatexCode|pst2pdf| document-class option if you are
    having trouble displaying an image created using \ctan{pstricks}. It
    is not guaranteed to work but it often fixes the problem.

    For example, the following quiz compiles only with the
    \LatexCode|pst2pdf| document-class option:
    \InputLatexCode{examples/pst2pdf}
    It produces the quiz:
    \ScreenShot{Example requiring the pst2pdf document class option}
               {examples/pst2pdf}

    \noindent
    \dbend{According to the \ctan{pst2pdf} manual:}

    \begin{quote}
      \LatexCode|pst2pdf| needs Ghostscript (>=9.14), perl (>=5.18), pdf2svg, pdftoppm
      and pdftops (from poppler-utils or xpdf-utils) to process a file
      using \ctan{pst2pdf}.
    \end{quote}

    Unfortunately, \LatexCode|pst2pdf| can fail silently without giving any
    warnings. If using \LatexCode|pst2pdf| does not produce an image then
    the problem might be that you have not installed all of the programs
    that \LatexCode|pst2pdf| relies upon, so look in your log files for
    error messages and check the list of programs given above. See
    \autoref{SS:pstricks} for more details.

  \item[tikz]\CrossIndex{document class options}{tikz}
    Giving this class option both loads the \ctan[pgf]{tikz} package
    (use the document-class option \LatexCode|tikz|, and not
    \LatexCode|\usepackage{tikz}|, in your \LaTeX\ file). As a bonus,
    this \ctan{tikz} document option fixes several issues with PGF that
    prevent it from working with \TeX 4ht. Thanks are due to Michal
    Hoftich for supplying both fixes!\footnote{The first issue is a bug
    that has been recently been fixed by the PGF developers but the fix
    may not be installed on all systems.  The second issue is that the
    PGF files hard code \BashCode|ISO-8859-1| encoding, which is a problem
    if you use UTF-8; see
    \href{https://tex.stackexchange.com/questions/390421}{Make4ht, tikz
    and UTF 8 encoding question}.  }

    \InputLatexCode{examples/tikz-ex}

    This produces:

    \ScreenShot{Tikz example}{examples/tikz-ex}
\end{description}

Most people use either \ctan{pstricks} or \ctan{tik}. A quiz that tries
to use both \ctan{pstrick} and  \ctan{tik} will probably compile.

\subsection{Configuring commands and environments for
\TeXfht}\label{SS:config}\macroIndex{DisplayAsImage}

  \WebQuiz uses \TeXfht{} to convert the quiz content from \LaTeX{} to
  HTML. If \TeXfht{} has not been configured to for some of the commands
  or environments that you are using then they may not display correctly
  in your on-line quizzes. The ``correct'' way to fix such problems is
  to write appropriate \TeXfht{} configuration commands, however, this
  can be tricky to do especially if you are not used to the inner
  workings of \TeX{} and \TeXfht.

  \WebQuiz provides the command \LatexCode|\DisplayAsImage|, which essentially
  tells \TeXfht to treat your command as an image when it creates the
  web page. This is an easy work-around that is much easier than writing
  your own \TeXfht configuration commands and it often produces good
  results. For example, the \ctan{mhchem} package is a powerful package
  that defines a macro \LatexCode{\ce} for writing chemical symbols but,
  unfortunately, the \LatexCode{\ce} macro has not (yet) been configured
  to work with \TeXfht, which means that this command does not work in
  \WebQuiz quizzes.  Putting the command \LatexCode{\DisplayAsImage{ce}}
  into your preamble fixes this and allows to you to produce quizzes
  using \LatexCode{\ce}. For example, you can produce the quiz
  \ScreenShot{Using DisplayAsImage}{examples/display-as-image}
  using the code:
  \InputLatexCode{examples/display-as-image}
  As the example code shows, \LatexCode|\DisplayAsImage| accepts an optional
  argument that can be used to fine-tune the placement of the image on
  the quiz web page. For those interested in the technical details, the
  definition of \LatexCode|\DisplayAsImage| is:
  \begin{latexcode}
    \RequirePackage{etoolbox}
    \renewcommand\DisplayAsImage[2][]{%
      \csletcs{real:#2}{#2}%
      \NewConfigure{#2}{2}
      \csdef{#2}##1{\Picture+[#1]{}\csuse{real:#2}{##1}\EndPicture}
      \Configure{#2}{\Picture+[#1]{}}{\EndPicture}
    }
  \end{latexcode}

\section{System requirements, installation and configuration}
  \label{S:configuration}
  \index{system requirements}
  \CrossIndex{system requirements}{tex4ht}
  \CrossIndex{system requirements}{make4ht}
  \CrossIndex{system requirements}{python}

    \WebQuiz takes a \LaTeX\ file and translates it into a functional
    web page. To use \WebQuiz the quiz author only needs to know how to
    use \LaTeX\ \textit{and} to have all of the programs used by \WebQuiz
    installed. Fortunately, most of the system requirements will already
    be installed on a system with an up-to-date installation of
    \TeXLive, however, some tweaking may still be necessary.

    Behind the scenes, \WebQuiz uses \TeXfht, python, javascript and
    several other tools to construct and operate the on-line quizzes.
    The \WebQuiz program has three main components:
      \begin{itemize}
           \item \LaTeX\ files (a class file and \TeXfht configuration files)
           \item Python3 executables that use \TeXfht to convert \LaTeX\
           files into web pages
           \item Web files (javascript, css and on-line documentation)
          %\item Documentation
      \end{itemize}
    Of course, to use the on-line quizzes created by \WebQuiz you
    need a web server. To use \WebQuiz all of these files need to
    be in appropriate places.  Fortunately, \Ctan takes care of
    most of this but the web-related files still need to be put onto
    your web server.

    \WebQuiz has been tested extensively on Linux and Mac operating
    systems. Several people have used \WebQuiz on windows computers,
    but I have not tested the program on a windows computer myself.

    \subsection{System requirements}\index{system requirements}

    In order to work \WebQuiz needs the following programs to be
    installed on your system:
    \begin{itemize}
       \item An up-to-date \LaTeX{} distribution, such as that provided
       by \TeXLive. In particular, you need to have \TeXfht{} and
       \ctan{make4ht} installed.

       Unfortunately, the version of \ctan{make4ht} that was released
       with \TeXLive{} 2018 had some bugs and there have been a number of
       additional recent changes to \ctan{make4ht} and \ctan{pstricks},
       so it is strongly recommended that you update all packages from \Ctan
       before you try and use \WebQuiz.

        \item \href{https://www.python.org/}{Python 3}. As of writing
        python 3.7 is available.
        \item Javascript
        \item If your quizzes use \ctan{pstricks}, or if you want to
        compile the on-line manual, then you need to ensure that
        \href{https://www.ghostscript.com/}{Ghostscript} and
        \ctan{dvisvgm} are installed and properly configured; see
        \autoref{SS:pstricks} for more details.
        \item A web server. As detailed in \autoref{SS:Initialise}, you
        will need to install the web components of \WebQuiz.
    \end{itemize}

    \subsection{Initialising \WebQuiz}\label{SS:Initialise}
    \CrossIndex{command-line options}{initialise}
    \index{installation}\index{initialisation}

    \WebQuiz is a tool for creating on-line quizzes and in order for it
    to work efficiently various files
    (\href{https://en.wikipedia.org/wiki/JavaScript}{javascript} and
     \href{https://www.w3schools.com/css/css_intro.asp}{cascading style
     sheets}) need to put onto your web server. \WebQuiz has an
     \textit{initialisation} routine that helps to streamline
     the process. In fact, until \WebQuiz has been initialised it will
     ask you if you to run the initialisation routine each time you use
     \WebQuiz. You can reinitialise \WebQuiz at any time using the
     command-line option:
    \begin{bashcode}
      > webquiz --initialise
    \end{bashcode}
    \WebQuiz will actually work without being initialised, however, any
    quiz web pages that are created before initialisation will be
    emblazoned with a message reminding you to initialise \WebQuiz.

    \textit{If you want to install the \WebQuiz files for general use on
    your system, or to save system wide settings, then you need to run
    the initialisation command }\BashCode|webquiz --initialise|
    \textit{using an administrator account or using \BashCode|sudo| on a
    UNIX or Mac OSX system.}

    The location of the files on the web server depends both on the
    operating system that is running on your computer and how your web
    server has been configured. It is essential that the \WebQuiz files
    are installed in a directory that is accessible from the web. It
    does not matter if they are put into a user web directory or into a
    system web directory.  If necessary please consult your system
    administrator.

    Common locations for the \textit{web root} of the server are:
    \begin{center}
      \begin{tabular}{ll}
        \toprule
        Operating system & Root of web server \\\midrule
        Mac OSX     & \BashCode|/Library/WebServer/Documents|\\
        Linux       & \BashCode|/var/www/html|\\
        Windows     & \BashCode|C:\inetpub\wwwroot|\\
        \bottomrule
      \end{tabular}
    \end{center}
    \WebQuiz needs to copy several files into a subdirectory of this
    \textit{web root}. When you run
    \begin{bashcode}
      > webquiz --initialise
    \end{bashcode}
    you will be prompted for the following:
    \begin{itemize}
      \item The location of the \WebQuiz web directory,
        which needs to be a directory on your local file system that is
        visible from your web server
      \item The relative URL for this directory, which tells your web
      browser where to find these files
    \end{itemize}
    For example, on my system the base directory for our web server is
    \BashCode|/usr/local/httpd/| and the \WebQuiz web files are in
    \BashCode|/usr/local/httpd/UoS/WebQuiz|.  So, I set:
    \begin{quote}
      \begin{tabular}{lll} \WebQuiz web
            directory &=& \BashCode|/usr/local/httpd/UoS/WebQuiz|\\
            \WebQuiz relative URL  &=& \BashCode|/UoS/WebQuiz|
        \end{tabular}
    \end{quote}
    Once the initialisation step is complete, \WebQuiz is ready to use! You can test your \WebQuiz
    installation by compiling the example files from the \WebQuiz
    manual. You can find these files in the directory
    \BashCode|web_root/doc/examples|, where the \BashCode|web_root| is
    the directory where you just installed \WebQuiz web files. If
    \href{https://www.ghostscript.com/}{Ghostscript} and \ctan{dvisvgm}
    are installed and properly configured (see \autoref{SS:graphics})
    then you should also be able to compile the on-line manual using
    \WebQuiz.

    \begin{trivlist}
      \item[\textbf{Remark}] To remove all \WebQuiz files from your web server
       use \BashCode|webquiz --uninstall|
    \end{trivlist}

    Once \WebQuiz is initialised you may also want to run:
    \begin{bashcode}
       > webquiz --edit-settings
    \end{bashcode}
    This allows you to set various system defaults for the quizzes that,
    for example, specify the name and URL for your department and
    institution as well as the default language and theme used for the
    quizzes. All of these defaults can be overridden in any \LaTeX\ file
    for a quiz.

    When changing the settings \WebQuiz tries to explain what it is
    doing at each step. When you run \BashCode|webquiz --edit-settings|
    you will be prompted for the following:
    \begin{description}[nosep, labelwidth=18ex]
      \item[breadcrumbs] breadcrumbs at the top of quiz page
      \item[department] name of department
      \item[department-url] url for department
      \item[engine] default tex engine used to compile web pages
      \item[hide-side-menu] do not display the side menu at start of quiz
      \item[institution] institution or university
      \item[institution-url] url for institution or university
      \item[language] default language used on web pages
      \item[one-page] display questions on one page
      \item[random-order] randomly order the quiz questions
      \item[theme] default colour theme used on web pages
      \item[version] webquiz version number for webquizrc settings
      \item[webquiz-url] relative url for webquiz web directory
      \item[webquiz-www] full path to webquiz web directory
      \item[make4ht] build file for make4ht
      \item[mathjax] url for mathjax
      \item[webquiz-format] name of python module that formats the quizzes
    \end{description}
    The last three options are \textit{advanced options} that you probably do not
    want to change. For more details see \autoref{SS:rcfile}, which
    describes the \webquizrc file.  The default values of all of these
    settings can be overridden in the \LaTeX{} file for the  quiz and
    the default values can be changed at any time using
    \BashCode|webquiz --edit-settings|.

     When changing the settings for \WebQuiz if you are unsure what
     a particular setting does then you should just hit return to accept
     the default value --- the default value is printed
     inside square brackets as \BashCode|[default]|. In particular, you
     are strongly recommended to accept the default values for all of
     the advanced options because it is very unlikely that you will need
     to change these when you first use \WebQuiz.

    \subsection{Pstricks configuration and dvisgm}
    \label{SS:pstricks}

\index{pstricks}
\noindent
Using \ctan{pstricks} is, in some sense, easier that \ctan[pgf]{tikz}
because it seems to work better with \ctan{tex4ht}, however, because of
recent changes in \ctan{tex4ht} and \ctan{pstricks} you should install
the latest updates of both packages.  In addition, because \ctan{tikz}
uses  \ctan{dvisvgm} to convert the images created by \ctan{pstricks},
it is necessary to make sure that this package is correctly
installed. At first sight this looks OK because \ctan{dvisvgm} is
included in \TeXLive and Mik\TeX, however, \ctan{dvisvgm} uses
\href{https://www.ghostscript.com/}{Ghostscript} and this needs to be
correctly configured and, as outlined in
\href{http://dvisvgm.bplaced.net/FAQ}{FAQ}, \ctan{dvisvgm} needs to know
where to find the \href{https://www.ghostscript.com/}{Ghostscript}
libraries. For example, to get \ctan{dvisvgm} to work on my system I
needed to add the line
\begin{bashcode}
  export LIBGS=/usr/local/lib/libgs.dylib
\end{bashcode}
to my \BashCode|.bashrc| file. To see whether you need to do something
similar on your system you nee to look at the output from the following
two commands:

  \begin{bashcode}
    > dvisvgm -h
    > dvisvgm -l
  \end{bashcode}

\noindent There are three possibilities:

  \begin{itemize}
    \item the \BashCode|-h| output does not contain \BashCode|-libgs|
    and the \BashCode|-l| output contains \BashCode|ps|:
    ghostscript was linked at build time, so everything should work
    \item
    the \BashCode|-h| output contains \BashCode|-libgs| and the
    \BashCode|-l| output does not contain \BashCode|ps|:
    gpostscript support is enabled but ghostscript is not linked. You
    need to locate the ghostscript library
    \BashCode|libgs.so| or \BashCode|libgs.dylib| on your system and set
    the \BashCode|LIBGS| environment variable, or equivalent,
    accordingly
    \item the \BashCode|-h| output does not contain \BashCode|-libgs|
    and the \BashCode|-l| output does not contain \BashCode|ps|:
    \ctan{dvisgm} was not built with postscript support. In this
    case, \WebQuiz will not be able to process \ctan{svg} images. You
    need to reinstall \ctan{dvisgm} with ghostscript support.
  \end{itemize}

\section{The WebQuiz program} \index{usage}

    The \WebQuiz program was designed to be run from the command-line,
    so to process the file \BashCode|quiz.tex| using \WebQuiz type:

    \BashCode|> webquiz quiz| \qquad or \qquad \BashCode|> webquiz quiz.tex|

    \noindent (Here ``$>$'' is the command-line prompt.) One useful
    feature of \WebQuiz is that you can ask it to process more than one
    quiz file at a time. For example, if you have quiz files
    \BashCode|quiz1.tex|, ..., \BashCode|quiz9.tex| in the current directory
    then, on a UNIX system, you can rebuild the web pages for all of
    these quizzes using the single command:

    \BashCode|> webquiz quiz[1-9].tex|

    \noindent
    This is useful if some aspect of all of the quizzes has changed. In
    fact, one would probably use

    \BashCode|> webquiz --qq quiz[1-9].tex|

    \noindent
    because the \BashCode|webquiz --qq| command-line option suppresses
    almost all of the output produced by \LaTeX\ and \TeXfht. The next
    section discusses the \WebQuiz command-line options.

    \subsection{Command-line options}\label{SS:commandline}
    \index{command-line options}

    \InputBashCode{webquiz.usage}

The command-line options are listed on separate lines here to improve
readability but they can appear in any order when you use them provided
that they are all on the same line.  The order that the command-line
options are listed in indicates how often you are likely to need this
option.

    \begin{description}
       \item[-h, --help] \CrossIndex{command-line options}{help}
          list the command-line options and exit

       \item[-l, \ddash lau] Use lualatex to compile the quiz
       \index{lualatex}

       \item[-s,\ddash shell-escape] \CrossIndex{command-line options}{shell-escape}
       \item[-s,\ddash shell-escape] \CrossIndex{command-line options}{shell-escape}
          Shell escape for htlatex/make4ht

       \item[-q, -qq, \ddash quiet] \CrossIndex{command-line options}{quiet mode}
       Suppress tex4ht messages: \BashCode|-q| is quiet and
       \BashCode|-qq| is  very quiet. If you use
       \BashCode|webquiz -qq texfile.tex| then almost no output will
       be printed by \WebQuiz when it is processing your quiz file. Be
       warned, however, that both of these options can make it harder to find and fix
       errors, so using the \BashCode|-q| and \BashCode|-qq| options is not
       recommended if your file is not compiling.

       \item[-x, \ddash xelatex] Use xelatex to compile the quiz
       \index{xelatex}

       \end{description}

        \noindent
       The following options relate to the settings for \WebQuiz:
       \begin{description}

       \item[-i, \ddash initialise] \CrossIndex{command-line options}{initialise}
          Initialise files and settings for webquiz. The command
          \BashCode|webquiz --initialise| should be run before using
          \WebQuiz. This command will help you to copy the web files needed by
          \WebQuiz into the directories used by your web server. See
          \autoref{S:configuration} for more details.

       \item[\ddash edit-settings] \CrossIndex{command-line options}{edit-settings}
          Edit the webquiz settings in the rc-file. The settings that
          are marked as advanced should be changed only with care. To
          change the settings for \BashCode|webquiz-www| and \BashCode|webquiz-url|
          use: \BashCode|webquiz --initialise|.

       \item[\ddash settings {[SETTING]}] \CrossIndex{command-line options}{settings}
          \SeeIndex{rc file}{webquizrc}
          \SeeIndex{default settings}{webquizrc}
          List system settings for webquiz stored in the \WebQuiz rc-file
          (\textit{run-time configuration file}); see the
          \BashCode|--rcfile| command-line option below. Optionally, a single
          \BashCode|SETTING| can be given in which case the value of only that
          setting is returned. Typical settings returned by this
          command look like:
          \InputBashCode{webquiz.settings}
          These settings are used by all of the quizzes created by
          \WebQuiz unless the quiz overrides them. These defaults are
          normally set when \WebQuiz is initialised, using the command:
          \BashCode|webquiz --initialise|. The defaults can be changed any time using:
          \BashCode|webquiz --edit-settings|.

          If \BashCode|SETTING| is omitted then the list of current
          settings are printed. Use
          \BashCode|SETTING|$=$\BashCode|verbose| for
          a more verbose listing of the settings and
          \BashCode|SETTING|$=$\BashCode|help| for a summary of the
          settings.

          \textit{If you do not have permission to write to the system
          rc-file, which is in the \WebQuiz scripts directory, then you
          will be given the option of saving the \WebQuiz settings to an
          rc-file in your home directory or another file of your choice.
          If you want to save the settings to a particular rc-file use
          the \BashCode|--rcfile| option. If you want to change the
          system \webquizrc file then use a administrator account or,
          on a unix-like system,} use \BashCode|sudo webquiz --edit-settings|.

       \item[-r RCFILE, \ddash rcfile RCFILE]\index{webquizrc}
          Specify the rc-file (\textit{run-time configuration file}), file
          that \WebQuiz should read (and write to).

          By default, \WebQuiz will first read the system rc-file (the
          file \webquizrc in the \WebQuiz script directory),
          followed by the users rc-file in your home directory, which
          will typically be in the home directory but first preference
          will be given to the \BashCode|.dotfiles/config| directory
          if it exists (see \autoref{SS:rcfile}). If \WebQuiz is unable to
          write to the system rc-file then \WebQuiz will create an
          rc-file in your home directory, which it will subsequently
          read whenever you run \WebQuiz. This setting is only
          necessary if you want to override the default rc-file.

          The rc-file can be edited by hand, however, it is
          recommended that you instead use
          \begin{bashcode}
            webquiz --edit-settings
          \end{bashcode}

          \item[\ddash uninstall] Remove all \WebQuiz files from your
          web server directory. This command only removes files that
          \WebQuiz may have installed on your web server. It does
          \textit{not} remove \WebQuiz from your \LaTeX{} distribution.

        \end{description}

        \noindent
        The remaining options are for advanced usage and will not be
        needed by most users:
        \textit{An incorrect value for these settings may stop \WebQuiz from working.}

        \begin{description}
           \item[\ddash make4ht MAKE4HT-OPTIONS]
           \CrossIndex{command-line options}{make4ht}
          Options to be passed to \ctan{make4ht} when converting the
          \LaTeX{} to \BashCode|xml|. This option is equivalent to setting
          the \BashCode|make4ht| in the \webquizrc file; see
          \autoref{SS:rcfile}. At least under UNIX, multiple arguments
          should be enclosed in quotes.  For example, to give
          \ctan{make4ht} an mk4 file you would use
          \begin{bashcode}
                    > webquiz --make4ht "-e file.mk4" quiz.tex
          \end{bashcode}
          \WebQuiz provides the command-line options
          \BashCode|--shell-escape|, \BashCode|-lua|, \BashCode|--xelatex|,
          so the \BashCode|make4ht| command-line option will be required
          only in rare instances.
          \index{lualatex}\index{xelatex}

       \item[\ddash webquiz-format WEBQUIZ-FORMAT] \CrossIndex{command-line options}{layout}
          Local python code for formatting the quiz web page. This
          option is equivalent to setting the \LatexCode|webquiz-format|
          in the \webquizrc file.

    \end{description}

    \subsection{The \webquizrc file}\label{SS:customisation}
    \label{SS:rcfile}\index{default settings}

    \WebQuiz stores various \textit{run-time configuration settings} in
    the \webquizrc file. The system rc-file, \webquizrc, is
    saved in the \BashCode|scripts/webquiz| subdirectory of the
    \BashCode|TEXMFLOCAL| directory. If you are using \TeXLive (and
    possibly Mik\TeX?) you can find this directory using the command
    \begin{bashcode}
            kpsewhich -var-value=TEXMFLOCAL
    \end{bashcode}
    By default, the \WebQuiz settings are saved here so that you do not
    need to reinitialise \WebQuiz whenever you update your \TeX{}
    distribution. If you do not have permission to write to this
    directory then you will be asked if you would like to save the
    rc-file somewhere else.

    Each time \WebQuiz is run it first reads the system rc-file, if it
    exits, followed by the \textit{users rc-file}, which is the file
    \BashCode|.webquizrc| in the users home directory -- again, the users
    rc-file is read only if it exists. You can specify an arbitrary file
    as an rc-file using command  \BashCode|webquiz --rc-file|. When you change
    the \WebQuiz settings, using \BashCode|webquiz --edit-settings|, the new
    settings are saved in the last rc-file that was read, which will
    always be the rc-file given by \BashCode|webquiz --rc-file| when this
    command-line option is used.  As with the system rc-file, if you do
    not have permission to write to the specified rc-file then you will
    be asked for a different install location.

    Defaults for many of the generic \WebQuiz settings can be set the
    \webquizrc file. To describe these settings we consider separately
    the ``information settings'', the ``layout settings'' and the
    ``advanced settings'' that most people will not need and should not
    change.

    \subsubsection*{Configuration settings}
    \begin{description}

        \item[webquiz-url] This is the relative URL for webquiz web directory

        \item[webquiz-www] This is the full path to the webquiz web
            directory. The on-line manual and other example code can be
            found in the \BashCode|docs| subdirectory. If you use
            \BashCode|bash|, then the command
            \begin{bashcode}
                > cd $(webquiz --settings webquiz-www)/docs
            \end{bashcode}%$
            will take you to the \WebQuiz on-line \BashCode|docs/| directory.

    \end{description}

    \subsubsection*{Department settings}
    \begin{description}

        \item[department]
          Sets the default department name; see \LatexCode|\Department| in
          \autoref{SS:breadcrumbs}\macroIndex{Department}

        \item[department-url]
          Sets the URL for the department; see \LatexCode|\DepartmentURL| in
          \autoref{SS:breadcrumbs}\macroIndex{DepartmentURL}

        \item[institution]
          Sets the default institution name. This is overrid see \LatexCode|\Institution| in
          \autoref{SS:breadcrumbs}\macroIndex{Department}

        \item[institution-url]
          Sets the URL for the institution; see \LatexCode|\InstitutionURL| in
          \autoref{SS:breadcrumbs} \macroIndex{InstitutionURL}

      \end{description}

      \subsubsection*{Formatting options}

      \begin{description}

        \item[breadcrumbs]\index{breadcrumbs}
          Sets the default breadcrumbs at the top of quiz page. The
          breadcrumbs can be overwritten in the quiz file using the
          \LatexCode{\BreadCrumbs} command. See \autoref{SS:breadcrumbs}
          for more details.  \macroIndex{BreadCrumbs}

        \item[engine] Sets the default \TeX{} engine to be used when
          compiling the quiz. By default, \BashCode|latex| is used, with
          the two other possibilities being \BashCode|lua| and
          \BashCode|xelatex|. Behind the scenes, the two choices
          correspond to invoking \ctan{make4ht} with the
          \BashCode|--lua| and \BashCode|--xelatex| options,
          respectively. The \BashCode|engine| setting in the  \webquizrc
          file is overridden by the command-line options
          \BashCode|--latex|, \BashCode|--lua| and \BashCode|--xelatex|.
          \index{lualatex}\index{xelatex}

        \item[language]\index{language}
          Sets the default language for the \WebQuiz web pages. This can
          be overridden in the quiz file by using the document class
          \LatexCode|language| option: \LatexCode|language=XXX|
          See \autoref{SS:classOptions}.

        \item[theme]\index{theme}
          Sets the default colour theme for the \WebQuiz web pages. This can
          be overridden in the quiz file by using the document class
          \LatexCode|theme| option: \LatexCode|theme=XXX|.
          See \autoref{SS:classOptions}.

      \end{description}

      \subsubsection*{Advanced options}

      \begin{description}

        \item[make4ht]\index{make4ht}
          Options for \ctan{make4ht}. Rather than using \TeXfht
          directly, \WebQuiz uses \ctan{make4ht} to convert the
          \LaTeX\ file to \BashCode|xml|. Use this option to
          customise how \ctan{make4ht} is called. See the
          \ctan[make4ht]{make4ht manual} for more information.

        \item[mathjax] \WebQuiz web pages use \href{https://www.mathjax.org/}{mathjax} to
          render the mathematics on the quiz web pages. By default this
          is done by loading \BashCode|mathjax| from
          \begin{bashcode}
             https://cdnjs.cloudflare.com/ajax/libs/mathjax/2.7.1/MathJax.js
          \end{bashcode}
          Fetching \BashCode|mathjax| from an external site can
          cause a delay when the quiz web pages are loaded. This setting
          in the rc-file allows you to change where \BashCode|mathjax| is
          loaded from. For example, if you install \BashCode|mathjax| on
          your web server then you would replace this will the
          corresponding relative URL.

        \item[webquiz-format]
          Sets the python file that writes the HTML file for the quiz.
          Most people will not need this option. The next subsection
          describes how to do this in more detail.

    \end{description}

    The values of \LatexCode|breadcrumbs|, \LatexCode|department|,
    \LatexCode|department_url|, \LatexCode|institution| and
    \LatexCode|institution_url| can all be overwritten using the
    corresponding  \LaTeX{} commands in the quiz file; see
    \autoref{SS:breadcrumbs} for more details.

    \subsection{Changing the layout of the \WebQuiz web pages} \index{layout}

    \begin{description}
      \item[\raisebox{-12mm}{\dbend}] \vspace*{-12mm}
      \textit{This is an advanced \WebQuiz feature that most people will
      not need. To change the layout of the quiz web pages created by
      \WebQuiz requires working knowledge of HTML and python.}
    \end{description}

    The construction of the online quizzes is controlled by the python
    file \PythonCode|webquiz_standard.py|. If you want to change the structure
    of the web pages for the quizzes then the easiest way to do this is
    to make a copy of \PythonCode|webquiz_standard.py|, say
    \PythonCode|webquiz_myformat.py|, and then edit this file directly.
    This will require working knowledge of python and HTML. To give you
    an idea of what needs to be done, the python file
    \PythonCode|webquiz_standard| contains a single function
    \PythonCode|write_web_page| that returns the HTML for the page as
    a string using the following:
        \begin{htmlcode}
        quiz_page = r'''<!DOCTYPE HTML>
        <html lang="en">
          <head>
            <title> {title} </title>
            {htmlpreamble}
          </head>

          <body>
            {breadcrumbs}
            {no_script}
            <div class="quiz_page">
              {side_menu}
              <div class="quiz_questions">
                {quiz_header}
                {quiz_questions}
              </div>
            </div>
            {webquiz_init}
          </body>
        </html>
        '''
        \end{htmlcode}
  By changing this output you can change the format of the quizzes
  produced by \WebQuiz. For example, by adding code to the
  \HTMLCode|<head>...</head>| section of \HTMLCode|quiz_page| it is easy to
  include new CSS code and by modifying \HTMLCode|<body>....</body>|.
  More sophisticated versions of \PythonCode|webquiz_standard.py|, where you
  change the underlying python code, are possible.  For example, at the
  University of Sydney we have a custom version of \PythonCode|webquiz_standard|
  that calls our content management system and, in this way, embeds the
  quiz web page inside a standard web page for our university.

  When experimenting with a new format can run
  \WebQuiz using the command:
  \begin{bashcode}
     > webquiz --webquiz-format webquiz_myformat quizfile.tex
  \end{bashcode}
  Once you have finalised the new format you can make this the default
  format using
  \BashCode|webquiz --edit-settings|.\index{edit-settings}
  \index{webquiz-format}

  If you do make modifications to any of these files it is advisable to
  create a version of the file \textit{with a different name}. This
  will make it easier for you to integrate your changes with any future
  released of \WebQuiz.

  \subsection{Bugs, issues and feature requests}\index{bug reports}
    Please report any bugs, issues or feature requests using the
    \textit{issue} tracker at
    \begin{quote}
    \href{https://\webquiz{repository}}{\webquiz{repository}}.
    \end{quote}
    When reporting a bug please provide a \textit{minimal working example}
    that clearly demonstrates your problem. This should be a compilable
    \LaTeX file that looks something like the following:
    \begin{latexcode}
       \documentclass{webquiz}
       \begin{document}
            ** insert problematic code here **
       \end{document}
    \end{latexcode}
    Bug reports that do not have a minimal working example are often
    hard to reproduce in which case it is not possible to fix them.
    Before submitting a bug export please first compile your quiz using
    (pdf)latex to check to see if your problem is an issue with \LaTeX
    or with \WebQuiz. If you can, please also test to see if your code
    compiles using \TeXfht directly.

\appendix
\newpage
\section*{Appendices}
  \addcontentsline{toc}{section}{Appendices}
\renewcommand\thesection{}
\renewcommand\thesubsection{\Alph{subsection}}

  \subsection{Catalogue of web page themes}\index{SS:themes}\label{SS:themes}
  \WebQuiz{} comes with quite a few different themes for the on-line
  quizzes, which can be set using the \LatexCode|theme| documentclass
  option or in the \webquizrc file; see \autoref{SS:classOptions} for
  more details. Themes are easy to construct in principle although
  finding colour that work well together can be tricky in practice.
  Some of the available themes are quite garish, so perhaps the
  different themes should be interpreted as proof-of-concept rather
  than the final product.

  \input webquiz-themes.tex

  I am happy to include any new themes sent to me and to consider
  modifications of existing themes, although there is a potential
  technical issue here in that the underlying \BashCode|css| files are
  written using \href{http://sass-lang.com/}{sass}.  Any new themes can
  be sent to me using the \textit{issue} tracker at
    \href{https://\webquiz{repository}}{\webquiz{repository}}.

\subsection{The on-line WebQuiz manual}\label{S:online}

  \WebQuiz has an
  \href{http://www.maths.usyd.edu.au/u/mathas/WebQuiz/webquiz-online-manual.html}{on-line
  manual} that is a \LaTeX file written with the \LatexCode|webquiz|
  document class. The conversion of the manual from \LaTeX{} to HTML is
  done by \WebQuiz. The PDF version of this manual is included here for
  your convenience. The source file for the on-line manual is included
  in the documentation of \WebQuiz to allow you to create a local
  version of the on-line manual. Look for the file
  \BashCode|webquiz-online-manual.tex| in the \BashCode|webquiz-www/docs|
  directory; see~\autoref{SS:rcfile}.

  The on-line manual was written for ``internal use'' when \WebQuiz was
  first written in 2004. \WebQuiz has evolved quite a lot since then.
  There is quite a bit of overlap between the on-line manual and
  previous sections, however, the on-line manual only describes how to
  typeset questions and it does not cover some of the more recent
  features of \WebQuiz or how to use the program.  If there are any
  discrepancies between the on-line manual and the earlier sections of
  this manual then the on-line manual should be discounted.

  The on-line manual has diagrams that are drawn using \ctan{pstricks}
  and, as a result, to create a PDF version of the on-line manual use
  \BashCode|latex webquiz-online-manual| to create a \BashCode|dvi file|. The
  dvi file can be  converted to PDF using \BashCode|dvipdf|. If you try to
  use \BashCode|pdflatex| directly you will get an error.

  \includepdf[pages=-,pagecommand={\pagestyle{webquiz}}]{webquiz-online-manual}


\subsection{Licence}

Copyright (C) 2013-2017

\href{https://www.gnu.org/licenses/gpl-3.0.en.html}{GNU General Public License, Version 3, 29 June 2007}

This program is free software: you can redistribute it and/or modify it under
the terms of the GNU General Public License (GPL) as published by the Free
Software Foundation, either version 3 of the License, or (at your option) any
later version.

This program is distributed in the hope that it will be useful, but WITHOUT ANY
WARRANTY; without even the implied warranty of MERCHANTABILITY or FITNESS FOR A
PARTICULAR PURPOSE.  See the GNU General Public License for more details.


\vfil
\begin{tabular}{@{}ll}
Authors             & \webquiz{authors}\\
Description         & \webquiz{description}\\
Maintainer          & \webquiz{name}\\
System requirements & \webquiz{requirements}\\
Licence             & \webquiz{licence}\\
Release date        & \webquiz{release date}\\
Repository          & \href{https://\webquiz{repository}}{\webquiz{repository}}
\end{tabular}
\eject

\printindex

\end{document}
