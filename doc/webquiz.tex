%-----------------------------------------------------------------------------
%  Copyright (C) 2004-2019 Andrew Mathas, University of Sydney
%
%  Distributed under the terms of the GNU General Public License (GPL)
%                  http://www.gnu.org/licenses/
%
% This file is part of the WebQuiz system.
%
% <Andrew.Mathas@sydney.edu.au>
%-----------------------------------------------------------------------------

\synctex=1

\PassOptionsToClass{tikz,svgnames}{xcolor}
\documentclass[svgnames]{article}
\usepackage[a4paper,margin=30mm]{geometry}
\parindent=4mm
\parskip=1mm
\hfuzz 5pt
\synctex=1

% load webquiz-ini and webquiz-doc code -- doc needs ini so ini is first
% -----------------------------------------------------------------------
%   webquiz-ini.sty | read the webquiz.ini file into pgfkeys
% -----------------------------------------------------------------------
%
%   Copyright (C) Andrew Mathas, University of Sydney
%
%   Distributed under the terms of the GNU General Public License (GPL)
%               http://www.gnu.org/licenses/
%
%   This file is part of the webquiz system.
%
%   <Andrew.Mathas@sydney.edu.au>
% ----------------------------------------------------------------------

%% Read the webquiz.ini file into \pgfkeys for later use so that we
%% don't need to hard-code version information. We allow unknown keys so
%% that we can slurp in all of the information in the file into
%% \pgfkeys{/webquiz}
\RequirePackage{pgfkeys}
\pgfkeys{/webquiz/.is family, /webquiz,
  % allow arbitrary unknown keys and set with \pgfkeyssetvalue
  .unknown/.code={\pgfkeyssetvalue{\pgfkeyscurrentpath/\pgfkeyscurrentname}{#1}},
}
\newcommand\webquiz[1]{\pgfkeysvalueof{/webquiz/#1}}

% split input line into key-value pairs -- necessary as commas can appear in the value
\RequirePackage{xparse}
% \AddIniFileKeyValue{key=value} - take a single argument and split it on =
\NewDocumentCommand{\AddIniFileKeyValue}{ >{\SplitList{=}} m }{%
  \AddIniFileValue #1%
}
% put a key-value pair into \pgfkeys{/webquiz}
% \AddIniFile{key}{value}  - note that value may contain commas
\newcommand\AddIniFileValue[2]{\expandafter\pgfkeys\expandafter{/webquiz,#1={#2}}}

% read the webquiz.ini file into \pgfkeys{/webquiz}
\newread\inifile% file handler
\def\apar{\par}% \ifx\par won't work but \ifx\apar will
\newcommand\AddEntry[1]{\expandafter\pgfkeys\expandafter{/webquiz, #1}}
\openin\inifile=webquiz.ini% open file for reading
\loop\unless\ifeof\inifile% loop until end of file
  \read\inifile to \iniline% read line from file
  \ifx\iniline\apar% test for, and ignore, \par
  \else%
    \ifx\iniline\empty\relax% skip over empty lines/comments
    \else\expandafter\AddIniFileKeyValue\expandafter{\iniline}
    \fi%
  \fi%
\repeat% end of file reading loop
\closein\inifile% close input file

\endinput

%-----------------------------------------------------------------------------
%  Copyright (C) 2004-2019 Andrew Mathas, University of Sydney
%
%  Distributed under the terms of the GNU General Public License (GPL)
%                  http://www.gnu.org/licenses/
%
% This file is part of the WebQuiz system.
%
% <Andrew.Mathas@sydney.edu.au>
%-----------------------------------------------------------------------------

% common latex code in the documentation files
\RequirePackage{cmap} % fix search and cut-and-paste in Acrobat

\RequirePackage[utf8]{inputenc}
\RequirePackage[T1]{fontenc}
\RequirePackage{xparse}

\RequirePackage{enumitem}
\setlist[itemize]{nosep}
\setlist[description]{
    font=\sffamily\bfseries\color{DodgerBlue},
    labelwidth=\textwidth
}

\RequirePackage{graphicx}

\RequirePackage{xspace}
\RequirePackage[svgnames]{xcolor}

\usepackage{hologo}
\newcommand\TeXfht{\href{https://www.ctan.org/tex4ht}{\hologo{TeX4ht}}\xspace}

\NewDocumentCommand\ctan{ O{pkg/#2} m}{\href{https://www.ctan.org/#1}{\texttt{#2}}\xspace}

\renewcommand\HTML{\href{https://www.w3schools.com/html/html_intro.asp}{HTML}\xspace}
\newcommand\CSS{\href{https://www.w3schools.com/css}{CSS}\xspace}
\newcommand\XML{\href{https://www.w3schools.com/xml/xml_whatis.asp}{XML}\xspace}
\newcommand\Javascript{\href{https://www.w3schools.com/Js/}{Javascript}\xspace}
\newcommand\python[1][python]{\href{https://www.python.org/}{#1}\xspace}
\newcommand\Ghostscript{\href{https://www.ghostscript.com/}{Ghostscript}\xspace}

%%%%%%%%%%%%%%%%%%%%%%%%%%%%%%%%%%%%%%%%%%%%%%%%%%%%%%%%%%%%%%%%%%%%%%%%%%%%%%%%%%%%%%%
%% listings code for python and latex examples in webquiz documentation
\RequirePackage{listings}
\lstset{%
    % Basic design
    backgroundcolor=\color{LightYellow!50!White},
    basicstyle={\small\ttfamily},
    boxpos=c, % centered
    % frame
    framerule=2pt,
    frame=l,
    framerule=0.5mm,
    rulesep=10mm,
    rulecolor=\color{Peru},
    breaklines=true,
    % colours for keywords etc
    commentstyle=\color{DarkRed},
    keywordstyle=\color{MediumBlue},
    linewidth=14cm,
    basewidth  = {.5em,0.5em},
    numbers=none,
    resetmargins=true,
    tabsize=2,
    xleftmargin=10mm,
    classoffset=1,
    % Code design
    keywordstyle={[1]\color{Blue}\bfseries},
    keywordstyle={[2]\color{ForestGreen}},
    keywordstyle={[3]\color{LimeGreen}},
    keywordstyle={[4]\color{DarkBlue}},
    commentstyle=\color{BurlyWood}\ttfamily,
    stringstyle=\color{Coral},
    tabsize=4,
    showtabs=false,
    showspaces=false,
    showstringspaces=false,
    inputencoding=utf8,
    extendedchars=true,
    literate={á}{{\'a}}1 {é}{{\'e}}1 {í}{{\'i}}1 {ó}{{\'o}}1 {ú}{{\'u}}1
      {Á}{{\'A}}1 {É}{{\'E}}1 {Í}{{\'I}}1 {Ó}{{\'O}}1 {Ú}{{\'U}}1
      {à}{{\`a}}1 {è}{{\`e}}1 {ì}{{\`i}}1 {ò}{{\`o}}1 {ù}{{\`u}}1
      {À}{{\`A}}1 {È}{{\'E}}1 {Ì}{{\`I}}1 {Ò}{{\`O}}1 {Ù}{{\`U}}1
      {ä}{{\"a}}1 {ë}{{\"e}}1 {ï}{{\"i}}1 {ö}{{\"o}}1 {ü}{{\"u}}1
      {Ä}{{\"A}}1 {Ë}{{\"E}}1 {Ï}{{\"I}}1 {Ö}{{\"O}}1 {Ü}{{\"U}}1
      {â}{{\^a}}1 {ê}{{\^e}}1 {î}{{\^i}}1 {ô}{{\^o}}1 {û}{{\^u}}1
      {Â}{{\^A}}1 {Ê}{{\^E}}1 {Î}{{\^I}}1 {Ô}{{\^O}}1 {Û}{{\^U}}1
      {œ}{{\oe}}1 {Œ}{{\OE}}1 {æ}{{\ae}}1 {Æ}{{\AE}}1 {ß}{{\ss}}1
      {ű}{{\H{u}}}1 {Ű}{{\H{U}}}1 {ő}{{\H{o}}}1 {Ő}{{\H{O}}}1
      {ç}{{\c c}}1 {Ç}{{\c C}}1 {ø}{{\o}}1 {å}{{\r a}}1 {Å}{{\r A}}1
      {€}{{\euro}}1 {£}{{\pounds}}1 {«}{{\guillemotleft}}1
      {»}{{\guillemotright}}1{ñ}{{\~n}}1{Ñ}{{\~N}}1{¿}{{?`}}1
      {á}{{\'a}}1{í}{{\'i}}1{é}{{\'e}}1{ý}{{\'y}}1{ú}{{\'u}}1{ó}{{\'o}}1
      {ě}{{\v{e}}}1{š}{{\v{s}}}1{č}{{\v{c}}}1{ř}{{\v{r}}}1{ž}{{\v{z}}}1{ď}{{\v{d}}}1
      {ť}{{\v{t}}}1{ň}{{\v{n}}}1{ů}{{\r{u}}}1{Á}{{\'A}}1{Í}{{\'I}}1{É}{{\'E}}1
      {Ý}{{\'Y}}1{Ú}{{\'U}}1{Ó}{{\'O}}1{Ě}{{\v{E}}}1{Š}{{\v{S}}}1{Č}{{\v{C}}}1
      {Ř}{{\v{R}}}1{Ž}{{\v{Z}}}1{Ď}{{\v{D}}}1{Ť}{{\v{T}}}1{Ň}{{\v{N}}}1{Ů}{{\r{U}}}1
}
\lstdefinestyle{latexcode}{
  language=[LaTeX]TeX,
  texcsstyle=*\bfseries\color{DodgerBlue},
  classoffset=1,
  morekeywords={UnitCode,UnitName,UnitURL,QuizzesURL,BreadCrumbs,BreadCrumb,Department,DepartmentURL,
                includegraphics,DeclareGraphicsExtensions,DisplayAsImage,thechoice,quiz,
                Institution,InstitutionURL,feedback,correct,incorrect,whenRight,whenWrong,answer,
                dref, qref, Qref},
  classoffset=2,
  morekeywords={quizindex, discussion, question, multiple, single, columns, choice},
  classoffset=3,
  alsoletter=2,
  morekeywords={theme, language, hide side menu, pst2pdf, tikz, fixed order, random order,
                show side menu, one page, separate pages, english, french, integer, complex,
                number, string, lowercase, pspicture},
  classoffset=4,
  morekeywords={tikzset, foreach, draw, filldraw, Configure, NewConfigure, csdef, csletcs, csuse,
                Picture, EndPicture, psplotThreeD, pstThreeDCoor, RequirePackage, DeclareGraphicsExtensions},
}
\lstnewenvironment{latexcode}{\lstset{style=latexcode}}{}
\NewDocumentCommand\LatexCode{v}{\lstinline[style=latexcode]|#1|}
\NewDocumentCommand\InputLatexCode{ O{examples/} m}{\lstinputlisting[style=latexcode]{#1#2}}

\lstdefinestyle{bashcode}{
  language=bash,
  upquote=true,
  classoffset=1,
  morekeywords={webquiz, kpsewhich, latex, pdflatex, xelatex. lualatex},
  classoffset=2,
  morekeywords={>}
}
\lstnewenvironment{bashcode}{\lstset{style=bashcode}}{}
\NewDocumentCommand\BashCode{v}{\lstinline[style=bashcode]|#1|}
\newcommand\InputBashCode[1]{\lstinputlisting[style=bashcode]{#1}}

\lstdefinestyle{pythoncode}{
  language=HTML,
  upquote=true,
  classoffset=1,
  morekeywords={quiz_page},
}
\lstnewenvironment{pythoncode}{\lstset{style=pythoncode}}{}
\NewDocumentCommand\PythonCode{v}{\lstinline[style=pythoncode]|#1|}
\newcommand\InputPythoCode[1]{\lstinputlisting[style=pythoncode]{#1}}

\lstdefinestyle{htmlcode}{
  language=HTML,
  upquote=true,
  morekeywords={no}
  classoffset=1,
  morekeywords={quiz_page, side_menu, quiz_questions, quiz_header, no_script, include, breadcrumbs,},
}
\lstnewenvironment{htmlcode}{\lstset{style=htmlcode}}{}
\NewDocumentCommand\HTMLCode{v}{\lstinline[style=htmlcode]|#1|}

\newcommand\DefaultValue[1]{\textcolor{ForestGreen}{\texttt{#1}}}
%%%%%%%%%%%%%%%%%%%%%%%%%%%%%%%%%%%%%%%%%%%%%%%%%%%%%%%%%%%%%%%%%%%%%%%%%%%%%%%%%%%%%%%
\RequirePackage{tcolorbox}
\tcbuselibrary{skins}
\NewDocumentCommand\ScreenShot{ O{0.7} m O{examples/} m }{%
  \begin{center}
    \begin{tcolorbox}[width=#1\textwidth, title=#2, arc=3mm,
      colframe=Peru,
      boxrule=0.2mm,
      colback=white,
      halign=center,
      enhanced,
      attach boxed title to bottom right={yshift=3mm, xshift=-3mm},
      fonttitle=\footnotesize,
      boxed title style={size=fbox, colback=white, arc=2mm},
      coltitle=Peru,
    ] \includegraphics[width=\textwidth]{#3#4}
    \end{tcolorbox}
  \end{center}%
}

%%%%%%%%%%%%%%%%%%%%%%%%%%%%%%%%%%%%%%%%%%%%%%%%%%%%%%%%%%%%%%%%%%%%%%%%%%%%%%%%%%%%%%%
%% hyperref settings

\AtEndPreamble{
  \RequirePackage{hyperref}
  \hypersetup{%
      colorlinks=true,
      linkcolor=SaddleBrown,
      urlcolor=Brown,
      pdfauthor   = {\webquiz{authors}                       },
      pdfkeywords = {\webquiz{keywords}                      },
      pdfsubject  = {\webquiz{description}                   },
      pdfinfo     = {%
          copyright    = {\webquiz{copyright}},
          licence      = {\webquiz{licence}},
          release date = {\webquiz{release date}},
          url          = {\webquiz{url}},
      }
  }
}
\endinput


% Can't use same definition in webquiz-online-manual because of conflict
% with code in webquiz.cfg to make color work with listings
\newcommand\WebQuiz{\textcolor{Sienna}{WebQuiz}\xspace}

\usepackage{textcomp}
\usepackage{pdfpages}
\usepackage{manfnt}
\usepackage{booktabs}
\usepackage{bbding}
\usepackage{pifont}
\usepackage{pgffor}

\usepackage{etoolbox}% used below to patch l@section to remove extraneous spacing
\usepackage{appendix}
\def\sectionautorefname{Chapter}
\def\subsectionautorefname{Section}
\def\subsubsectionautorefname{\S\kern-0.8ex}
\def\appendixautorefname{Appendix}

\usepackage{imakeidx}
\indexsetup{level=\section*,toclevel=section,noclearpage}
\makeindex[intoc,columns=3]

\newenvironment{heading}[1][]
{\noindent\trivlist\item[\hskip\labelsep\textbf{#1}]}
  {\endtrivlist}

\newenvironment{dangerous}{\trivlist\item[\hskip\labelsep\textcolor{red}{\textdbend}]}{\endtrivlist}

\setcounter{secnumdepth}{3}
\setcounter{tocdepth}{3}
\renewcommand\thesubsubsection{\thesubsection\alph{subsubsection}}
\newcommand\lowerCaseIndex[1]{%
  \lowercase{\def\temp{#1}}%
  \expandafter\index\expandafter{\temp@#1}%
}
% usage: \CrossIndex[*]{main entry}[*]{subentry} with *'s for macros
\NewDocumentCommand\CrossIndex{ smsm }{%
  \IfBooleanTF{#1}{%
    \lowercase{\def\tempa{#2}}%
    \xdef\tempa{\tempa@\noexpand\textbackslash#2}%
  }{\def\tempa{#2}}%
  \IfBooleanTF{#3}{%
    \lowercase{\def\tempb{#4}}%
    \xdef\tempb{\tempb@\noexpand\textbackslash#4}%
  }{\def\tempb{#4}}%
  \expandafter\index\expandafter{\tempa!\tempb}%
  \expandafter\index\expandafter{\tempb}%
}
\newcommand\macroIndex[1]{%
  \lowercase{\def\temp{#1}}%
  \expandafter\index\expandafter{\temp@\textbackslash#1}%
}
\newcommand\gobbleone[1]{}% https://tex.stackexchange.com/questions/318472
\newcommand{\See}[2]{\unskip\emph{see } #1}
\newcommand\SeeIndex[2]{\index{#1!zzzz@\protect\gobbleone|See{#2}}}

\newif\ifCtan\Ctantrue % condition compilation for ctan distribution

\renewcommand*\contentsname{\relax}

% hyperref links to ctan
\newcommand\TeXLive{\ctan[/]{\TeX Live}\xspace}
\newcommand\Ctan{\ctan[]{ctan}\xspace}
\newcommand\ddash{\texttt{\textemdash\textemdash}}
\NewDocumentCommand\webquizrc{s}{\index{webquizrc}%
\BashCode|webquizrc| file\IfBooleanF{#1}{ (\autoref{SS:rcfile})}\xspace%
}

\NewDocumentCommand\OnlineManual{s}{%
  \href{http://www.maths.usyd.edu.au/u/mathas/WebQuiz/webquiz-online-manual.html}{Online manual}%
  \IfBooleanF{#1}{ (\hyperref[S:online]{Appendix B})}\xspace%
}

\indexprologue{%
  \noindent\textit{This is an index only for the main \WebQuiz manual.
  It does not index the \OnlineManual.}}

%%%%%%%%%%%%%%%%%%%%%%%%%%%%%%%%%%%%%%%%%%%%%%%%%%%%%%%%%%%%%%%%%%%%%%%%%%%%%%%%%%%%%%%
% -----------------------------------------------------------------------
%   webquiz.cfg | webquiz TeX4ht configuration file
% -----------------------------------------------------------------------
%
%   Copyright (C) Andrew Mathas and Don Taylor, University of Sydney
%
%   Distributed under the terms of the GNU General Public License (GPL)
%               http://www.gnu.org/licenses/
%
%   This file is part of the WebQuiz system.
%
%   <Andrew.Mathas@sydney.edu.au>
% ----------------------------------------------------------------------

\NeedsTeXFormat{LaTeX2e}[1996/12/01]
\ProvidesFile{webquiz.cfg}[\webquiz{release date} v\webquiz{version} WebQuiz tex4ht configuration]

\makeatletter% quite surprisingly, this is needed!

% shortcut for inserting newlines into xml ifle
\newcommand\HNewLine{\HCode{\Hnewline}}

% Generate HTML5 + MathML code
\Preamble{xhtml,mathml,html5,NoFonts,charset="utf-8",ext=xml,svg}

% Don't output xml version tag
\Configure{VERSION}{}

% Output a WebQuiz doctype instead of the default for HTML5
\Configure{DOCTYPE}{\HCode{<?xml version="1.0" encoding="UTF-8" ?>\Hnewline
<!DOCTYPE webquiz>}}

\ifWQ@Debugging\def\WQ@debugging{true}\else\def\WQ@debugging{false}\fi

% store the quiz specifications as attributes to <quiz>
\Configure{HTML}{%
  \HNewLine%
  \Tg<webquiz debugging="\WQ@debugging"
                 hide_side_menu="\WQ@ClassOption{hidesidemenu value}"
                 language="\WQ@ClassOption{language}"
                 one_page="\WQ@ClassOption{onepage value}"
                 pst2pdf="\WQ@ClassOption{pst2pdf value}"
                 random_order="\WQ@ClassOption{randomorder value}"
                 src="\jobname.tex"
                 time_limit="\WQ@ClassOption{timelimit}"
                 theme="\WQ@ClassOption{theme}">
  \HNewLine%
  \Tg<title>\@title\Tg</title>%
  \HNewLine%
  \Tg<breadcrumb breadcrumbs="\WQ@breadcrumbs">\WQ@breadcrumb\Tg</breadcrumb>%
  \HNewLine%
  \Tg<unit_name url="\uos@url" quizzes_url="\WQ@quizzesUrl">\uos@name\Tg</unit_name>%
  \HNewLine%
  \Tg<unit_code>\uos@code\Tg</unit_code>%
  \HNewLine%
  \Tg<department url="\WQ@departmentURL">\WQ@department\Tg</department>%
  \HNewLine%
  \Tg<institution url="\WQ@institutionURL">\WQ@institution\Tg</institution>%
  \HNewLine%
}{\Tg</webquiz>}

% reset all configurations
\Configure{HEAD}{}{}
\Configure{BODY}{}{}
\Configure{TITLE}{}{}
\Configure{TITLE+}{}
\Configure{thanks author date and}{}{}{}{}{}{}{}{}
\renewcommand{\maketitle}{}

% convert pictures to svg
\Configure{Picture}{.svg}
%\Configure{Picture*}{.svg}

%\Configure{textit}{\HCode{<span class="textit">}\NoFonts}{\EndNoFonts\HCode{</span>}}
%\Configure{textsf}{\HCode{<span class="textsf">}\NoFonts}{\EndNoFonts\HCode{</span>}} 

% The webquiz macros and environments need to write opening and closing tags to
% the xml file. Opening these tags is easy. To close them we use \WQ@closeXXX
% and \WQ@@closeXXX macros. Each time a tag is opened we \let the \WQ@closeTag
% macro to the corresponding \WQ@@close macro. The \WQ@@close macro closes the
% tag and then \let's the \WQ@closeTag macro to \relax. This allows us to use
% the \WQ@close macros at the start of the tag routines below, which results in
% fairly transparent and readable code.

% make the @close tags \relax initially
\let\WQ@closeTag\relax % closes \correct, \incorrect, \feedback, \whenRight and \whenWrong
\let\WQ@closeText\relax   % closes text

% text tags - everything in wrapped inside <text>...</text> tags
\def\WQ@openText{\Tg<text>\HtmlParOff\HCode{<![CDATA[}\let\WQ@closeText\WQ@@closeText}
\def\WQ@@closeText{\HCode{]]>}\HtmlParOn\Tg</text>\HNewLine\global\let\WQ@closeText\relax}

% -----------------------------------------------------------------------
% discussion environment
% -----------------------------------------------------------------------

% discussion environments are number -1, -2, ... in the webquiz.js
\renewcommand\thediscussion{-\arabic{discussion}}

% inside a discussion environment any label defines two labels, the second of 
% which is a label to the short-title for the discussion item. Of there are many
% labels this is slightly in efficient but, because of tex4ht magic, it proved 
% too painful to extract the discussion number from the label.
\let\WQ@reallabel=\label
\newcommand\WQ@discussionLabel[1]{%
  \WQ@reallabel{#1}%
  \bgroup\let\@currentlabel\WQ@shortDiscussionTitle%
  \WQ@reallabel{discussion-title-#1}%
  \egroup%
}

% discussion environments
\RenewDocumentEnvironment{discussion}{O{Discussion}O{#1}}{%
    \WQ@DiscussionStart%
    \HNewLine%
    \Tg<discussion>\HNewLine
    \Tg<short_heading>#1\Tg</short_heading>\HNewLine
    \Tg<heading>#2\Tg</heading>\HNewLine
    \Tg<text>\HCode{<![CDATA[}% don't use \WQ@openText as we want paragraphs
    \def\WQ@shortDiscussionTitle{#1}%
    \let\label\WQ@discussionLabel%
}{\EndP\HCode{]]>}\Tg</text>\HNewLine\Tg</discussion>\HNewLine\WQ@DiscussionEnd}

% ---------------------------------------------------------------------------
% Cross referencing question and discussion environments
% ---------------------------------------------------------------------------

% tex4ht redefines the \ref internals so that r@<label> uses \rEfLiNK,
% so we temporarily redefine \rEfLiNK to access the question/discussion
% number
\def\WQ@rEfLiNK#1#2{#2}
% \WQ@Ref[optional *][optional text]{mandatory label}
\RenewDocumentCommand\WQ@ref{ m s o m }{%
  \HNewLine%
  \bgroup%
  \let\rEfLiNK\WQ@rEfLiNK%
  \IfBooleanTF{#2}{% link
    \HCode{<a onClick="gotoQuestion(}\ref{#4}\HCode{);">}%
    \IfNoValueTF{#3}{\ref{#1#4}}{#3}%
    \HCode{</a>}%
  }{% button
    \HCode{<span class="button blank" onClick="gotoQuestion(}\ref{#4}%
    \HCode{);">}%
    \IfNoValueTF{#3}{\ref{#1#4}}{#3}%
    \HCode{</span>}%
  }%
  \egroup%
}

\renewcommand\dref{\WQ@ref{discussion-title-}}
\renewcommand\qref{\WQ@ref{}}
\RenewDocumentCommand\Qref{ s O{#3} m }{%
  \HNewLine%
  \IfBooleanTF{#1}{% link
    \HCode{<a onClick="gotoQuestion(#3);">}#2\HCode{</a>}%
  }{% button
    \HCode{<span class="button blank" onClick="gotoQuestion(#3);">}#2\HCode{</span>}%
  }%
}

% -----------------------------------------------------------------------
% question environments
% -----------------------------------------------------------------------

\renewenvironment{question}{%
  \WQ@QuestionStart%
  \IgnorePar\HCode{\Hnewline<question>}% open the question tag
  \WQ@openText% start some text
}{\WQ@QuestionEnd\WQ@closeText\WQ@closeTag\Tg</question>\HNewLine}

% -----------------------------------------------------------------------
% choice environment, \correct and \incorrect choices and \feedback
% -----------------------------------------------------------------------

\RenewDocumentEnvironment{choice}{O{}}{%
  \WQ@ChoiceStart{#1}%
  \WQ@closeText% close any open text tags and then insert tex4ht code
  \IgnorePar%
  \Tg<choice type="\pgfkeysvalueof{/webquiz checker/mode}"
             columns="\pgfkeysvalueof{/webquiz checker/columns value}">%
  \let\WQ@closeTag\relax%
}{\WQ@ChoiceEnd\WQ@closeTag\Tg</choice>}

% correct and incorrect items
\def\WQ@@closeItem{\WQ@closeText\Tg</item>\HNewLine\let\WQ@closeTag\relax}
\def\WQ@Item#1{%
  \WQ@ItemStart%
  \WQ@closeTag%
  \Tg<item correct="#1" symbol="\thechoice">\WQ@openText%
  \let\WQ@closeTag\WQ@@closeItem%
}
\def\correct{\WQ@Item{true}}
\def\incorrect{\WQ@Item{false}}

% feedback
\def\WQ@@closeFeedback{\WQ@closeText\Tg</feedback>\HNewLine\let\WQ@closeTag\relax}
\def\feedback{%
  \WQ@FeedbackStart%
  \WQ@closeTag\Tg<feedback>\WQ@openText\let\WQ@closeTag\WQ@@closeFeedback%
}

% -----------------------------------------------------------------------
% The \answer macro and \whenRight, \whenWrong
% -----------------------------------------------------------------------

\RenewDocumentCommand\answer {sO{string}m}{%
  \WQ@AnswerStart{#2}%
  \WQ@closeText%
  \IfBooleanTF{#1}{\def\WQ@prompt{false}}{\def\WQ@prompt{true}}%
  \Tg<answer prompt="\WQ@prompt" comparison="#2">\HNewLine%
  \WQ@openText#3\WQ@closeText%
  \Tg</answer>%
  \HNewLine%
  \let\WQ@closeTag\relax%
  \WQ@openText%
}

% whenRight and whenWrong
\def\WQ@closeWhen{\Tg</when>\HNewLine\let\WQ@closeTag\relax}
\def\WQ@When#1{%
  \WQ@WhenStart{#1}%
  \WQ@closeText\WQ@closeTag%
  \IgnorePar\Tg<when type="#1">\WQ@openText%
  \let\WQ@closeTag\WQ@closeWhen%
}
\def\whenRight{\WQ@When{right}}
\def\whenWrong{\WQ@When{wrong}}

% -----------------------------------------------------------------------
% the quizindex environment and \quiz macro
% -----------------------------------------------------------------------

% quizindex environment and \quiz
\ConfigureEnv{quizindex}{\Tg<quizindex>\HNewLine}{\Tg</quizindex>\HNewLine}{}{}
\RenewDocumentCommand\quiz{som}{\refstepcounter{quiz}%
  \WQ@QuizStart%
  \IfBooleanTF{#1}{\def\WQ@prompt{false}}{\def\WQ@prompt{true}}%
  \IfNoValueTF{#2}{\def\WQ@url{quiz\thequiz.html}}{\def\WQ@url{#2}}%
  \Tg<index_item prompt="\WQ@prompt" url="\WQ@url">\WQ@openText #3 \WQ@closeText\Tg</index_item>\HNewLine%
}

% -----------------------------------------------------------------------
% making listing environment work with colour I
% - this code appears to break \color and \textcolor in the online manual
% -----------------------------------------------------------------------


% from https://tex.stackexchange.com/questions/225554/syntax-highlighting-in-an-html-presentation
% extract current color as hexadecimal value
\makeatletter
\ifdefined\lst@version
\newcommand\tsf@getColor[1][.]{%
      % colorname `.` holds current color
      \extractcolorspec{.}{\tsf@color}
      \expandafter\convertcolorspec\tsf@color{HTML}\tsf@color
      %\tmpcolor
  }

  % write css color for given css selector
  \newcommand\CssColor[1]{%on-l
      % save current color
      \tsf@getColor%
      \Css{#1{color:\#\tsf@color;}}%
  }
\fi

% Do not set indent and noindent classes on paragraphs
\Configure{HtmlPar}
    {\EndP\Tg<p>}
    {\EndP\Tg<p>}
    {\HCode{</p>\Hnewline}}
    {\HCode{</p>\Hnewline}}

\renewcommand\DisplayAsImage[2][]{%
  \csletcs{real:#2}{#2}%
  \NewConfigure{#2}{2}
  \csdef{#2}##1{\Picture+[#1]{}\csuse{real:#2}{##1}\EndPicture}
  \Configure{#2}{\Picture+[#1]{}}{\EndPicture}
}

\begin{document}

% disable \title after \begin{document}
\def\title{\WQ@Error{\@backslashchar title can only be used in the preamble}\@ehc}

% -----------------------------------------------------------------------
% making listing environment work with colour II
% - this code appears to break \color and \textcolor in the online manual
% -----------------------------------------------------------------------

%% this code is for processing the listing environments in the online manual
\ifdefined\lst@version
  % from https://www.mail-archive.com/tex4ht@tug.org/msg00116.html
  \let\savecolor\color
  \NewConfigure{color}[2]{\def\a@color{#1}\def\b@color{#2}}
  \def\color@@tmp#1{\a@color#1\b@color\savecolor{#1}\aftergroup\endspan}
  \let\color\color@@tmp
  \def\endspan{\Tg</span>}
  \Configure{color}{\HCode{<span style="color:}}{\HCode{">}}

  \ConfigureEnv{lstlisting}
     {\let\color\savecolor
      \ifvmode \IgnorePar\fi \EndP
      \gHAdvance\listingN by 1
      \HCode{<!--l. \the\inputlineno-->}%
      \gdef\start:LstLn{\HCode{<div class="lstlisting" id="listing-\listingN">}%
        \gdef\start:LstLn{\leavevmode\special{t4ht@+\string&{35}x00A0{59}}x%
      \HCode{<br/>}}}
      \bgroup
         \Configure{listings}
           {{\everypar{}\leavevmode}}
           {{\everypar{}\leavevmode}}
           {\start:LstLn \HCode{<span class="label">}}
           {\HCode{</span>}}%
     }
     {\egroup
      \ifvmode \IgnorePar\fi \EndP \HCode{</div>}\par}
     {}{}

  % for listings in webquiz-manual
  % from https://tex.stackexchange.com/questions/225554/syntax-highlighting-in-an-html-presentation
  \newcommand\LstCss[2]{%
      \bgroup%
          \csname lst@#2\endcsname%
          \CssColor{#1}%
      \egroup%
  }

  \LstCss{div.lstlisting .ecbx-1000}{keywordstyle}
  \LstCss{div.lstlisting .ecss-1000}{commentstyle}
  \LstCss{div.lstlisting .ectt-1000}{basicstyle}
\fi

\makeatother

\EndPreamble

% bug fix for \mathbf from http://tex.stackexchange.com/questions/362178
%\Configure{mathbf}{\HCode{<mi mathvariant="bold">}\PauseMathClass}{\EndPauseMathClass\HCode{</mi>}}

\endinput
%%
%% End of file `webquiz.cfg'.
 % defines a comma separated list of themes \WebQuizThemes
\newcommand\ListWebQuizThemes{%
  \begin{quote}
    \def\sep{}
    \foreach \theme in \WebQuizThemes {%
    \ifx\theme\empty\gdef\sep{ and }%
        \else\sep\index{theme!\theme}\theme\gdef\sep{, }%
        \fi
    }
  \end{quote}
}
\newcommand\ShowcaseThemes{%
  \foreach \theme in \WebQuizThemes {
    \ifx\theme\empty\relax%
    \else%
       \ScreenShot[0.75]{Example of the \textbf{\theme} theme}{theme-\theme}%
       \index{theme!\theme}%
    \fi
  }
}

% -----------------------------------------------------------------------
%   webquiz.cfg | webquiz TeX4ht configuration file
% -----------------------------------------------------------------------
%
%   Copyright (C) Andrew Mathas and Don Taylor, University of Sydney
%
%   Distributed under the terms of the GNU General Public License (GPL)
%               http://www.gnu.org/licenses/
%
%   This file is part of the WebQuiz system.
%
%   <Andrew.Mathas@sydney.edu.au>
% ----------------------------------------------------------------------

\NeedsTeXFormat{LaTeX2e}[1996/12/01]
\ProvidesFile{webquiz.cfg}[\webquiz{release date} v\webquiz{version} WebQuiz tex4ht configuration]

\makeatletter% quite surprisingly, this is needed!

% shortcut for inserting newlines into xml ifle
\newcommand\HNewLine{\HCode{\Hnewline}}

% Generate HTML5 + MathML code
\Preamble{xhtml,mathml,html5,NoFonts,charset="utf-8",ext=xml,svg}

% Don't output xml version tag
\Configure{VERSION}{}

% Output a WebQuiz doctype instead of the default for HTML5
\Configure{DOCTYPE}{\HCode{<?xml version="1.0" encoding="UTF-8" ?>\Hnewline
<!DOCTYPE webquiz>}}

\ifWQ@Debugging\def\WQ@debugging{true}\else\def\WQ@debugging{false}\fi

% store the quiz specifications as attributes to <quiz>
\Configure{HTML}{%
  \HNewLine%
  \Tg<webquiz debugging="\WQ@debugging"
                 hide_side_menu="\WQ@ClassOption{hidesidemenu value}"
                 language="\WQ@ClassOption{language}"
                 one_page="\WQ@ClassOption{onepage value}"
                 pst2pdf="\WQ@ClassOption{pst2pdf value}"
                 random_order="\WQ@ClassOption{randomorder value}"
                 src="\jobname.tex"
                 time_limit="\WQ@ClassOption{timelimit}"
                 theme="\WQ@ClassOption{theme}">
  \HNewLine%
  \Tg<title>\@title\Tg</title>%
  \HNewLine%
  \Tg<breadcrumb breadcrumbs="\WQ@breadcrumbs">\WQ@breadcrumb\Tg</breadcrumb>%
  \HNewLine%
  \Tg<unit_name url="\uos@url" quizzes_url="\WQ@quizzesUrl">\uos@name\Tg</unit_name>%
  \HNewLine%
  \Tg<unit_code>\uos@code\Tg</unit_code>%
  \HNewLine%
  \Tg<department url="\WQ@departmentURL">\WQ@department\Tg</department>%
  \HNewLine%
  \Tg<institution url="\WQ@institutionURL">\WQ@institution\Tg</institution>%
  \HNewLine%
}{\Tg</webquiz>}

% reset all configurations
\Configure{HEAD}{}{}
\Configure{BODY}{}{}
\Configure{TITLE}{}{}
\Configure{TITLE+}{}
\Configure{thanks author date and}{}{}{}{}{}{}{}{}
\renewcommand{\maketitle}{}

% convert pictures to svg
\Configure{Picture}{.svg}
%\Configure{Picture*}{.svg}

%\Configure{textit}{\HCode{<span class="textit">}\NoFonts}{\EndNoFonts\HCode{</span>}}
%\Configure{textsf}{\HCode{<span class="textsf">}\NoFonts}{\EndNoFonts\HCode{</span>}} 

% The webquiz macros and environments need to write opening and closing tags to
% the xml file. Opening these tags is easy. To close them we use \WQ@closeXXX
% and \WQ@@closeXXX macros. Each time a tag is opened we \let the \WQ@closeTag
% macro to the corresponding \WQ@@close macro. The \WQ@@close macro closes the
% tag and then \let's the \WQ@closeTag macro to \relax. This allows us to use
% the \WQ@close macros at the start of the tag routines below, which results in
% fairly transparent and readable code.

% make the @close tags \relax initially
\let\WQ@closeTag\relax % closes \correct, \incorrect, \feedback, \whenRight and \whenWrong
\let\WQ@closeText\relax   % closes text

% text tags - everything in wrapped inside <text>...</text> tags
\def\WQ@openText{\Tg<text>\HtmlParOff\HCode{<![CDATA[}\let\WQ@closeText\WQ@@closeText}
\def\WQ@@closeText{\HCode{]]>}\HtmlParOn\Tg</text>\HNewLine\global\let\WQ@closeText\relax}

% -----------------------------------------------------------------------
% discussion environment
% -----------------------------------------------------------------------

% discussion environments are number -1, -2, ... in the webquiz.js
\renewcommand\thediscussion{-\arabic{discussion}}

% inside a discussion environment any label defines two labels, the second of 
% which is a label to the short-title for the discussion item. Of there are many
% labels this is slightly in efficient but, because of tex4ht magic, it proved 
% too painful to extract the discussion number from the label.
\let\WQ@reallabel=\label
\newcommand\WQ@discussionLabel[1]{%
  \WQ@reallabel{#1}%
  \bgroup\let\@currentlabel\WQ@shortDiscussionTitle%
  \WQ@reallabel{discussion-title-#1}%
  \egroup%
}

% discussion environments
\RenewDocumentEnvironment{discussion}{O{Discussion}O{#1}}{%
    \WQ@DiscussionStart%
    \HNewLine%
    \Tg<discussion>\HNewLine
    \Tg<short_heading>#1\Tg</short_heading>\HNewLine
    \Tg<heading>#2\Tg</heading>\HNewLine
    \Tg<text>\HCode{<![CDATA[}% don't use \WQ@openText as we want paragraphs
    \def\WQ@shortDiscussionTitle{#1}%
    \let\label\WQ@discussionLabel%
}{\EndP\HCode{]]>}\Tg</text>\HNewLine\Tg</discussion>\HNewLine\WQ@DiscussionEnd}

% ---------------------------------------------------------------------------
% Cross referencing question and discussion environments
% ---------------------------------------------------------------------------

% tex4ht redefines the \ref internals so that r@<label> uses \rEfLiNK,
% so we temporarily redefine \rEfLiNK to access the question/discussion
% number
\def\WQ@rEfLiNK#1#2{#2}
% \WQ@Ref[optional *][optional text]{mandatory label}
\RenewDocumentCommand\WQ@ref{ m s o m }{%
  \HNewLine%
  \bgroup%
  \let\rEfLiNK\WQ@rEfLiNK%
  \IfBooleanTF{#2}{% link
    \HCode{<a onClick="gotoQuestion(}\ref{#4}\HCode{);">}%
    \IfNoValueTF{#3}{\ref{#1#4}}{#3}%
    \HCode{</a>}%
  }{% button
    \HCode{<span class="button blank" onClick="gotoQuestion(}\ref{#4}%
    \HCode{);">}%
    \IfNoValueTF{#3}{\ref{#1#4}}{#3}%
    \HCode{</span>}%
  }%
  \egroup%
}

\renewcommand\dref{\WQ@ref{discussion-title-}}
\renewcommand\qref{\WQ@ref{}}
\RenewDocumentCommand\Qref{ s O{#3} m }{%
  \HNewLine%
  \IfBooleanTF{#1}{% link
    \HCode{<a onClick="gotoQuestion(#3);">}#2\HCode{</a>}%
  }{% button
    \HCode{<span class="button blank" onClick="gotoQuestion(#3);">}#2\HCode{</span>}%
  }%
}

% -----------------------------------------------------------------------
% question environments
% -----------------------------------------------------------------------

\renewenvironment{question}{%
  \WQ@QuestionStart%
  \IgnorePar\HCode{\Hnewline<question>}% open the question tag
  \WQ@openText% start some text
}{\WQ@QuestionEnd\WQ@closeText\WQ@closeTag\Tg</question>\HNewLine}

% -----------------------------------------------------------------------
% choice environment, \correct and \incorrect choices and \feedback
% -----------------------------------------------------------------------

\RenewDocumentEnvironment{choice}{O{}}{%
  \WQ@ChoiceStart{#1}%
  \WQ@closeText% close any open text tags and then insert tex4ht code
  \IgnorePar%
  \Tg<choice type="\pgfkeysvalueof{/webquiz checker/mode}"
             columns="\pgfkeysvalueof{/webquiz checker/columns value}">%
  \let\WQ@closeTag\relax%
}{\WQ@ChoiceEnd\WQ@closeTag\Tg</choice>}

% correct and incorrect items
\def\WQ@@closeItem{\WQ@closeText\Tg</item>\HNewLine\let\WQ@closeTag\relax}
\def\WQ@Item#1{%
  \WQ@ItemStart%
  \WQ@closeTag%
  \Tg<item correct="#1" symbol="\thechoice">\WQ@openText%
  \let\WQ@closeTag\WQ@@closeItem%
}
\def\correct{\WQ@Item{true}}
\def\incorrect{\WQ@Item{false}}

% feedback
\def\WQ@@closeFeedback{\WQ@closeText\Tg</feedback>\HNewLine\let\WQ@closeTag\relax}
\def\feedback{%
  \WQ@FeedbackStart%
  \WQ@closeTag\Tg<feedback>\WQ@openText\let\WQ@closeTag\WQ@@closeFeedback%
}

% -----------------------------------------------------------------------
% The \answer macro and \whenRight, \whenWrong
% -----------------------------------------------------------------------

\RenewDocumentCommand\answer {sO{string}m}{%
  \WQ@AnswerStart{#2}%
  \WQ@closeText%
  \IfBooleanTF{#1}{\def\WQ@prompt{false}}{\def\WQ@prompt{true}}%
  \Tg<answer prompt="\WQ@prompt" comparison="#2">\HNewLine%
  \WQ@openText#3\WQ@closeText%
  \Tg</answer>%
  \HNewLine%
  \let\WQ@closeTag\relax%
  \WQ@openText%
}

% whenRight and whenWrong
\def\WQ@closeWhen{\Tg</when>\HNewLine\let\WQ@closeTag\relax}
\def\WQ@When#1{%
  \WQ@WhenStart{#1}%
  \WQ@closeText\WQ@closeTag%
  \IgnorePar\Tg<when type="#1">\WQ@openText%
  \let\WQ@closeTag\WQ@closeWhen%
}
\def\whenRight{\WQ@When{right}}
\def\whenWrong{\WQ@When{wrong}}

% -----------------------------------------------------------------------
% the quizindex environment and \quiz macro
% -----------------------------------------------------------------------

% quizindex environment and \quiz
\ConfigureEnv{quizindex}{\Tg<quizindex>\HNewLine}{\Tg</quizindex>\HNewLine}{}{}
\RenewDocumentCommand\quiz{som}{\refstepcounter{quiz}%
  \WQ@QuizStart%
  \IfBooleanTF{#1}{\def\WQ@prompt{false}}{\def\WQ@prompt{true}}%
  \IfNoValueTF{#2}{\def\WQ@url{quiz\thequiz.html}}{\def\WQ@url{#2}}%
  \Tg<index_item prompt="\WQ@prompt" url="\WQ@url">\WQ@openText #3 \WQ@closeText\Tg</index_item>\HNewLine%
}

% -----------------------------------------------------------------------
% making listing environment work with colour I
% - this code appears to break \color and \textcolor in the online manual
% -----------------------------------------------------------------------


% from https://tex.stackexchange.com/questions/225554/syntax-highlighting-in-an-html-presentation
% extract current color as hexadecimal value
\makeatletter
\ifdefined\lst@version
\newcommand\tsf@getColor[1][.]{%
      % colorname `.` holds current color
      \extractcolorspec{.}{\tsf@color}
      \expandafter\convertcolorspec\tsf@color{HTML}\tsf@color
      %\tmpcolor
  }

  % write css color for given css selector
  \newcommand\CssColor[1]{%on-l
      % save current color
      \tsf@getColor%
      \Css{#1{color:\#\tsf@color;}}%
  }
\fi

% Do not set indent and noindent classes on paragraphs
\Configure{HtmlPar}
    {\EndP\Tg<p>}
    {\EndP\Tg<p>}
    {\HCode{</p>\Hnewline}}
    {\HCode{</p>\Hnewline}}

\renewcommand\DisplayAsImage[2][]{%
  \csletcs{real:#2}{#2}%
  \NewConfigure{#2}{2}
  \csdef{#2}##1{\Picture+[#1]{}\csuse{real:#2}{##1}\EndPicture}
  \Configure{#2}{\Picture+[#1]{}}{\EndPicture}
}

\begin{document}

% disable \title after \begin{document}
\def\title{\WQ@Error{\@backslashchar title can only be used in the preamble}\@ehc}

% -----------------------------------------------------------------------
% making listing environment work with colour II
% - this code appears to break \color and \textcolor in the online manual
% -----------------------------------------------------------------------

%% this code is for processing the listing environments in the online manual
\ifdefined\lst@version
  % from https://www.mail-archive.com/tex4ht@tug.org/msg00116.html
  \let\savecolor\color
  \NewConfigure{color}[2]{\def\a@color{#1}\def\b@color{#2}}
  \def\color@@tmp#1{\a@color#1\b@color\savecolor{#1}\aftergroup\endspan}
  \let\color\color@@tmp
  \def\endspan{\Tg</span>}
  \Configure{color}{\HCode{<span style="color:}}{\HCode{">}}

  \ConfigureEnv{lstlisting}
     {\let\color\savecolor
      \ifvmode \IgnorePar\fi \EndP
      \gHAdvance\listingN by 1
      \HCode{<!--l. \the\inputlineno-->}%
      \gdef\start:LstLn{\HCode{<div class="lstlisting" id="listing-\listingN">}%
        \gdef\start:LstLn{\leavevmode\special{t4ht@+\string&{35}x00A0{59}}x%
      \HCode{<br/>}}}
      \bgroup
         \Configure{listings}
           {{\everypar{}\leavevmode}}
           {{\everypar{}\leavevmode}}
           {\start:LstLn \HCode{<span class="label">}}
           {\HCode{</span>}}%
     }
     {\egroup
      \ifvmode \IgnorePar\fi \EndP \HCode{</div>}\par}
     {}{}

  % for listings in webquiz-manual
  % from https://tex.stackexchange.com/questions/225554/syntax-highlighting-in-an-html-presentation
  \newcommand\LstCss[2]{%
      \bgroup%
          \csname lst@#2\endcsname%
          \CssColor{#1}%
      \egroup%
  }

  \LstCss{div.lstlisting .ecbx-1000}{keywordstyle}
  \LstCss{div.lstlisting .ecss-1000}{commentstyle}
  \LstCss{div.lstlisting .ectt-1000}{basicstyle}
\fi

\makeatother

\EndPreamble

% bug fix for \mathbf from http://tex.stackexchange.com/questions/362178
%\Configure{mathbf}{\HCode{<mi mathvariant="bold">}\PauseMathClass}{\EndPauseMathClass\HCode{</mi>}}

\endinput
%%
%% End of file `webquiz.cfg'.
 % defines a comma separated list of themes \WebQuizLanguages
\newcommand\ListWebQuizLanguages{%
  \begin{quote}
    \def\sep{}
    \foreach \lang in \WebQuizLanguages {%
    \ifx\lang\empty\gdef\sep{ and }%
        \else\sep\index{language!\lang}\lang\gdef\sep{, }%
        \fi
    }
  \end{quote}
}

%%%%%%%%%%%%%%%%%%%%%%%%%%%%%%%%%%%%%%%%%%%%%%%%%%%%%%%%%%%%%%%%%%%%%%%%%%%%%%%%%%%%%%%
%% WebQuiz title box for front page
\usepackage{tikz}
\usetikzlibrary{shadows.blur}

\definecolor{stone}{HTML}{E9E0D8}
\tikzset{shadowed/.style={blur shadow={shadow blur steps=5},
                          top color=stone,
                          bottom color=PapayaWhip,
                          draw=SaddleBrown,
                          shade,
                          font=\normalfont\Huge\bfseries\scshape,
                          rounded corners=8pt,
      },
      boxes/.style={draw=Sienna,
                    fill=Cornsilk,
                    font=\sffamily\small,
                    inner sep=5pt,
                    rectangle,
                    rounded corners=8pt,
                    text=Brown,
     }
}

\def\WebQuizTitle{
  \begin{tikzpicture}[remember picture,overlay]
      \node[yshift=-3cm] at (current page.north west)
        {\begin{tikzpicture}[remember picture, overlay]
          \draw[shadowed](30mm,0) rectangle node[Brown]{\WebQuiz} (\paperwidth-30mm,16mm);
          \node[Sienna,font=\normalfont\small\itshape] at (\paperwidth/2,2mm)
          {\small \webquiz{description}};
          \node[anchor=west,boxes] at (4cm,0cm) {\webquiz{name}};
          \node[anchor=east,boxes] at (\paperwidth-4cm,0) {Version \webquiz{version}};
         \end{tikzpicture}
        };
   \end{tikzpicture}
}


%%%%%%%%%%%%%%%%%%%%%%%%%%%%%%%%%%%%%%%%%%%%%%%%%%%%%%%%%%%%%%%%%%%%%%%%%%%%%%%%%%%%%%%

%% headers and footers
\makeatletter
\def\ps@webquiz{
  \ps@empty
  \def\@oddfoot{\tiny\WebQuiz\space -- version \webquiz{version}\hfill%
     \textsc{\ifodd\thepage The \WebQuiz manual\else \webquizheader\fi}\hfill\thepage}
}
% hijack section and subsectionmark for our headers
\def\webquizheader{\WebQuiz}
\def\sectionmark#1{\def\webquizheader{#1}}
\def\subsectionmark#1{\def\webquizheader{#1}}
\pagestyle{webquiz}
\patchcmd\l@section{1.0em}{0.5em}{}{}
\makeatother

\begin{document}
  \hypersetup{pdftitle={WebQuiz manual}}

      \WebQuizTitle

      \begin{quote}
        \WebQuiz is a \LaTeX{} package for writing online quizzes. It allows
        the quiz author to concentrate on the content of quizzes, written
        in standard \LaTeX, unencumbered by the technicalities of \HTML and
        \Javascript. Online quizzes written using \WebQuiz can contain any
        material that can be written using \LaTeX, including text,
        mathematics, graphics and diagrams.
        \ScreenShot[0.68]{An example \WebQuiz web page}{quiz-page}
        \begin{center}
          \hfil
          \begin{minipage}{0.75\textwidth}
            \tableofcontents
          \end{minipage}
        \end{center}
      \end{quote}

      \newpage

  \section{Introduction}
      Online quizzes provide a good way to reinforce learning, especially
      because they can give ``interactive'' feedback to the
      students\footnote{Throughout this manual, ``student'' means any
      person taking the online quiz.\index{student}} based on the answers that they
      give. Unfortunately, in addition to writing the actual quiz content there
      are significant technical hurdles that need to be overcome when
      writing an online quiz -- and there are additional complications if
      the quiz involves mathematics or diagrams.

      \WebQuiz makes it possible to write online quizzes using \LaTeX,
      which is the typesetting language used by mathematicians who use
      \LaTeX\ to write their research papers, books and teaching
      materials. In principle, a \WebQuiz quiz can contain anything that
      can be typeset using \LaTeX.  In practise, the \LaTeX\ is converted
      to \HTML using \TeXfht (and
      \ctan{make4ht}), so the
      quizzes can contain any \LaTeX commands that are understood by \TeX
      4ht, which is almost everything. In particular, it is possible to
      use graphics constructed using \ctan{pstricks} and
      \ctan{tikz}; see \autoref{SS:graphics}.

      \WebQuiz supports the following three types of questions:
      \begin{itemize}
        \item Multiple choice questions with a unique correct answer
        \item Multiple choice questions zero or more correct answers
        \item Questions with an answer that is supplied by the student.
      \end{itemize}
      Each time a student answers a question it is possible to give them
      feedback, reinforcing their learning when they answer correctly and
      giving them further hints when they are wrong. This allows the
      quiz author to give targeted feedback to the student based on their
      answer.

      The online quizzes constructed using \WebQuiz can, in principle,
      contain anything that can be typeset by \LaTeX. In particular, they
      do not need to contain mathematics. In fact, the quizzes do not even
      have to contain ``questions'' as it is possible for a \WebQuiz
      ``quiz'' to contain only \LatexCode|discussion| environments that can
      be used to revise material, or to introduce new material, for the
      students; see \autoref{SS:discussion}.

      This introduction outlines how to use \WebQuiz, however, the
      impatient reader may want to skip ahead directly to the
      \autoref{S:documentclass}, where the \LaTeX\ commands used by
      \WebQuiz are described.

      The easiest way to explain how \WebQuiz works is by example. The
      following \LaTeX\ file defines a quiz with a single multiple choice
      question that has four possible answers, each of which has a
      customised feedback. Giving feedback to the students in each
      question is optional but the capability of being able to give
      students feedback on their answer is one of the main pedagogical
      advantages of online quizzes.

      \InputLatexCode{simple}

      Since this is a \LaTeX\ file it can be processed using
      \BashCode|pdflatex|, or \BashCode|latex|, to produce a readable and
      printable version of the quiz, which can be useful when
      proofreading. With the example above, the PDF version of the quiz looks
      like this:
      \ScreenShot[0.5]{Sample \BashCode|PDF| output}{simple-pdf}
      Of course, the real reason for using \WebQuiz is to create a
      web page for the quiz, which you do by processing the quiz using
      the \BashCode|webquiz| command )instead of, say
      \BashCode|pdflatex|). If you do this and open the resulting
      web page in your favourite browser, after selecting answer~(a), you
      will see a web page like this:
      \ScreenShot{Sample web page}{simple-html}
      The actual page that you see may be slightly different to
      this because the appearance of the web page depends partly on your
      choice of browser.

      By default, the online version of the quiz displays one question at
      a time, with the question buttons serving the dual purpose of
      navigation between questions and displaying how successful the
      student was in answering the question. The decorations on the
      question buttons indicate whether the question has been attempted
      and, if so, whether it was answered correctly or incorrectly on the
      first or subsequent attempts. One of the main points of \WebQuiz is
      that (optional) targeted feedback can be given to the student taking
      the quiz based on their answer.

  \subsection{What \WebQuiz does and does not do}

      \WebQuiz is a tool that makes it possible to write ``interactive''
      online quizzes using \LaTeX{}. To use \WebQuiz you only need
      basic working knowledge of \LaTeX{}. In particular, no familiarity
      of the underlying \CSS, \HTML or \Javascript is required.

      \WebQuiz can be used to ask students a series of ``quiz''
      questions. In addition, your online quiz web pages can contain
      course material using the \WebQuiz \LatexCode|discussion|
      environment; see \autoref{SS:discussion}. You can write \WebQuiz
      quizzes that only contain questions, and no
      \LatexCode|discussion|, quizzes that contain questions and
      \LatexCode|discussion|, and (pseudo) quizzes that contains only
      \LatexCode{discussion} and no quiz questions.

      By default, the online quizzes display one question (or
      \LatexCode|discussion| environment) at a time.  It is also possible
      to display all of the quiz questions on a single web page
      (\autoref{SS:classOptions}). One of the key features of \WebQuiz is
      that you can give feedback to the students based on their answers.
      In this way you can give hints to the students to correct their
      mistakes and you can reinforce the students' understanding when they
      are correct. Each question in a quiz, and each quiz itself, can be
      attempted as many times as the student wants. \WebQuiz does not
      limit the number of times that questions can be attempted.

      As described in \autoref{SS:Questions}, \WebQuiz  supports the following question types:
      \begin{itemize}
        \item Multiple choice questions with a unique correct answer
        \item Multiple choice questions zero or more correct answers
        \item Questions that require students to type in an answer. There
        are several different ``comparison'' methods available for
        comparing the students answer where
        for example, the entered answers can be a ``string'' or a ``number''.
      \end{itemize}

      Questions can appear in either the same order that they appear in
      the \LaTeX{} file for the quiz or in a random order that changes
      each time the quiz page is loaded. For multiple choice questions the
      order in which the choices appear is always the order that they
      appear in the \LaTeX{} file for the quiz, even if the questions
      appear in random order.

      \WebQuiz supports several different languages and it
      provides a number of different colour schemes (see
      \autoref{SS:commandline} and \hyperref[SS:themes]{Appendix A}).

      \WebQuiz quizzes are not timed and they do not have time-limits.

      Quizzes made using \WebQuiz are intended to be used as a
      revision resource rather than as an assessment tool. In particular,
      \WebQuiz does not provide a mechanism for recording the marks
      obtained by the students taking the quiz
      (\autoref{SS:classOptions}). Technically, it probably would not be
      very hard to record marks but this introduces a significant amount
      of extra overhead in terms of student authentication and interfacing
      with a database. In addition, if \WebQuiz were used as an assessment
      tool then there would be additional ``security issues'' to ensure
      that the quiz content is secure. Currently, even though the
      solutions to the quiz questions do not appear in the \HTML source
      code for the quiz pages it is possible to access the answers if you
      know what you are doing.

      The questions in a \WebQuiz quiz are static. In particular,
      \WebQuiz quiz questions do not accept variables.

      The \WebQuiz program was designed to be run from the command-line.
      To process the file \BashCode|quiz.tex| using \WebQuiz you would
      type
      \begin{bashcode}
        > webquiz quiz         or         > webquiz quiz.tex
      \end{bashcode}
      \index{$>$ command-line prompt}\index{command-line prompt!$>$}
      (Throughout this manual, \BashCode|>| is used for the command-line
      prompt.)

      It is possible to use \WebQuiz from inside GUI programs like \TeX
      Shop, but exactly how this is done will depend on the program the
      GUI that you use. In the case of \TeX Shop you need to define a
      new \textit{engine} following, for example the instructions at
      \href{https://tex.stackexchange.com/questions/376649}
           {tex.stackexchange.com/questions/376649}.\index{\TeX Shop}

  \subsection{Credits}
      \WebQuiz{} was written and developed in the
      \href{http://www.maths.usyd.edu.au/}{School of Mathematics and
      Statistics} at the \href{http://www.usyd.edu.au/}{University of
      Sydney}.  The system is built on \LaTeX{} with the conversion from
      \LaTeX{} to \HTML being done by Eitan Gurari's
      \TeXfht and \ctan{make4ht}.

      To write quizzes using \WebQuiz it is only necessary to know
      \LaTeX, however, the underlying \WebQuiz system actually has three
      components:
      \begin{itemize}
        \item A \href{https://www.latex-project.org/}{\LaTeX} document class
        file, \BashCode|webquiz.cls|, and a \TeXfht
        configuration file, \BashCode|webquiz.cfg|, that enables the
        quiz files to be processed by \LaTeX{} and \TeXfht, respectively.
        \item A \python program,
        \BashCode|webquiz|, that translates the
        \LaTeX{} into \XML, using \TeXfht, and then into \HTML.
        \item \CSS, \HTML and \Javascript code controls and style the
        quiz web pages.
      \end{itemize}

     The \LaTeX{} component of \WebQuiz{} was written by Andrew Mathas and
     the \python, \CSS and \Javascript code was written by Andrew Mathas
     based on an initial prototype that was written by Don Taylor in 2001-2.
     Since 2004 the program has been maintained and developed by Andrew
     Mathas. Although the program has changed substantially since 2004,
     Don's idea of using \TeXfht, and some of his code, is still
     in use. Prior to releasing \WebQuiz on \Ctan, the program was known
     as \texttt{MathQuiz}.

     Hendrik Suess contributed code to improve session history
     and suggested the \LatexCode|\qref| command.

     Thanks are due to Bob Howlett for general help with \CSS and to
     Michal Hoftich for invaluable technical advice on \TeXfht. Thanks
     are due to
        Thomas Cailleteau
        Michael Palmer and
        Hendrik Suess,
     for helpful feedback on the package.

   \section{The \WebQuiz document class --- \hologo{LaTeX}{} commands}
   \label{S:documentclass}

    This chapter describes the commands and environments provided by the
    \WebQuiz document class. This assumes that you have already installed
    and configured \WebQuiz. If you have not yet initialised \WebQuiz
    then please follow the instructions in \autoref{S:configuration}.

    All of the code examples given in this and other sections can be
    found in the \BashCode|examples| subdirectory of the \WebQuiz web
    directory.\footnote{After you have initialised \WebQuiz, you can find the
    \WebQuiz example directory from the command-line using:
    \texttt{webquiz --settings webquiz-www}.}
    Additional examples can be found in the \OnlineManual*,
    which is included, in \BashCode|PDF| form, as
    \hyperref[S:online]{Appendix B}.

    Zoomed out, the structure of a typical \WebQuiz quiz file is a \LaTeX{} file of the form:
    \begin{latexcode}
      \documentclass{webquiz}
      \title{A quiz}% optional, but potentially informative, title
      \begin{document}

        \begin{question}% text for first question
        \end{question}

        \begin{question}% text for second question
        \end{question}

        \begin{question}% text for third question
        \end{question}

        ...

      \end{document}
    \end{latexcode}
    You should write your quizzes using the editor that you normally use
    to write \LaTeX{} documents. As you write your quiz, say
    \BashCode|quiz.tex|, you should use \BashCode|pdflatex|
    (or \BashCode|latex|), in the usual way:
    \begin{bashcode}
        > pdflatex quiz
    \end{bashcode}
    This is the easiest way to check that your quiz compiles and to
    proofread the output, just as if you were writing a normal \LaTeX{}
    document. When you are satisfied with the content of the quiz, then
    you can convert the quiz to an online quiz using the command
    \begin{bashcode}
       > webquiz quiz    or    > webquiz -d quiz  # -d = draft mode = faster!
    \end{bashcode}
    The quiz file, \BashCode|quiz.tex|, should be in a directory on your
    web server because \WebQuiz creates a number of different files and
    directories when it converts the file into an online quiz and all
    of these files are needed to display the quiz on the web.

    The reasons for using this workflow are:
    \begin{itemize}
      \item \textit{Every file that you give to \WebQuiz must be a valid
      \LaTeX{} file!}

      \item The \BashCode|dvi| or \BashCode|PDF| file produced by \LaTeX{}
      shows all of the information in the quiz
      \textit{in an easy-to-read format}. That is, the \BashCode|PDF| file
      displays the questions, the answers and the feedback that you are
      giving to the students. In contrast, by design, the online version
      of the quiz hides most of this information and shows it to the
      student only when they need to see it.

      \item Typesetting the quiz file with \LaTeX{} is \textit{much
      faster} than processing it with \WebQuiz. In fact, \WebQuiz uses
      \BashCode|htlatex| to process the quiz file at least
      \textit{three times} in order to produce an \XML file and it is only
      then that the \WebQuiz program kicks in to rewrite this data as an
      \HTML file. (If you use \textit{draft mode} then
      \BashCode|htlatex| processes the quiz file only once.)

      \item If \LaTeX{} produces errors then \WebQuiz will produce more
      errors. Further, \textit{\LaTeX{} error messages are much easier to
      read and understand than those produced by \TeXfht and \WebQuiz}.
    \end{itemize}
    This said, \WebQuiz does check for more errors in the quiz than
    \LaTeX{} is (easily) able to do.

    The \BashCode|PDF| version of a quiz does not contain information about
    the unit, department or institution, which can be used in the
    breadcrumbs.

    The next sections describe the commands and environments provided by
    \WebQuiz for typesetting quizzes as well as the document-class options
    for the package. If you plan to use \ctan{pstricks} or
    \ctan[pgf]{tikz} then you should read \autoref{SS:graphics}, which
    describes how to use graphics in a \WebQuiz quiz.
    \autoref{SS:config} describes a work-around for using (some?) \LaTeX{}
    features that have not been configured for use with \TeXfht.

  \subsection{Question environments}\label{SS:Questions}

  The \WebQuiz document class defines the following four environments:
  \begin{quote}
    \begin{description}
      \item[question] Each quiz question needs to be inside
      \LatexCode|question| environment
      \item[choice] Typesets multiple choice questions, with
      one or more correct answers
      \item[discussion] Includes (optional) discussion, or revision,
      material at the start of the quiz web page
      \item[quizindex] Writes an index file for the quizzes in
      a ``unit of study'' and generates drop-drop menus in each quiz for
      the unit
    \end{description}
  \end{quote}
  This section describes these environments and gives examples
  of their use.

  \subsubsection{Question environments and the \textcolor{blue}{\textbackslash answer} macro}
  \index{environment!question}
  \index{question environment}

  Each quiz question must be placed inside a \LatexCode|question|
  environment. Typically, a quiz has several questions, each wrapped in
  its own \LatexCode|question| environment.  For brevity, most of the
  examples in this chapter have only one question. See the
  \OnlineManual, in the \WebQuiz web directory for a more complete quiz
  file.

  This manual describes the \WebQuiz commands, often by example. The
  following code-block generates a quiz with one question for which the
  student has to enter the answer. This answer is then compared with the
  correct answer as a string, which it must match exactly.
    \InputLatexCode{answer-string}
    \macroIndex{answer}\CrossIndex{question environment}*{answer}
    \CrossIndex*{answer}*{whenRight}
    \CrossIndex*{answer}*{whenWrong}
    \index{feedback!whenRight}
    \index{feedback!whenWrong}
    The \textit{optional} macros \LatexCode|\whenRight| and
    \LatexCode|\whenWrong| are used to give the student additional
    feedback, when they are right, or further hints etc for approaching
    the question, when they are wrong. This feedback is displayed on the
    quiz page only when the student checks their answer.

    The web page created by the code above, when an incorrect answer of
    ``canberra'', instead of ``Canberra'' is given, looks something like
    the following:
    \ScreenShot{A question with an answer}{answer-string}
    This example shows one way of using the \LatexCode|\answer| macro, which
    asks for the student to type in an answer to the question. The general syntax of
    this macro is:
    \begin{latexcode}
        \answer[comparison type]{correct answer}
    \end{latexcode}
    where the optional \textit{comparison type} is one of:
    \begin{latexcode}
        complex    integer    lowercase    number    string
    \end{latexcode}
    with \LatexCode|string| being the default.  In addition, there is a
    $*$-variant of the answer macro that does not print the word
    ``Answer'' (or equivalent in other languages) before the input box.
    The syntax for the \LatexCode|\answer*| command is identical to that
    for the \LatexCode|\answer| except, of course, that there is a~$*$:
    \begin{latexcode}
        \answer*[comparison type]{correct answer}
    \end{latexcode}
    As \LatexCode|string| is the default answer comparison method, if we
    instead use \LatexCode|\answer*{Canberra}| in the last example then
    the quiz page generated by \WebQuiz looks like:
    \ScreenShot{A question with answer$*$}{answer-star}
    Notice that the word \LatexCode|Answer| no longer appears in front of
    the answer box. Of course, if you use the unstarred version of the
    \LatexCode|\answer|-macro together with the document class option
    \LatexCode|language=xxx| to change the default language
    (\autoref{SS:classOptions}), then the appropriate translation of
    \LatexCode|Answer| will appear on the web page.

    We now give a description of the other comparison types for the
    \LatexCode|\answer| and \LatexCode|\answer*| macros together with
    code examples and screenshots.

    \CrossIndex*{answer}{complex}
    \begin{heading}[complex comparison] The answers are compared as complex
    numbers: the answer is marked as correct if it has the same real and
    imaginary parts.
    \InputLatexCode{answer-complex}
    \ScreenShot{A question with a complex answer}{answer-complex}
    \end{heading}
    Observe that the correct answer is given in the quiz file as $7+i$
    and that \WebQuiz accepts $i+7$ as the correct answer.

    \begin{heading}[integer comparison] The answers are compared as
    integers. If the correct answer was $18$ and a student entered
    $36/2$ then their answer would be marked wrong.
    \CrossIndex*{answer}{integer}
    \InputLatexCode{answer-integer}
    \ScreenShot{A question with a integer answer}{answer-integer}
    \end{heading}

    \begin{heading}[lowercase comparison] The quiz answer and the
    students' answer are both converted to lower case and then compared as
    strings.
    \CrossIndex*{answer}{lowercase}
    \InputLatexCode{answer-lowercase}
    \ScreenShot{A question with a lowercase string answer}{answer-lowercase}
    \end{heading}

    \begin{heading}[number comparison] The quiz answer and the students'
      answer are compared as numbers.     \CrossIndex*{answer}{number}
      \InputLatexCode{answer-number}
      \ScreenShot{A question with a numeric answer}{answer-number}
      Notice that the answer is given in the \LaTeX{} file as $3/4$ and
      that the equivalent fraction, $0.75$, is accepted as the correct
      answer.
    \end{heading}

    \begin{heading}[string comparison]
    This is the default, so \LatexCode|\answer{word}| and
    \LatexCode|\answer[string]{word}| are equivalent. The student's
    answer is marked correct if and only if it agrees exactly with
    the quiz answer.
    \CrossIndex*{answer}{string}
    \end{heading}

    \subsubsection{Multiple choice questions}\label{SS:choice}
    \index{environment!choice}
    \index{question environment!choice}
    \SeeIndex{multiple choice}{choice environment}

  The multiple choice options for a quiz question need to be placed inside
  a \LatexCode|choice| environment -- and every \LatexCode|choice|
  environment needs to be inside a \LatexCode|question| environment.

  The \LatexCode|choice| environment accepts two optional arguments,
  can appear in any order:
  \begin{itemize}
    \item\CrossIndex{choice environment}{single}
         \CrossIndex{choice environment}{multiple}
    The word \LatexCode|single| (default, and can be omitted) or
    \LatexCode|multiple|, which indicates whether the quiz has a
    \textit{single} correct answer or whether 0 or more of the answers are
    correct, respectively.
    \item \index{choice environment!columns}
    The number of \textit{columns} in which to typeset the choices. This
    is specified as \LatexCode|columns=n|, where $n$ is a non-negative integer.
    By default, the choices appear in $n=1$ columns
    (\DefaultValue{columns=1}).\\
    \textit{Use \LatexCode|columns=n| with care when $n>1$ as multiple
    columns do not always display well on mobile devices.}
  \end{itemize}
  The key difference between these two types of \LatexCode|choice|
  questions is that a \LatexCode|single|-choice environment uses radio
  boxes, so it is only possible to select \textit{one correct answer}.
  In a \LatexCode|multiple|-choice environment checkboxes are used, so
  that is possible to select \textit{zero or more correct answers}.

  A \LatexCode|choice| environment modelled on the standard
  \LaTeX{} list environments (\textit{enumerate}, \textit{itemize},
  \textit{description}, ...), except that instead of using the
  \LatexCode|\item| command to separate the items the
  \LatexCode|choice| environment uses \LatexCode|\correct|
  \CrossIndex{choice environment}*{correct} \LatexCode|\incorrect|
  \CrossIndex{choice environment}*{incorrect}, which indicate correct
  and incorrect answers respectively.  In addition, after each
  \LatexCode|\correct| or \LatexCode|\incorrect| you can, optionally, use
  \LatexCode|\feedback| \CrossIndex{choice environment}*{feedback} to give
  feedback to the student taking the quiz. Like
  \LatexCode|\whenRight| and \LatexCode|\whenWrong| this feedback is
  displayed only when the student checks their answer.
  \index{feedback}
  \CrossIndex{feedback}*{whenRight}
  \CrossIndex{feedback}*{whenWrong}

  Here is an example of a \LatexCode|single|-choice question with a unique
  answer:
    \InputLatexCode{choice-single}
  Note that \LatexCode|single| is the default, so this could also be written as
  \begin{latexcode}
    \begin{choice}[columns=3, single]
      ...
    \end{choice}
  \end{latexcode}
  It is not necessary to put the \LatexCode|\feedback| lines on the same
  line as the \LatexCode|\incorrect| and \LatexCode|\incorrect|; this is
  done only to make the example more compact. This results in the
  following web page:
  \ScreenShot{Single answer multiple choice question}{choice-single}

  Here is an example of a multiple choice question that has
  two correct answers:
    \InputLatexCode{choice-multiple}
  Notice that this example uses the document-class option
  \LatexCode{theme=vibrant}, which changes the \WebQuiz colour theme;
  see \autoref{SS:classOptions}.
  \ScreenShot{Multi-answer multiple choice question}{choice-multiple}

  When the optional argument \LatexCode|multiple| to the
  \LatexCode|choice| environment is used, as above, then the question is
  marked correct if and only if \textit{all} of the correct choices, and
  none of the incorrect choices, are selected. If the student's
  selections are not completely correct then are given feedback that is
  randomly selected from amongst their wrong choices (that is, the
  feedback is randomly selected from the set of \LatexCode|\correct|
  choices that were not selected and the \LatexCode|\incorrect| choices
  that were selected).

  Finally, observe that the multiple choice items in the screenshot
  above are labelled by roman numerals. The items in a
  \LatexCode{choice} environment are labelled by a standard \LaTeX{}
  counter, that is also called \LatexCode|choice|. Redefining the
  \LaTeX{} macro \LatexCode|\thechoice| changes how the corresponding
  question choices are labelled in the online quiz. For example, to
  label the items in a \LatexCode|choice| environment by A), B), C)
  $\dots$ add the line:
  \begin{latexcode}
    \renewcommand\thechoice{\Alph{choice})}
  \end{latexcode}
  to the preamble of the \LaTeX{} file for your quiz.
  \CrossIndex{choice environment}*{thechoice}\macroIndex{thechoice}

  \subsubsection{Discussion environments}\label{SS:discussion}
    \index{environment!discussion}

  In addition to asking questions, it is possible to display revision,
  or discussion, material on the quiz web page using the
  \LatexCode|discussion| environment.  All discussion material is
  displayed on the quiz page \textit{before} the questions in the quiz
  and the (short) titles for the discussion items appear before the quiz
  questions in the menu on the left-hand side of the page --- it is
  not possible to interleave discussion items and questions in the side
  menu.

  Each quiz can have zero or more discussion environments. These
  environments can, in principle, contain arbitrary \LaTeX{} code. The
  syntax for the \LatexCode|discussion| environment is:
  \begin{latexcode}
    \begin{dicussion}[short heading][heading]
      ...
    \end{dicussion}
  \end{latexcode}
  The \textit{short heading} is used in the side-menu.  Both the
  \textit{heading} and \textit{short-heading} are optional, both
  defaulting to \LatexCode|Discussion|. If only one heading is given
  then this sets both the \textit{short heading} and \textit{heading}
  for the discussion item. For example, running the following \LaTeX\
  file, which uses \LatexCode|theme=muted|, through \WebQuiz
  \InputLatexCode{discussion}
  produces the quiz page:
  \ScreenShot{Web page: discussion environment}{discussion}
  As with the questions, only one \LatexCode|discussion| environment is
  displayed on the quiz web page at a time (unless the document-class
  option \LatexCode|onepage| is used). It is possible to have quizzes
  that contain only \LatexCode|discussion| environments, with no
  questions, and quizzes that contain only \LatexCode|question|
  environments, with no discussion.

  If you have a mixture of \LatexCode|discussion| and
  \LatexCode|question| environments then it is useful to be able to add
  thinks between them.  \LaTeX{} provides the \LatexCode|\label| and
  \LatexCode|\ref| commands for cross-referencing, so \WebQuiz  builds
  on this idea and provides the three commands \LatexCode|\dref|,
  \LatexCode|\qref| and \LatexCode|\Qref| to reference
  \LatexCode|discussion| and \LatexCode|question| environments. The
  syntax for these commands is as follows:
  \begin{latexcode}
    \dref[optional text]{LaTeX label}      % inserts discussion button
    \dref*[optional text]{LaTeX label}     % inserts discussion link
    \qref[optional text]{LaTeX label}      % inserts question button by label
    \qref*[optional text]{LaTeX label}     % inserts question link by label
    \Qref[optional text]{question number}  % inserts question button by number
    \Qref*[optional text]{question number} % inserts question link by number
  \end{latexcode}
  \CrossIndex{cross-reference}*{dref}
  \CrossIndex{cross-reference}*{qref}
  \CrossIndex{cross-reference}*{Qref}
  \CrossIndex{cross-reference}*{label}
  In each case the text in the link or button defaults to either the
  short-title, for discussion environments, or the question number for
  questions.  The ``optional text'' is used instead whenever it is
  supplied. These commands can be used anywhere in a quiz, including
  \LatexCode|discussion|, \LatexCode|question| and \LatexCode|choice|
  environments and inside feedback text for the students written using
  \LatexCode|\feedback|, \LatexCode|\whenRight| and \LatexCode|\whenWrong|.

  The macros \LatexCode|\dref| and \LatexCode|\qref| use a standard
  \LaTeX{} \textit{label}, defined using the \LatexCode|\label| command,
  to insert a button or a link.  In contrast, \LatexCode|\Qref| uses the actual
  \textit{question number}, so it is not necessary to define a
  \LatexCode|\label| for a question when using \LatexCode|\Qref|.

  Even though the macros \LatexCode|\qref| and \LatexCode|\Qref| are
  quite similar they serve different functions when the
  \LatexCode|randomorder| document class option is used (see
  \autoref{SS:classOptions}). In this case we do not know ahead of time
  the question numbers that will be used in the quiz. So if
  \LatexCode|q:one| is the label for the first question in the \LaTeX{}
  file then \LatexCode|\qref{q:one}| will insert a button to this
  question \textit{but} this will almost certainly not be Question~$1$.
  On the other hand, we can create a ``Start quiz'' button, for example,
  that will open Question~1 on any quiz, using
  \LatexCode|\Qref[Start quiz]{1}|.

  Here is an example that shows how \LatexCode|\Qref| works:
  \InputLatexCode{discussion-Qref}\macroIndex{Qref}
  which produces the online quiz:
  \ScreenShot{Crossing-referencing using question numbers}{discussion-Qref}
  Similarly, here is an example showing how \LatexCode|\dref| and \LatexCode|qref|
  are used:
  \InputLatexCode{discussion-ref}\macroIndex{dref}\macroIndex{qref}
  This code produces the online quiz:
  \ScreenShot{Crossing-referencing using labels}{discussion-ref}

  \begin{dangerous}
    When using the \LatexCode|randomorder| document class option
    (\autoref{SS:classOptions}), ``optional text'' should always be
    given when using \LatexCode|\qref|. This is because the question
    number that is displayed by default will always be the question
    number \textit{in the \LaTeX{} file} rather than the question number
    in the online quiz.
  \end{dangerous}

  \subsubsection{Index pages for quizzes}\label{SS:index}
    \index{quizindex environment}
    \index{environment!quizindex}

    Most quizzes occur in sets that cover related material, such as for
    a particular unit of study. The \LatexCode|quizindex| environment
    creates an index web page for related sets of quizzes. The \LaTeX{}
    files for all of these quizzes must be in the same directory, or
    folder, on the web server.  The index web page is a \WebQuiz file of
    the form:
    \InputLatexCode{index-en}
    which generates a web page that looks like:
    \ScreenShot[0.7]{Example index page}{index-en}
    As the next section describes, index files are also used to
    automatically add a drop-down menu that contains the quiz-index to
    the breadcrumbs on all of the  quiz web pages. This drop-down menu
    provides an easy way to navigate between all of the quizzes for a
    particular unit of study.

    As the example above shows, the entries in the \LatexCode|quizindex| are
    given using the \LatexCode|\quiz| or the \LatexCode|\quiz*| command.
    \CrossIndex{quizindex environment}*{quiz}
    The syntax for these commands is
    \begin{latexcode}
        \quiz[URL for quiz]{Title for quiz}
        \quiz*[URL for quiz]{Title for quiz}
    \end{latexcode}
    The \LatexCode|\quiz| macro automatically inserts the quiz numbers
    into the index listing. The \LatexCode|\quiz*| command is identical
    \textit{except} that it does not add \BashCode|Quiz 1.|,
    \BashCode|Quiz 2| etc to the index listing. By default, the URLs for
    the quizzes in the index are assumed to be of the form
    \BashCode|quiz1.html|, \BashCode|quiz2.html|, \BashCode|quiz3.html|,
    .... These URLs can be changed using the optional argument of the
    \LatexCode|\quiz| and \LatexCode|\quiz*| commands. For example,
    \begin{latexcode}
         \quiz[realquiz.html]{This is the real quiz}
    \end{latexcode}
    would create an item in a quiz index that links to the web page
    \BashCode|realquiz.html|.

    Index pages in other languages are produced in exactly the same way. For example,
    \InputLatexCode{index-cz}
    produces the web page:
    \ScreenShot[0.7]{A Czech index}{index-cz}
    The next section describes the \LatexCode|\BreadCrumb| and
    \LatexCode|\BreadCrumbs| commands.

    At most one file in each directory should contain a
    \LatexCode|quizindex| environment. This is because \WebQuiz creates
    the file \Javascript file \BashCode|quizindex.js| whenever it
    compiles a \LatexCode|quizindex| environment. This file contains
    \Javascript commands for displaying the quiz index.

    \subsubsection{Breadcrumbs}\label{SS:breadcrumbs}\index{breadcrumbs}
    \WebQuiz provides a straightforward way to place navigation breadcrumbs
    at the top of the quiz web page. By default the
    breadcrumbs are disabled. If you have a \LatexCode|\BreadCrumbs| command
    like:
    \begin{latexcode}
        \BreadCrumbs{Mathematics /|Math1001 /u/Math1001|quizindex|title}
    \end{latexcode}
    in your \LaTeX{} file then \WebQuiz will add a corresponding strip
    of breadcrumbs, or navigation links, to the top of your quiz page:
    \ScreenShot{Example breadcrumbs}{breadcrumbs}
    The drop-down menu is normally hidden,  appearing only after the
    $\equiv$ ``button'' on the web page is clicked.

    Usually, most of the breadcrumbs are navigation links to other web pages. In
    the example above:
    \begin{itemize}
      \item The first ``crumb' inside the
      \LatexCode|\BreadCrumbs| command is \BashCode|Mathematics /|. This inserts the text
      \BashCode|Mathematics| together with a (relative) URL to
      \BashCode|/|, the root directory for the web
      server, which is often the correct URL for the department (or the
      institution)
      \item The \BashCode|Math101 /u/math101| inserts the text
      \BashCode|Math101| as the second breadcrumb with URL
      \BashCode|/u/math101|.
      \item The \BashCode|quizindex| inserts the text \BashCode|Quizzes|
      together with the symbol $\equiv$, which opens a drop-down menu
      that contains the list of quizzes in the quiz index for the unit.
      This is described in more detail below.
      \item Finally the \LatexCode|title| in the breadcrumbs inserts, as
      text, the part of the title \textit{before the first colon} in the title,
      where the title is given by \LatexCode|\title{...}|.
    \end{itemize}
    The breadcrumbs for the quiz web page can be either be configured
    quiz-by-quiz, using the \LatexCode|\BreadCrumbs| macro
    \macroIndex{BreadCrumbs}, as above, or by setting default
    \LatexCode|breadcrumbs| in the \webquizrc using the command-line
    option
    \begin{bashcode}
      > webquiz --edit-settings
    \end{bashcode}
    as described in \autoref{SS:commandline}. Breadcrumbs are disabled by default.

    The breadcrumbs inside the \LatexCode|\BreadCrumbs{...}| command are given as
    a ``|-separated list''. For example, quite reasonable breadcrumbs are given by:
    \begin{latexcode}
      \BreadCrumbs{ department | unitcode | quizindex | title }
    \end{latexcode}
    To make this the default set of breadcrumbs use
    \BashCode{webquiz --edit-settings} to set breadcrumbs
    in the \webquizrc to:
    \begin{latexcode}
       department | unitcode | quizindex | title
    \end{latexcode}
    More generally, the breadcrumbs can be specified as:
    \begin{latexcode}
      \BreadCrumbs{ crumb1 | crumb2 | crumb3 | crumb4 | ...  }
    \end{latexcode}
    In principle, there can be arbitrarily many crumbs in your
    breadcrumbs but, in practice, five is more than enough.
    The crumbs inside a \LatexCode|\BreadCrumbs| command have the
    following meanings:

    \begin{description}
      \item[breadcrumb] The breadcrumb for the current quiz, which is
      set using the \LatexCode|\BreadCrumb| macro.  This breadcrumb is
      purely descriptive, with no hyperlink being added: only the text given by
      \LatexCode|\BreadCrumb| appears.
      \CrossIndex*{BreadCrumbs}{breadcrumb}\macroIndex{BreadCrumb}

      \item[department] This expands to a link to your department, where
      the department text is set using the macro \LatexCode|\Department| and its URL is set by
      \LatexCode|\DepartmentURL|.
      \macroIndex{Department}\macroIndex{DepartmentURL}
      \CrossIndex*{BreadCrumbs}{department}

      \item[institution] This expands to a link to your institution, where
      the institution text is set using \LatexCode|\Institution| and its URL is set by
      \LatexCode|\InstitutionURL|. The institution also appears in the
      side-menu above the \WebQuiz copyright notice.
      \macroIndex{Institution}\macroIndex{InstitutionURL}
      \CrossIndex*{BreadCrumbs}{institution}

      \item[quizindex] This expands to \BashCode|Quizzes|, which is a
      link to the index page for your unit, as defined by
      \LatexCode|\QuizzesURL|, which is described below.  In addition,
      if the directory contains a BashCode|quizindex.js| file, which is
      created by the \LatexCode|quizindex| environment (see
      \autoref{SS:index}), then the symbol {\large$\equiv$} will appear,
      giving access to a drop-down menu to the index page, looking
      something like this:
      \ScreenShot{Drop-down menu giving index of quizzes}{quizindex-dropdown}
      \macroIndex{QuizzesURL}
      \CrossIndex*{BreadCrumbs}{quizindex} Such drop-down menus are
      automatically added to quiz web pages that have a
      \LatexCode|quizindex| breadcrumb| as soon as
      an quiz page that contained a \WebQuiz \LatexCode|quizindex| environment has
      been compiled in the current directory.
      \CrossIndex{breadcrumbs}{quizindex}

      For those interested in how this is done, whenever \WebQuiz
      compiles a \LatexCode|quizindex| environment it creates a
      \Javascript file \BashCode|quizindex.js|. Every quiz file includes
      this \Javascript file, if it exists, and this allows it to display
      a drop-down menu for the quiz index.

      \item[Title] This expands to the full title of the quiz page,
      without a hyperlink, as given by the \LatexCode|\title|.
      \CrossIndex*{BreadCrumb*}{Title}

      \item[title] This expands to the part of title of the quiz page,
      without a hyperlink, that occurs \textit{before} the first colon
      in the title of the quiz.  For example, if the title is given as
      \begin{latexcode}
        \title{Quiz 1: The wonders of life}
      \end{latexcode}
      then ``Quiz 1'' will be added to the list of breadcrumbs. If the
      title does not contain a colon then the full title is printed.
      \CrossIndex*{BreadCrumbs}{title}

      \item[unitcode] This expands to a link to the unit code, where the
      unit code text is set using \LatexCode|\UnitCode|, and its URL is
      set by
      \LatexCode|\UnitURL|
      \macroIndex{UnitCode}\macroIndex{UnitURL}
      \CrossIndex*{BreadCrumbs}{unitcode}

      \item[unitname] This expands to a link to the unit name, where the
      unit name text is set using \LatexCode|\UnitName|, and its URL is
      set by
      \LatexCode|\UnitURL|
      \macroIndex{UnitName}\macroIndex{UnitURL}
      \CrossIndex*{BreadCrumbs}{unitname}

    \end{description}

    \noindent
    In addition, each \textit{crumb} in a breadcrumb, except for the
    ``magic crumbs'' listed above, is allowed to be arbitrary text ---
    although, non-ascii characters may cause problems. In this case, the
    last ``word'' in the crumb is treated as a URL if it either beings
    with a forward slash, \BashCode|/|, or if it begins with
    \BashCode|http|.  For example, the code:
    \InputLatexCode{montypython}
    \noindent
    results in the following breadcrumbs:
    \ScreenShot{Guaranteed to offend some one}{montypython}
    Notice that it is necessary to correctly escape spaces etc in
    URLs that are specified this way. Similarly, all of the characters
    in the breadcrumbs should be ascii characters as unicode is likely to
    cause encoding issues (compare with the Czech index given in
    \autoref{SS:index}).

    If any part of a ``magic'' breadcrumb has not been defined then it
    is printed with a question mark on the web page. For example, the
    quiz file
    \InputLatexCode{nounits}
    does not define the unit code, so it results in the ``questionable'' first breadcrumb:
    \ScreenShot{Breadcrumbs with no unit code}{nounits}

    Here is the list of \WebQuiz macros that we be used to set the
    values of the ``magic'' breadcrumbs inside a
    \LatexCode|\BreadCrumbs|. Note that default values for many of these
    ``crumbs'' can be given in the \webquizrc.

    \newcommand\Item[1]{\item[\textbackslash#1]\CrossIndex{breadcrumbs}*{#1}}
  \begin{description}
    \Item{BreadCrumb}\macroIndex{BreadCrumb}
      The \LatexCode|\BreadCrumb| command sets the \LatexCode|breadcrumb|
      variable in the \LatexCode|\BreadCrumbs|. The primary use for this
      is when you have default breadcrumbs in the \webquizrc like
      \begin{bashcode}
          breadcrumbs = department | unitcode | breadcrumb
      \end{bashcode}
      Using \LatexCode|\BreadCrumb| allows you to set the last crumb to
      some meaningful text that describes the quiz.

    \Item{Department}\macroIndex{Department}
      The \LatexCode|\Department| command sets the name of the
      \LatexCode|department|. As described earlier in this section, by
      default, the \LatexCode|department| is the first item in the
      breadcrumbs that appear at the top of the web page.

      The default department can be set in the \webquizrc using
      \BashCode|webquiz --edit-settings|.

      Default value: \BashCode{''} \qquad(i.e. the empty string)

    \Item{DepartmentURL}\macroIndex{DepartmentURL}
      The \LatexCode|\DepartmentURL| command sets URL for the department. As
      described earlier in this section, by default the department URL is
      the link in the first breadcrumb on each web page.

      The default department URL can be set in the \webquizrc using
      \BashCode|webquiz --edit-settings|.

      Default value: \DefaultValue{/}

      \Item{Institution}\macroIndex{Institution}
      The \LatexCode|\Institution| command sets the institution, or university.
      The \LatexCode|institution| appears below the question buttons in the
      left-hand navigation menu that appears on every quiz web page
      (provided that the screen size is not too small).  As described
      earlier in this section, the institution can be used in the web
      page breadcrumbs.

      The default institution can be set in the \webquizrc using
      \BashCode|webquiz --edit-settings|.

      Default value: \BashCode{''} \qquad(i.e. the empty string)


      \Item{InstitutionURL}\macroIndex{InstitutionURL}
      The \LatexCode|\InstitutionURL| command sets the institution URL. This is
      used as the link for the \LatexCode|institution| in the left-hand
      navigation menu that appears on every quiz page.  As described
      earlier in this section, the institution URL can be used in the
      web page breadcrumbs.

      The default institution URL can be set in the \webquizrc using
      \BashCode|webquiz --edit-settings|.

      Default value: \DefaultValue{/}

      \Item{QuizzesURL}\macroIndex{QuizzesURL}
      The \LatexCode|\QuizzesURL| command sets the URL for the suite of quizzes
      attached to this unit of study. As described earlier in this
      section, this can be used in the breadcrumb at the top of the quiz
      web page.

      Default value:  \DefaultValue{UnitURL/Quizzes}, where \DefaultValue{UnitURL} is set
      using \LatexCode|\UnitURL|

      \Item{UnitCode}\macroIndex{UnitCode}
      The \LatexCode|\UnitCode| command sets the unit of study code for the
      unit that the quiz is part of.

      \Item{UnitName}\macroIndex{UnitName}
      The \LatexCode|\UnitName| command sets the name of the unit of study for
      the unit that the quiz is attached to.

      \Item{UnitURL}\macroIndex{UnitURL}
      The \LatexCode|\UnitURL| command sets the URL for the unit of study code
      for the unit that the quiz is attached to.

  \end{description}

  \noindent
  It makes sense to set defaults for \LatexCode|\BreadCrumbs|
  \LatexCode|\Department|, \LatexCode|\DepartmentURL|,
  \LatexCode|\Institution| and \LatexCode|\InstitutionURL| in the
  \webquizrc. After doing this, a typical \WebQuiz file might look like this:

  \begin{latexcode}
    \documentclass{webquiz}
    \UnitName{Fundamental stuff}
    \UnitCode{Stuff101}
    \UnitURL{/courses/stuff101}

    \title{Stuffing: the art of taxidermy}

    \begin{document}
      \begin{question}
        first question...
      \end{question}
      \begin{question}
        second question...
      \end{question}
    \end{document}
  \end{latexcode}

  If you have many quizzes for many different units then a better approach is to create
  a \LaTeX{} ``package'', say \BashCode|ourunits.sty|, to set these variables:
  \begin{latexcode}
    % ourunits.sty - set unit names, codes and URLs
    \DeclareOption{stuff101}{
      \UnitName{Fundamental stuff}
      \UnitCode{Stuff101}
      \UnitURL{/courses/stuff101}
    }
    \DeclareOption{stuff102}{
      \UnitName{Fundamental stuff too}
      \UnitCode{Stuff102}
      \UnitURL{/courses/stuff102}
    }
    \ProcessOptions
    \endinput
  \end{latexcode}
  Then you can replace the opening lines of the quiz file with
  \LatexCode|\usepackage[stuff101]{ourunits}|. Of course, you could also
  set the default \LatexCode|\Department|, \LatexCode|\DepartmentURL|,
  \LatexCode|\Institution| and \LatexCode|\InstitutionURL| in such a
  style file as well.

  \subsection{\WebQuiz document class options}\label{SS:classOptions}

    The \WebQuiz document class supports the following options:
    \begin{latexcode}
      fixedorder, hidesidemenu, language, onepage, pst2pdf,
      separatepages, showsidemenu, theme, tikz
    \end{latexcode}
    This section describes all of these document-class options except for
    \ctan{tikz} and \ctan{pst2pdf}, which are discussed in
    \autoref{SS:graphics}. Many of the document-class options below occur
    in pairs and defaults for many of these can be set in the \webquizrc
    file. The settings given in the \LaTeX{} file
    for a quiz will always override the default settings in the \webquizrc*.

  \begin{description}

    \item[fixedorder, randomorder]
    \CrossIndex{document class options}{fixedorder}
    \CrossIndex{document class options}{randomorder}
      By default, the questions in the quiz are displayed in a
      \DefaultValue{fixedorder} for all students who take the quiz.
      This order is the order that the question appear in the \LaTeX{}
      file for the quiz. That is, the first question in the online quiz
      is the first question appearing in the \LaTeX file, the second
      question in the quiz is the second question in the \LaTeX{} file,
      and so on. With the \LatexCode|randomorder| option the questions
      in the quiz are displayed in a random order that changes each time
      that the quiz is run. With this option, the online quiz questions
      are generally in a different order for every student. For example,
      the code:
      \InputLatexCode{random}
      produces the quiz with randomly arranged quiz questions, such as:
      \ScreenShot{Randomly ordered questions}{random}
      (So the first question in the \LaTeX{} file is being displayed as
      the fourth online quiz question. The next time the page is
      loaded, such as for the next student, the question order will
      change again.) When using the \LatexCode|randomorder|
      document-class option only the questions appear in random order.
      If the quiz contains multiple choice questions then the choices
      are \textit{not} randomly permuted. That is, the choices always
      appear in the order that they are written in the \LaTeX{} file.

    \item[hidesidemenu, sidemenu]
    \CrossIndex{document class options}{hidesidemenu}
    \CrossIndex{document class options}{showsidemenu}
      If the \LatexCode|hidesidemenu| option is set then the side menu
      on the left-hand side of the quiz web page will not be displayed
      when the quiz first loads. By default, the side menu appears
      unless the screen size is too small, such as on a mobile phone.
      Many examples of the \LatexCode|hidesidemenu| and
      \LatexCode|showsidemenu| class options can be found above.

      The display of the side menu can is also toggled by clicking on
      the {\textcolor{red}{\small\XSolidBold}} and
      {\textcolor{red}{\large\ding{118}}} symbols. The side-menu
      automatically disappears for devices with narrow screens, such as
      mobile phones.

    \item[language=<lang>] \CrossIndex{document class options}{language}
    Set the language used by the \textit{web pages} constructed by \WebQuiz.
    The following languages are currently supported by \WebQuiz:
    \ListWebQuizLanguages
    The languages files are used to print the various buttons and text
    that is generated on the web pages constructed by \WebQuiz. The
    \LatexCode|language| option does not affect the DVI or
    \BashCode|PDF| versions of the quiz and it does not load language
    packages like \ctan{babel} or \ctan{polyglossia}.

    The \LatexCode|language| keyword can be in upper or lower case, with
    the result that either (but not both!) of the following two lines set the
    quiz language to German:
    \begin{latexcode}
      \documentclass[language=German]{webquiz}
      \documentclass[language=german]{webquiz}
    \end{latexcode}

    Typical usage of the \LatexCode|language| option is the following:
    \lstinputlisting[style=latexcode,extendedchars=false]{examples/french}
    This produces a web page like this:
    \ScreenShot{A web page with \textsf{language=french}}{french}
    As a general rule, \LaTeX{} and \TeXfht do not cope well with
    unicode characters, so if your quiz contains (a lot of) unicode
    characters then we recommend using \hologo{LuaLaTeX} or \hologo{XeLaTeX}, which
    corresponds to the \WebQuiz command-line options
    \BashCode|webquiz -x| or \BashCode|webquiz -l|, respectively. The
    default \TeX{} engine can be set in the \webquizrc.
    \index{unicode}\index{lualatex}\index{xelatex}

    The language files were created largely using google translate so
    they may well need fine-tuning\footnote{The word ``Copyright'' in
    the left-hand side-margin is not translated but perhaps it should
    be.}. You can use \BashCode|kpsewhich| to look at the language
    files, which all have names of the form \BashCode|webquiz-xxx.lang|,
    where \BashCode|xxx| is the name of the language in lower case.  For example,
    the \textit{English} language file, which is the default, can be
    found using the command:
    \begin{bashcode}
         > kpsewhich webquiz-english.lang
    \end{bashcode}
    The file \BashCode|webquiz-english.lang| contains the following:
    \InputBashCode{webquiz-english.lang}
    In these files, the material to the left of the equals signs are
    effectively variables, and so they should never be changed, or
    deleted, whereas anything to the right of the equals signs is the
    text that will appear on the \WebQuiz web pages. \textit{The pairs
    of braces, }\LatexCode|{}|, \textit{in the language files must be present
    because in the online quizzes they expand to expressions like}
    \LatexCode|(a)|, \LatexCode|(b)|\dots

    To add \WebQuiz support for a new language, say language
    \LatexCode|xxx|, copy any \WebQuiz language file to a new file
    \BashCode|webquiz-xxx.lang| and then translate all of the words to
    the right of the equals signs.\footnote{\WebQuiz assumes that all
    language names are in lower case so \texttt{xxx}, and not
    \texttt{XXX}, should be used.} \WebQuiz will be able to find the new
    language file as long as it appears in the \LaTeX{} search
    path\footnote{To help \LaTeX/\WebQuiz fined your language file you
    may need to run a program like \texttt{mktexlsr} using an
    administrators account}. Once the new language file
    \BashCode|webquiz-xxx.lang| is in the \LaTeX{} search path it can be
    used by \WebQuiz using \LatexCode|language=xxx| as a document-class
    option:
    \begin{latexcode}
       \documentclass[language=xxx]{webquiz}
       \begin{document}
            ...quiz code here...
       \end{document}
    \end{latexcode}
    Please submit any new language files, or corrections to existing
    language files, as a \textit{new issue} at:
      \href{https://\webquiz{repository}}{\webquiz{repository}}.

    \item[onepage, separatepages]
    \CrossIndex{document class options}{onepage}
    \CrossIndex{document class options}{separatepages}
    By default, only one question, or one discussion environment, is displayed
    by the quiz at any time. As \LatexCode|separatepages| is the default, every
    example so far is of this form. With the document-class option
    \LatexCode|onepage| all questions, and discussion environments, are
    displayed at the same time on a single web page. So, for example, the code:
    \InputLatexCode{onepage}
    produces the quiz page:
    \ScreenShot{A one page quiz}{onepage}

    \item[theme]\CrossIndex{document class options}{theme}

       \WebQuiz\ has a small number of different themes for setting the
       colours on the quiz web pages. The theme can be set as an option to
       the document class or in the \webquizrc. Most, but not all, of the examples
       so far have used the \LatexCode|default| theme.
       \WebQuiz currently supports the following themes:
        \ListWebQuizThemes
       Example screenshots of all \WebQuiz themes can be found in
       \hyperref[SS:themes]{Appendix A}.

  \end{description}


  \subsection{Including graphics and using pstricks and tikz}\label{SS:graphics}
  It is also possible to include complicated diagrams in \WebQuiz
  quizzes using packages like \ctan{tikz} and \ctan{pstricks}.
  \textit{As there have been several recent updates to these packages it
  is advisable to install the latest version of both of these packages,
  as well as the packages \ctan{make4ht}, \ctan{pgf} and \TeXfht.} In
  fact, it is recommended that you update all installed \TeX{} packages.

  By far the easiest way to include images when using \WebQuiz is by
  adding the following lines to your document preamble:
  \begin{latexcode}
      \usepackage[dvipdfmx]{graphicx}
      \DeclareGraphicsExtensions{.png}
  \end{latexcode}
  to your document preamble. You need to use
  \LatexCode|\DeclareGraphicsExtensions| to tell \WebQuiz the different
  types of images you are using, so the code above works for
  \LatexCode{png} images. More generally, you can use a comma separated
  list of extensions, such as:
  \begin{latexcode}
    \DeclareGraphicsExtensions{.png, .jpg, .gif}
  \end{latexcode}
  The option \LatexCode|dvipdfmx|
  to \LatexCode|graphicx| is only necessary if you want to be able to
  rescale images. For example, the code:
      \InputLatexCode{ctanLion}
  produces the quiz page:
  \ScreenShot{Including a lion}{ctanLion}
  Note that \WebQuiz assumes that all images are \BashCode|SVG| images by
  default so it is necessary to give the full filename in any
  \LatexCode|\includegraphics| command.

  Using \ctan{pstricks} is often just as easy, such as the following
  code that works out of the box:
    \InputLatexCode{pstricks-ex}
  to produce the web page:
    \ScreenShot{Pstricks example}{pstricks-ex}

  Even though \ctan{tikz} and \ctan{pstricks} can be used in \WebQuiz
  quizzes, both of these packages sometimes have problems. \WebQuiz tries
  to solve some these problems for you if you use the
  \LatexCode|pst2pdf| or \LatexCode|tikz| document-class options, which
  we now describe.

  \begin{description}
      \item[pst2pdf] \index{pstricks}\CrossIndex{document class options}{pst2pdf}

      For the most part, \ctan{pstricks} drawings display correctly.
      When they do fail they can sometimes be salvaged using
      \ctan{pst2pdf}. Applying \ctan{pst2pdf} to a \WebQuiz quiz is not
      completely straightforward, so \WebQuiz provides the
      document-class option \ctan{pst2pdf} to automatically apply
      \ctan{pst2pdf} as part of the quiz web page build process. If your
      \ctan{pstricks} drawings do not display correctly it is worthwhile
      to see if \ctan{pst2pdf} fixes the problems.

      For example, the following quiz compiles only with the
      \ctan{pst2pdf} document-class option:
      \InputLatexCode{pst2pdf}
      to produce the quiz:
      \ScreenShot{Example requiring the pst2pdf document class option}{pst2pdf}
      The position of the image adjusts with the screen size and it
      does, in fact, display well on a mobile device.  \WebQuiz is not
      able to display this image without the \LatexCode|pst2pdf|
      document-class option.

      \begin{dangerous}
        Unfortunately, \ctan{pst2pdf} can fail silently without giving any
        warnings. If you plan to use the \LatexCode|pst2pdf|
        document-class option then you should first check to make that the
        \ctan{pst2pdf} package and executable is properly installed.
        According to the \ctan{pst2pdf} manual:
        \begin{quote}
          \ctan{pst2pdf} needs \Ghostscript (>=9.14), perl (>=5.18), pdf2svg, pdftoppm
          and pdftops (from poppler-utils or xpdf-utils) to process a file
          using \ctan{pst2pdf}.
        \end{quote}
        If using \ctan{pst2pdf} does not produce an image then, rather
        than \ctan{pst2pdf} not working, the problem might be that you
        have not installed all of the programs that \ctan{pst2pdf} relies
        upon, so look in your log files for error messages and check that
        all of the programs listed above are correctly installed, with the
        specified version numbers. See also \autoref{SS:pstricks}.
      \end{dangerous}

    \item[tikz]\CrossIndex{document class options}{tikz}
      Giving this class option both loads the \ctan[pgf]{tikz} package
      and it fixes several issues with PGF that prevent it from working
      with \TeXfht.  It is important that you use the \WebQuiz
      \LatexCode|tikz| document-class option, and not
      \LatexCode|\usepackage{tikz}|, because \WebQuiz loads slightly
      different configuration files for \ctan[pgf]{tikz} that are
      optimised for use with \TeXfht. \textit{Please note that for
      \ctan[pfd]{tikz} you need to use \TeXLive \textsf{2018} with all
      packages updated.} Thanks are due to Michal Hoftich for the
      enormous amount of effort that he has put into making \TeXfht and
      \ctan[pgf]{tikz} more compatible. As an example, the quiz file
      \InputLatexCode{tikz-ex}
      produces the online quiz:
      \ScreenShot{Tikz example}{tikz-ex}
  \end{description}

  Most people use either \ctan{pstricks} or \ctan{tikz}. A quiz that
  tries to use both \ctan{pstricks} and  \ctan{tikz} will probably not
  compile.

  \subsection{Configuring commands and environments for \TeXfht}\label{SS:config}\macroIndex{DisplayAsImage}

    The underlying engine used by \WebQuiz is \TeXfht so, because
    \TeXfht is not able process all \LaTeX{} code, there is \LaTeX{}
    code that \WebQuiz is not able to cope with.  This said, \TeXfht is
    able to display \textit{most} \LaTeX{} code and \WebQuiz has been
    used to write literally thousands of quiz questions so it is likely
    that you will be able to typeset what you want in your online
    quizzes.  In particular, as discussed in \autoref{SS:graphics}, it
    is possible to use \ctan[pgf]{tikz} and \ctan{pstricks} in \WebQuiz
    quizzes.

    \WebQuiz uses \TeXfht{} to convert the quiz content from \LaTeX{} to
    \HTML. If \TeXfht{} has not been configured to for some of the
    commands or environments that you are using then they may not
    display correctly in your online quizzes. The ``correct'' way to
    fix such problems is to write appropriate \TeXfht{} configuration
    commands, however, this can be tricky to do --- especially if you
    are not familiar with the inner workings of \TeX{} and \TeXfht.

    As a workaround, \WebQuiz provides the command
    \LatexCode|\DisplayAsImage| that, in effect, tells \TeXfht to treat
    your command as an image when it creates the web page. This is an
    easy work-around that often produces good results -- and it is much
    easier than writing your own \TeXfht configuration commands.

    For example, the \ctan{mhchem} package is a powerful package that
    defines a macro \LatexCode{\ce} for writing chemical symbols but,
    unfortunately, the \LatexCode{\ce} macro has not (yet) been
    configured to work with \TeXfht, which means that this command does
    not work very well when used in \WebQuiz quizzes.

    For example, the following code:
    \InputLatexCode{display-as-image}
    shows that if you add the line
    \LatexCode|\DisplayAsImage{ce}| to your quiz after
    \LatexCode|\begin{document}| then it is
    possible to use \LatexCode|\ce| in your quizzes:
    \ScreenShot{Using DisplayAsImage}{display-as-image}
    As the example code shows, \LatexCode|\DisplayAsImage| accepts an optional
    argument that can be used to fine-tune the placement of the image on
    the quiz web page using \CSS. For those interested in the technical details, the
    definition of \LatexCode|\DisplayAsImage| is:
    \begin{latexcode}
      \RequirePackage{etoolbox}
      \renewcommand\DisplayAsImage[2][]{%
        \csletcs{real:#2}{#2}%
        \NewConfigure{#2}{2}
        \csdef{#2}##1{\Picture+[#1]{}\csuse{real:#2}{##1}\EndPicture}
        \Configure{#2}{\Picture+[#1]{}}{\EndPicture}
      }
    \end{latexcode}

  \section{System requirements, installation and configuration}
    \label{S:configuration}
    \index{system requirements}
    \CrossIndex{system requirements}{tex4ht}
    \CrossIndex{system requirements}{make4ht}
    \CrossIndex{system requirements}{python}

    \WebQuiz takes a \LaTeX\ file and translates it into a functional
    web page. To use \WebQuiz the quiz author only needs to know how to
    use \LaTeX\ \textit{and} to have all of the programs used by \WebQuiz
    installed. Fortunately, most of the system requirements will already
    be installed on a system with an up-to-date installation of
    \TeXLive, however, some tweaking may still be necessary.

    Behind the scenes, \WebQuiz uses \TeXfht, \python, \Javascript and
    several other tools to construct and operate the online quizzes.
    The \WebQuiz program has three main components:
      \begin{itemize}
           \item \LaTeX\ files (a class file and \TeXfht configuration files)
           \item A Python3 program that uses \TeXfht to convert \LaTeX\
           files into web pages
           \item Web files (\Javascript, \CSS and online documentation)
          %\item Documentation
      \end{itemize}
    Of course, to use the online quizzes created by \WebQuiz you
    need a web server. To use \WebQuiz all of these files need to
    be in appropriate places.  Fortunately, \Ctan takes care of
    most of this but the web-related files still need to be put onto
    your web server.

    \WebQuiz has been tested extensively on Linux and Mac operating
    systems. Several people have used \WebQuiz on windows computers,
    but I have not tested the program on a windows computer myself.

  \subsection{System requirements}\index{system requirements}

  In order to work \WebQuiz needs the following programs to be
  installed on your system:
  \begin{itemize}
     \item An up-to-date \LaTeX{} distribution, such as that provided
     by \TeXLive. In particular, you need to have \TeXfht{} and
     \ctan{make4ht} installed.

     Unfortunately, the version of \ctan{make4ht} that was released
     with \TeXLive{} 2018 had some bugs and there have been many
     recent changes to \ctan{make4ht} and \ctan{pstricks},
     so it is strongly recommended that you update all packages
     from \Ctan before you try and use \WebQuiz.

      \item \python[Python 3]. As of writing python 3.7.2 is available.
      \item \Javascript
      \item If your quizzes use \ctan{pstricks}, or if you want to
      compile the \OnlineManual, then you need to ensure that
      \Ghostscript and \ctan{dvisvgm} are installed and properly
      configured; see \autoref{SS:pstricks} for more details.
      \item A web server. As detailed in \autoref{SS:Initialise}, you
      will need to install the web components of \WebQuiz.
  \end{itemize}

  \subsection{Initialising \WebQuiz}\label{SS:Initialise}
  \CrossIndex{command-line option}{initialise}
  \index{initialisation}

  \WebQuiz is a tool for creating online quizzes and in order for
  it to work efficiently various files
  (\href{https://en.wikipedia.org/wiki/JavaScript}{javascript} and
  \href{https://www.w3schools.com/css/css_intro.asp}{cascading style
  sheets}) need to put onto your web server. \WebQuiz has an
  \textit{initialisation} routine for installing the web components
  of the program. In fact, until \WebQuiz has been initialised it
  will ask you if you to run the initialisation routine every time
  you use \WebQuiz. You can reinitialise \WebQuiz at any time using
  the command-line option:
  \begin{bashcode}
    > webquiz --initialise
  \end{bashcode}
  \WebQuiz will actually work without being initialised, however, any
  quiz web pages that are created before initialisation will be
  emblazoned with a message reminding you to initialise \WebQuiz.

  The location of the files on the web server depends both on the
  operating system that is running on your computer and how your web
  server has been configured. It is essential that the \WebQuiz files
  are installed in a directory that is accessible from the web. It
  does not matter if they are put into a user web directory or into a
  system web directory.  If in doubt please consult your system
  administrator.

  Common locations for the \textit{web root} of the server are:
  \begin{center}
    \begin{tabular}{ll}
      \toprule
      Operating system & Root of web server \\\midrule
      Mac OSX     & \BashCode|/Library/WebServer/Documents|\\
      Linux       & \BashCode|/var/www/html|\\
      Windows     & \BashCode|C:\inetpub\wwwroot|\\
      \bottomrule
    \end{tabular}
  \end{center}
  \WebQuiz needs to copy several files into a subdirectory of this
  \textit{web root}. When you run
  \begin{bashcode}
    > webquiz --initialise
  \end{bashcode}
  you will be prompted for the following:
  \begin{itemize}
    \item The location of the \WebQuiz web directory, which needs to
    be a directory that is visible to your web server
    \item The relative URL for this directory, which tells your web
    browser where to find these files
  \end{itemize}
  For example, on my system the web root for our web server is
  \BashCode|/usr/local/httpd/| and the \WebQuiz web files are in
  the directory \BashCode|/usr/local/httpd/UoS/WebQuiz|.  So, I set:
  \begin{quote}
    \begin{tabular}{lll} \WebQuiz web
          directory &=& \BashCode|/usr/local/httpd/UoS/WebQuiz|\\
          \WebQuiz relative URL  &=& \BashCode|/UoS/WebQuiz|
      \end{tabular}
  \end{quote}
  Once the initialisation step is complete, \WebQuiz is ready to use
  although you, possibly from an administrators account or using
  \BashCode|sudo|, may also want to run:
  \begin{bashcode}
     > webquiz --edit-settings
  \end{bashcode}
  This will talk you the process of setting system defaults for the quizzes that,
  for example, specify the name and URL for your department and
  institution as well as the default language and theme used for the
  quizzes. If in doubt about any of the option press return to accept
  the default. See \autoref{SS:rcfile} for more details.

  You can test your \WebQuiz installation by compiling the example
  files from the \WebQuiz manual. You can find these files in the
  directory \BashCode|web_root/doc/examples|, where the
  \BashCode|web_root| is the directory where you just installed
  the \WebQuiz web files. If \Ghostscript and \ctan{dvisvgm} are
  installed and properly configured (see \autoref{SS:graphics}) then you
  should also be able to compile the \OnlineManual using \WebQuiz.

  \begin{dangerous}
      \textit{To install the \WebQuiz files for general use on
      your system, or to save system wide settings, you need to run
      the initialisation command }\BashCode|webquiz --initialise|
      \textit{using an administrator account or using \BashCode|sudo| on a
      UNIX or Mac OSX system.}

     If you have already saved a \textit{user} \webquizrc file then to
     change the \textit{system} \webquizrc* you will use need to
     use the \BashCode|--rcfile| command-line option:
     \begin{latexcode}
         > webquiz --rcfile /path/to/system/rc-file --initialise
     \end{latexcode}\index{rcfile}
  \end{dangerous}

  \begin{heading}[Remark]
    To remove all \WebQuiz files from your web server use:
    \begin{bashcode}
        > webquiz --uninstall
    \end{bashcode}
  \end{heading}

  \subsection{Graphics and \texorpdfstring{\ctan{dvisvgm}}{dvisvgm}}
  \label{SS:pstricks}

  \index{pstricks}
  \noindent
  \WebQuiz, via \TeXfht, uses \ctan{dvisvgm} to convert certain images to
  \href{https://en.wikipedia.org/wiki/Scalable_Vector_Graphics}{Scalable
  Vector Graphics} (SVG). This is done using his is done
  using \ctan{dvisvgm}. At first sight this is OK because \ctan{dvisvgm} is
  included in \TeXLive and Mik\TeX, however, \ctan{dvisvgm} uses
  \Ghostscript and this needs to be correctly configured and, as
  outlined in \href{http://dvisvgm.bplaced.net/FAQ}{FAQ}, \ctan{dvisvgm}
  needs to know where to find the \Ghostscript libraries. For example,
  to get \ctan{dvisvgm} to work on my system I needed to add the line
  \begin{bashcode}
    export LIBGS=/usr/local/lib/libgs.dylib
  \end{bashcode}
  to my \BashCode|.bashrc| file. To see whether you need to do something
  similar on your system you need to look at the output from the following
  two commands:

    \begin{bashcode}
      > dvisvgm -h
      > dvisvgm -l
    \end{bashcode}

  \noindent There are three possibilities:

    \begin{itemize}
      \item the \BashCode|-h| output does not contain \BashCode|-libgs|
      and the \BashCode|-l| output contains \BashCode|ps|:
      ghostscript was linked at build time, so everything should work
      \item
      the \BashCode|-h| output contains \BashCode|-libgs| and the
      \BashCode|-l| output does not contain \BashCode|ps|:
      gpostscript support is enabled but ghostscript is not linked. You
      need to locate the ghostscript library
      \BashCode|libgs.so| or \BashCode|libgs.dylib| on your system and set
      the \BashCode|LIBGS| environment variable, or equivalent,
      accordingly
      \item the \BashCode|-h| output does not contain \BashCode|-libgs|
      and the \BashCode|-l| output does not contain \BashCode|ps|:
      \ctan{dvisvgm} was not built with postscript support. In this
      case, \WebQuiz will not be able to process \ctan{svg} images. You
      need to reinstall \ctan{dvisvgm} with ghostscript support.
    \end{itemize}

  \section{The \WebQuiz program} \index{usage}

      The \WebQuiz program was designed to be run from the command-line.
      To produce an online quiz from a \WebQuiz \LaTeX{} file \BashCode|quiz.tex|
      type:
      \begin{bashcode}
        > webquiz quiz          or          > webquiz quiz.tex
      \end{bashcode}
      One feature of \WebQuiz is that you can process more than one
      quiz file at a time. For example, if you have quiz files
      \BashCode|quiz1.tex|, ..., \BashCode|quiz9.tex| in the current directory
      then, on a UNIX system, you can rebuild the web pages for all of
      these quizzes using the single command:
      \begin{bashcode}
        > webquiz quiz[1-9].tex
      \end{bashcode}
      This is useful if some generic aspect of all of the quizzes has
      changed, such as the theme, the language, thebreadcrumbs or a
      department URL. In fact, one would probably use
      \begin{bashcode}
        > webquiz -dqq quiz*.tex
      \end{bashcode}
      because the \BashCode|webquiz --qq| command-line option suppresses
      almost all of the output produced by \LaTeX\ and \TeXfht and
      \BashCode|-d|, which is \textit{draft mode},  is faster and it is
      probably sufficient if the quizzes were compiled recently. The
      next section discusses the \WebQuiz command-line options.

      \subsection{Command-line options}\label{SS:commandline}
      \index{command-line option}
      Typing \BashCode|webquiz -h|, or \BashCode|webquiz --help| on
      the command-line gives the following summary of the \WebQuiz
      command-line options:
      \InputBashCode{webquiz.usage}
      The command-line options are listed on separate lines above to
      improve readability. In practice, the different options need to be
      one the same line, although they can appear in any order.

      We describe the different options, grouping them according to
      functionality.

      \begin{description}
         \item[-h, --help] \CrossIndex{command-line option}{help}
            list the command-line options and exit
            \CrossIndex{command-line option}{help}

         \item[-q, -qq, \ddash quiet] \CrossIndex{command-line option}{quiet mode}
         Suppress \LaTeX{} and \TeXfht error messages: \BashCode|-q| is
         quiet and \BashCode|-qq| is  very quiet. If you use
         \BashCode|webquiz -qq texfile.tex| then almost no output will
         be printed by \WebQuiz when it is processing your quiz file. Be
         warned, however, that both of these options can make it harder
         to find and fix errors, so using the \BashCode|-q| and
         \BashCode|-qq| options is recommended if your file is known
         to compile.
         \CrossIndex{command-line option}{quiet}

         \end{description}

         \subsubsection*{\TeX{} options}

         \begin{description}

         \item[-d, \ddash draft] draft mode. The \LaTeX file is processed
         only once by \BashCode|htlatex|, which gives a much faster
         compilation time but cross references etc may not be completely
         up to date. This is equivalent to using:
         \BashCode|make4ht --mode draft|
         \CrossIndex{command-line option}{draft mode}

         \item[-s,\ddash shell-escape] Shell escape for \LaTeX/ht\LaTeX/make4ht
         \CrossIndex{command-line option}{shell-escape}

         \item[\ddash latex] Use \LaTeX{} to compile the quiz (the default)
         \CrossIndex{command-line option}{lua}

         \item[-l,\ddash lua] Use lua\LaTeX{} to compile the quiz
         \CrossIndex{command-line option}{lua}

         \item[-x, \ddash xelatex] Use xe\LaTeX{} to compile the quiz
         \CrossIndex{command-line option}{xelatex}
         \index{xelatex}

         \end{description}

         \subsubsection*{Settings and configuration}

         \begin{description}

            \item[-r {[RCFILE]}, \ddash rcfile {[RCFILE]}]
            \CrossIndex{command-line option}{rcfile} Specify the
            location of the \webquizrc file to use. This setting is only
            necessary if you want to override the default \webquizrc*.

            \item[-i, \ddash initialise] \CrossIndex{command-line
            option}{initialisation} Initialise files and settings for
            webquiz. See \autoref{SS:Initialise} for more details.

            \item[\ddash edit-settings] \CrossIndex{command-line
            option}{edit-settings} Edit the webquiz settings in the
            \WebQuiz rc-file. See \autoref{SS:rcfile} for more details.

            \textit{If you do not have permission to write to the system
            rc-file, which is in the \WebQuiz scripts directory, then you
            will be given the option of saving the \WebQuiz settings to an
            rc-file in your home directory or another file of your choice.
            If you want to save the settings to a particular rc-file use
            the \BashCode|--rcfile| option. If you want to change the
            system \webquizrc then use a administrator account or,
            on a unix-like system,} use \BashCode|sudo webquiz --edit-settings|.

            \item[\ddash settings {[SETTING]}] \CrossIndex{command-line option}{settings}
            \SeeIndex{rcfile}{webquizrc}
            \SeeIndex{default settings}{webquizrc}
            List system settings for webquiz stored in \webquizrc, the
            (\textit{run-time configuration file}). Optionally, a single
            \BashCode|SETTING| can be given in which case the value of
            only that setting is returned.  If \BashCode|SETTING| is
            omitted then the list of current settings are printed. Use
            \BashCode|SETTING=verbose| for a more verbose listing of the
            settings and \BashCode|SETTING=help| for a summary of the
            settings.

            The rc-file can be edited by hand, however, it is
            recommended that you instead use
            \begin{bashcode}
              webquiz --edit-settings
            \end{bashcode}
            Typical settings returned by this command look like:
            \InputBashCode{webquiz.settings}
            See \autoref{SS:rcfile} for more details.

          \end{description}

          \subsubsection*{Advanced command-line options}

          \begin{dangerous}
            \textit{Change these settings with care: an incorrect value for
            these settings can stop \WebQuiz from working.}
          \end{dangerous}

          \begin{description}
             \item[\ddash make4ht MAKE4HT-OPTIONS]
                 \CrossIndex{command-line option}{make4ht}
                Options to be passed to \ctan{make4ht} when converting the
                \LaTeX{} to \XML. This option is equivalent to setting
                the \BashCode|make4ht| in the \webquizrc; see
                \autoref{SS:rcfile}. At least under UNIX, multiple arguments
                should be enclosed in quotes.  For example, to give
                \ctan{make4ht} a custom mk4 file (note that
                \BashCode|myquiz.mk4| is included, if it exists), you would use
                \begin{bashcode}
                          > webquiz --make4ht "-e file.mk4" myquiz.tex
                \end{bashcode}
                The \BashCode|make4ht| command-line option will be required
                only in rare instances.
                \index{lualatex}\index{xelatex}

            \item[\ddash uninstall] Remove all \WebQuiz files from your
                web server directory. This command only removes files that
                \WebQuiz may have installed on your web server. It does
                \textit{not} remove \WebQuiz from your \LaTeX{} distribution.

           \item[\ddash webquiz-layout WEBQUIZ-LAYOUT]
           \CrossIndex{command-line option}{layout}
              Name of (local) \python code that controls the layout of quiz web page. This
              option is equivalent to setting the \LatexCode|webquiz-layout|
              in the \webquizrc. See \autoref{SS:layout} for
              more details.

      \end{description}

      \subsubsection*{Other options}
      \begin{description}
        \item[\ddash version] Prints the \WebQuiz version number and
        exit (currently \webquiz{version})

        \item[\ddash debugging] Displays extra debugging information
        when compiling and prevents \WebQuiz from deleting the many
        intermediary files that are created when building the quiz web
        pages.
      \end{description}

      \subsection{\WebQuiz settings and the webquizrc file}
      \label{SS:rcfile}\index{default settings}\index{webquizrc}

      \WebQuiz stores the following default sets in \webquizrc*, a
      \textit{run-time configuration file}:
    \begin{description}[nosep, labelwidth=18ex]
      \item[breadcrumbs] breadcrumbs at the top of quiz page (\autoref{SS:breadcrumbs})
      \item[department] name of department (\autoref{SS:breadcrumbs})
      \item[department-url] url for department (\autoref{SS:breadcrumbs})
      \item[engine] default tex engine used to compile web pages (\autoref{SS:classOptions})
      \item[hide-side-menu] do not display the side menu at start of quiz
            (\autoref{SS:classOptions})
      \item[institution] institution or university (\autoref{SS:breadcrumbs})
      \item[institution-url] url for institution or university (\autoref{SS:breadcrumbs})
      \item[language] default language used on web pages (\autoref{SS:classOptions})
      \item[one-page] display questions on one page (\autoref{SS:classOptions})
      \item[random-order] randomly order the quiz questions (\autoref{SS:classOptions})
      \item[theme] default colour theme used on web pages (\autoref{SS:classOptions})
      \item[version] webquiz version number for webquizrc settings
                  (\autoref{SS:rcfile})
      \item[webquiz-url] relative url for webquiz web directory (\autoref{SS:Initialise})
      \item[webquiz-www] full path to webquiz web directory (\autoref{SS:Initialise})
      \item[make4ht] build file for make4ht (\autoref{SS:Initialise})
      \item[mathjax] url for mathjax (\autoref{SS:Initialise})
      \item[webquiz-layout] name of \python module that formats the quizzes
      (\autoref{SS:layout})
    \end{description}
    The last three options are \textit{advanced options} that you should
    change with care.

    The default values of all of these settings can be overridden in the
    \LaTeX{} file for the  quiz, or with the command-line options. The
    default values can be changed at any time using
    \begin{bashcode}
        webquiz --edit-settings
    \end{bashcode}
    When changing the settings \WebQuiz tries to explain what it is
    doing at each step. If you are unsure what a particular setting does
    then \textit{press return} to accept the default value --- the
    default value is printed inside square brackets as
    \BashCode|[default]|. In particular, when you first start using
    \WebQuiz it is recommended that you accept the default values
    for all of the advanced options because it is very unlikely that you
    will need to change these initially.

    The first line of output from \BashCode|webquiz --settings| gives
    the location of the rc-file being used.  The system rc-file,
    \webquizrc*, is saved in the \BashCode|scripts/webquiz| subdirectory
    of the \BashCode|TEXMFLOCAL| directory. A typically location for
    this file is
    \begin{bashcode}
        /usr/local/texlive/texmf-local/scripts/webquiz/webquizrc
    \end{bashcode}
    By default, the \WebQuiz settings are saved here so that you do not
    need to reinitialise \WebQuiz whenever you update your \TeX{}
    distribution. If you do not have permission to write to this
    directory then you will be asked if you would like to save the
    rc-file somewhere else. The location of the user rc-file is:
    \begin{itemize}
        \item \BashCode|~/.dotfiles/config/webquizrc|
             if the directory \BashCode|~/.dotfiles/config| exists,
        \item \BashCode|~/.config/webquizrc|
             if the directory \BashCode|~/.config| exists,
        \item \BashCode|~/.webquizrc|  otherwise.
    \end{itemize}
    Each time \WebQuiz is run it first reads the system rc-file followed
    by the \textit{users rc-file}, if they exist. When using
    \BashCode|webquiz --edit-settings| you will be promoted for a
    different installation location if you do not have permission to
    write to the specified rc-file.  To use a particular \webquizrc*
    use the \BashCode|--rcfile| command-line option:
    \begin{bashcode}
        > webquiz --rcfile <full path to rc-file> ...
    \end{bashcode}
    If you save the settings to a non-standard location then you will
    need to use the command-line \BashCode|webquiz --rcfile| to access
    these settings.

  \noindent To describe the \WebQuiz defaults settings we consider them by category.

  \subsubsection*{Institution settings}
    \begin{description}

      \item[department]
        Sets the default department name. This can be overridden in
        the \LaTeX{} file using \LatexCode|\Department| in
        \autoref{SS:breadcrumbs}
        \CrossIndex{webquizrc}*{Department}

      \item[department-url]
        Sets the URL for the department. This can be overridden in
        the \LaTeX{} file using \LatexCode|\DepartmentURL| in
        \autoref{SS:breadcrumbs}
        \CrossIndex{webquizrc}*{DepartmentURL}

      \item[institution]
        Sets the default institution name. This can be overridden in the
        \LaTeX{} file using \LatexCode|\Institution| in
        \autoref{SS:breadcrumbs}
        \CrossIndex{webquizrc}*{Institution}

      \item[institution-url]
        Sets the URL for the institution.This can be overridden in
        the \LaTeX{} file using \LatexCode|\InstitutionURL| in
        \autoref{SS:breadcrumbs}
        \CrossIndex{webquizrc}*{InstitutionURL`}

      \end{description}

    \subsubsection*{Formatting options}

      \begin{description}

        \item[breadcrumbs]\index{breadcrumbs}
          Sets the default breadcrumbs at the top of quiz page. The
          default breadcrumbs can be overwritten in the quiz file using
          the \LatexCode{\BreadCrumbs} command. See
          \autoref{SS:breadcrumbs} for more details.
          \CrossIndex{webquizrc}*{BreadCrumbs}

        \item[engine] Sets the default \TeX{} engine to be used when
          compiling the quiz. By default, \BashCode|latex| is used, with
          the two other possibilities being \BashCode|lua| and
          \BashCode|xelatex|, for \Hologo{LuaLaTeX} and \hologo{XeLaTeX}
          respectively. Behind the scenes, the two choices
          correspond to invoking \ctan{make4ht} with the
          \BashCode|--lua| and \BashCode|--xelatex| options,
          respectively. The \BashCode|engine| setting in the
          \webquizrc* can be overridden by the webquizrc
          \BashCode|--latex|, \BashCode|--lua| and \BashCode|--xelatex|.
          \CrossIndex{webquizrc}{engine}
          \CrossIndex{webquizrc}{latex}
          \CrossIndex{webquizrc}{lualatex}
          \CrossIndex{webquizrc}{xelatex}

        \item[language]\index{language}
          Sets the default language for the \WebQuiz web pages. This can
          be overridden in the quiz file by using the document class
          \LatexCode|language| option: \LatexCode|language=xxx|
          See \autoref{SS:classOptions}.
          \CrossIndex{webquizrc}{language}

        \item[theme]\index{theme}
          Sets the default colour theme for the \WebQuiz web pages. This can
          be overridden in the quiz file by using the document class
          \LatexCode|theme| option: \LatexCode|theme=xxx|.
          See \autoref{SS:classOptions}.
          \CrossIndex{webquizrc}{theme}

      \end{description}

    \subsubsection*{Advanced options}

      \begin{description}

        \item[make4ht]\index{make4ht}
          Options for \ctan{make4ht}. Rather than using \TeXfht
          directly, \WebQuiz uses \ctan{make4ht} to convert the
          \LaTeX\ file to \XML. Use this option to
          customise how \ctan{make4ht} is called. See the
          \ctan[make4ht]{make4ht manual} for more information.
          \CrossIndex{webquizrc}{make4ht}

        \item[mathjax] \WebQuiz web pages use \href{https://www.mathjax.org/}{mathjax} to
          render the mathematics on the quiz web pages. By default this
          is done by loading \BashCode|mathjax| from
          \begin{bashcode}
             https://cdnjs.cloudflare.com/ajax/libs/mathjax/2.7.1/MathJax.js
          \end{bashcode}
          Fetching \BashCode|mathjax| from an external site can
          cause a delay when the quiz web pages are loaded. This setting
          in the rc-file allows you to change where \BashCode|mathjax| is
          loaded from. For example, if you install \BashCode|mathjax| on
          your web server then you would replace this will the
          corresponding relative URL.
          \CrossIndex{webquizrc}{mathjax}

        \item[webquiz-layout]
          Sets the \python file that writes the \HTML file for the quiz.
          Most people will not need this option. The next subsection
          describes how to do this in more detail.
          \CrossIndex{webquizrc}{layout}

      \end{description}

    \subsubsection*{Configuration settings}
       Use \BashCode|webquiz --initialise| to change these settings.

    \begin{description}

      \item[webquiz-url] This is the relative URL for \WebQuiz web directory
          \CrossIndex{webquizrc}{webquiz-url}

      \item[webquiz-www] This is the full path to the \WebQuiz web
      directory. The \OnlineManual and other example code can be found
      in the \BashCode|docs| subdirectory. If you use \BashCode|bash|,
      then the command
      \begin{bashcode}
          > cd $(webquiz --settings webquiz-www)/docs
      \end{bashcode}%$
      will take you to the \WebQuiz online \BashCode|docs/| directory.
      \CrossIndex{webquizrc}{webquiz-www}

    \end{description}

    \subsection{Changing the layout of the \WebQuiz web pages}
    \label{SS:layout}\index{layout}

    \begin{dangerous}
      \textit{This is an advanced \WebQuiz feature that most people will
      not need. To change the layout of the quiz web pages created by
      \WebQuiz requires working knowledge of \HTML and \python.}
    \end{dangerous}

    The construction of the online quizzes is controlled by the \python
    file \PythonCode|webquiz_standard.py|. If you want to change the
    structure of the web pages for the quizzes then the easiest way to
    do this is to make a copy of \PythonCode|webquiz_standard.py|, say
    \PythonCode|webquiz_myformat.py|, and then edit this file directly.
    This will require working knowledge of \python and \HTML. To give
    you an idea of what needs to be done, the \python file
    \PythonCode|webquiz_standard| contains a single function
    \PythonCode|write_web_page| that returns the \HTML for the page as a
    string using the following:
    \begin{htmlcode}
      quiz_page = r'''<!DOCTYPE HTML>
      <html lang="en">
      <head>
        <title> {title} </title>
        {htmlpreamble}
      </head>
      <body>
        {no_script}{breadcrumbs}
        <div class="quiz-page">
          {side_menu}
          <div class="quiz-questions">
            {quiz_header}
            {quiz_questions}
          </div>
        </div>
        {webquiz_init}
      </body>
      </html>
      '''
    \end{htmlcode}
    By changing this output you can change the layout of the quizzes
    produced by \WebQuiz. For example, by adding code to the
    \HTMLCode|<head>...</head>| section of \HTMLCode|quiz_page| it is
    easy to include new \CSS code and by modifying
    \HTMLCode|<body>....</body>| you can change the layout of the page.
    More sophisticated versions of \PythonCode|webquiz_standard.py|,
    where you change the underlying \python code, are possible.  At the
    University of Sydney we have a custom version of
    \PythonCode|webquiz_standard.py| that calls our content management
    system and, in this way, embeds the quiz web page inside a web page
    that used the official ``branding'' required by our university.

    When experimenting with a new layout can run
    \WebQuiz using the command:
    \begin{bashcode}
       > webquiz --webquiz-layout webquiz_myformat quizfile.tex
    \end{bashcode}
    Once the new layout is finalised you can make it the default
    layout by setting \BashCode|webquiz-layout| equal  to \BashCode|webquiz_my_format|
    in the \webquizrc using \BashCode|webquiz --edit-settings|.
    \index{edit-settings}\index{webquiz-layout}

    If you do make modifications to any of these files then, by the
    \WebQuiz Licensing agreement, you are required to create a new
    version of this file that has a \textit{different name}. Doing this
    will also make it easier for you to integrate your changes with any
    future releases of \WebQuiz.

  \subsection{Bugs, issues and feature requests}\index{bug reports}
    Please report any bugs, issues or feature requests using the
    \textit{issue} tracker at
    \begin{quote}
    \href{https://\webquiz{repository}}{\webquiz{repository}}.
    \end{quote}
    When reporting a bug please provide a \textit{minimal working example}
    that clearly demonstrates your problem. This should be a compilable
    \LaTeX file that looks something like the following:
    \begin{latexcode}
       \documentclass{webquiz}
       \begin{document}
            ** insert problematic code here **
       \end{document}
    \end{latexcode}
    Bug reports that do not have a minimal working example can be
    hard to reproduce in which case it is not possible to fix them.
    Before submitting a bug export please first compile your quiz using
    (pdf)latex to check to see if your problem is an issue with \LaTeX
    or with \WebQuiz. If you can, please also test to see if your code
    compiles using \TeXfht directly.

\appendix
\section*{Appendices}
  \addcontentsline{toc}{section}{Appendices}
\renewcommand\thesection{}
\renewcommand\thesubsection{\Alph{subsection}}

  \subsection{Catalogue of web page themes}\index{theme}\label{SS:themes}
  \WebQuiz{} comes with different themes for changing the colour scheme
  of the online quizzes, which can be set using the \LatexCode|theme|
  document-class option or in the \webquizrc; see
  \autoref{SS:classOptions} for more details. Themes are easy to
  construct in principle although finding colours that work well together
  can be tricky in practice so, as a result, there are some themes that
  I would not personally recommend!

  New themes, and modifications to existing themes, can be submitted as
  an \textit{issue} at:
  \begin{quote}
  \href{https://\webquiz{repository}/issues}{\webquiz{repository}/issues}.
  \end{quote}
  Where possible these will be incorporated into a future release of the
  package, although there is a potential technical issue here in that
  the underlying \CSS files are written using
  \href{http://sass-lang.com/}{sass}.

  \ShowcaseThemes

\subsection{The online \WebQuiz manual}\label{S:online}

  \WebQuiz has an \OnlineManual* that is a \LaTeX file written with the
  \LatexCode|webquiz| document class. The conversion of the manual from
  \LaTeX{} to \HTML is done by \WebQuiz. The \BashCode|PDF| version of
  this manual is included here as an easy reference.  The source file
  for the \OnlineManual* is included in the documentation of \WebQuiz to
  allow you to create a local version of the \OnlineManual*. Look for the
  file \BashCode|webquiz-online-manual.tex| in the
  \BashCode|webquiz-www/docs| directory; see~\autoref{SS:rcfile}.

  The online manual can either be compiled as a PDF file (see below),
  or using \WebQuiz to produce an online version of the manual.
  The \OnlineManual* was written for ``internal use'' when \WebQuiz was
  first written in 2004. \WebQuiz has evolved quite a lot since then.
  There is some overlap between the \OnlineManual* and previous sections,
  however, the \OnlineManual* only describes how to typeset questions and
  it does not cover some of the more recent features of \WebQuiz, such
  as the document class options, or how to use the program.  If there
  are any discrepancies between the \OnlineManual* and the earlier
  sections of this manual then the \OnlineManual* should be discounted.

  The \OnlineManual* has diagrams that are drawn using \ctan{pstricks}
  and, as a result, to create a \BashCode|PDF| version of the \OnlineManual*
  use \BashCode|latex webquiz-online-manual| to create a
  \BashCode|dvi| file. The \BashCode|dvi| file can be  converted to
  \BashCode|PDF| using \BashCode|dvipdf|. The online manual needs to be
  compiled using \BashCode|latex| rather than \BashCode|pdflatex|,
  which will generate errors.

  \includepdf[pages=-,pagecommand={\pagestyle{webquiz}}]{webquiz-online-manual}


\subsection{Licence}

Version 3, Copyright (C) 2007,

\href{https://www.gnu.org/licenses/gpl-3.0.en.html}
     {GNU General Public License, Version 3, 29 June 2007}

This program is free software: you can redistribute it and/or modify it under
the terms of the GNU General Public License (GPL) as published by the Free
Software Foundation, either version 3 of the License, or (at your option) any
later version.

This program is distributed in the hope that it will be useful, but WITHOUT ANY
WARRANTY; without even the implied warranty of MERCHANTABILITY or FITNESS FOR A
PARTICULAR PURPOSE.  See the GNU General Public License for more details.


\vfil
\begin{tabular}{@{}ll}
Authors             & \webquiz{authors}\\
Description         & \webquiz{description}\\
Maintainer          & \webquiz{name}\\
System requirements & \webquiz{requirements}\\
Licence             & \webquiz{licence}\\
Release date        & \webquiz{release date}\\
Repository          & \href{https://\webquiz{repository}}{\webquiz{repository}}
\end{tabular}
\eject

\printindex

\end{document}
