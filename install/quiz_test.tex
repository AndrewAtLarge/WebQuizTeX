\documentclass{mathquiz}
\DeclareMathOperator{\cis}{cis}
\newcommand{\R}{\mathbb R}
\newcommand{\C}{\mathbb C}

\title{Quiz 1: Numbers and sets}

\begin{document}

% \begin{discussion}
% This quiz is concerned with set notation, and with the arithmetic of complex numbers. 
% 
% Recall
% that a \textit{complex number} is a number of the form $a+ib$, where
% $i^2=-1$ and $a,b\in\R$. The addition/subtract, multiplication and
% division laws for complex numbers are as follows:\medskip
% 
% \begin{tabular}{lp{100mm}}
% \textbf{Addition/subtraction} & $(a+ib)\pm(c+id)=(a\pm c)+i(b\pm d)$\\
% \textbf{Multiplication} & $(a+ib)(c+id)%=ac+iad+ibc+i^2bd
%                                        =(ac-bd)+i(ad+bc)$\\
% \textbf{Division} &If $c+id\ne0$ then
%            $\dfrac{a+ib}{c+id}%=\dfrac{a+ib}{c+id}\dfrac{c-id}{c-id}
%                                      =\dfrac{(a+ib)(c-id)}{c^2+d^2}$.
% \end{tabular}
% 
% In addition, recall that the \textit{complex conjugate} of
% $z=a+ib$ is the complex number $\bar z=a-ib$ and the 
% \textit{modulus} of $z$ is the real number $|z|=\sqrt{a^2+b^2}$.
% \end{discussion}


%%% 1
\begin{question}
Which of the following are correct ways of writing the set
\[ A=\{x \in {\mathbb R}\ |\ -3 < x \leq -1\ \rm{or}\ x\geq 0\}\ \rm{?}\]
\begin{choice}
\incorrect $(-3,\infty)$
\response $(-3,\infty)$ includes the real numbers between $-1$ and $0$, \\
which do
not belong to $A$.

\incorrect $[-3,\infty)$
\response $[-3,\infty)$ includes $-3$, as well as 
the real numbers between $-1$ and $0$, none of which
 belong to $A$.

\incorrect $[-3,-1]\cap[0,\infty)$
\response $[-3,-1]\cap[0,\infty)$ is the empty set $\emptyset$. As $A$ is not empty (for example, $A$ includes $-1$), this option can't be correct. 

\incorrect $(-3,-1)\cup[0,\infty)$
\response $-1$ is not in $(-3,-1)\cup[0,\infty)$, but $-1$ is in $A$.

\correct $(-3,-1]\cup[0,\infty)$

\end{choice}
\end{question}

%%%% 2
\begin{question}
What is another way of writing the set
\[B= \{x \in {\mathbb R}\ |\ |x-3|<2 \}\ \rm{?}\]
\begin{choice}
\incorrect $(2,3]$
\response For example, $4$ belongs to $B$ but is not in $(2,3]$.

\incorrect $[2,4]$
\response For example, $1.5$ belongs to $B$ but is not in $[2,4]$.

\correct $(1,5)$

\response $B$ is the set of all points whose distance from 3 on the
number line is less than 2. \\
The solution to $|x-3|<2$ is $1<x<5$.

\incorrect $[1,5]$
\response Neither $1$ nor $5$  belong to $B$, but both $1$ and $5$ belong to $[1,5]$.

\incorrect $[2,3)$
\response For example, $4$ belongs to $B$ but is not in $[2,3)$.

\end{choice}
\end{question}


\end{document}
